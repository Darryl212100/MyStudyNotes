\documentclass[12pt]{article}

% --- 基本宏包 ---
\usepackage{geometry}
\usepackage{amsmath}
\usepackage{amssymb}

\usepackage{amsthm}
\usepackage{geometry}
\usepackage{graphicx}
\usepackage{tikz}     
% --- 页面设置 ---
\geometry{a4paper, margin=1in, headsep=1cm}
\linespread{1.1} % 增加行间距,提高可读性

% --- 文档标题 ---
\title{Variational Principles and Lagrange's Equations}
\author{Darryl}
\date{\today}


% --- 文档开始 ---
\begin{document}
	
	\maketitle
	
	\section*{2.1 Hamilton's Principle}
	We call the integral of the Lagrangian over time the \textbf{action}, denoted by $I$.
	\begin{equation*}
		I = \int_{t_1}^{t_2} L \, dt
	\end{equation*}
	Hamilton's Principle states that the action has a \textbf{stationary value} for the actual path of motion. This means that the variation of the action is zero:
	\begin{equation*}
		\delta I = \int_{t_1}^{t_2} \delta L \, dt = 0
	\end{equation*}
	
	Let's show that Lagrange's equations imply a stationary action. The variation of the Lagrangian is:
	\begin{equation*}
		\delta L = \sum_j \left( \frac{\partial L}{\partial q_j} \delta q_j + \frac{\partial L}{\partial \dot{q}_j} \delta \dot{q}_j \right)
	\end{equation*}
	Given that $\delta \dot{q}_j = \frac{d}{dt}(\delta q_j)$, the variation of the action becomes:
	\begin{align*}
		\delta I &= \int_{t_1}^{t_2} \sum_j \left( \frac{\partial L}{\partial q_j} \delta q_j + \frac{\partial L}{\partial \dot{q}_j} \frac{d}{dt}(\delta q_j) \right) dt \\
		&= \int_{t_1}^{t_2} \sum_j \frac{\partial L}{\partial q_j} \delta q_j \, dt + \int_{t_1}^{t_2} \sum_j \frac{\partial L}{\partial \dot{q}_j} d(\delta q_j)
	\end{align*}
	Integrating the second term by parts:
	\begin{align*}
		\delta I &= \int_{t_1}^{t_2} \sum_j \frac{\partial L}{\partial q_j} \delta q_j \, dt + \left[ \sum_j \frac{\partial L}{\partial \dot{q}_j} \delta q_j \right]_{t_1}^{t_2} - \int_{t_1}^{t_2} \sum_j \frac{d}{dt}\left(\frac{\partial L}{\partial \dot{q}_j}\right) \delta q_j \, dt \\
		&= \int_{t_1}^{t_2} \sum_j \left( \frac{\partial L}{\partial q_j} - \frac{d}{dt}\left(\frac{\partial L}{\partial \dot{q}_j}\right) \right) \delta q_j \, dt + \left[ \sum_j \frac{\partial L}{\partial \dot{q}_j} \delta q_j \right]_{t_1}^{t_2}
	\end{align*}
	The variations at the endpoints are zero, $\delta q_j(t_1) = \delta q_j(t_2) = 0$, so the boundary term vanishes. If we now use Lagrange's equation, the term inside the integral is zero:
	\begin{equation*}
		\frac{\partial L}{\partial q_j} - \frac{d}{dt}\left(\frac{\partial L}{\partial \dot{q}_j}\right) = 0
	\end{equation*}
	Therefore, it follows that
	\begin{equation*}
		\implies \delta I = 0
	\end{equation*}
	
	
	\section*{2.2 Techniques of the Calculus of Variations}
	Let's consider a one-dimensional problem where we want to find the path $y(x)$ between two points $(x_1, y_1)$ and $(x_2, y_2)$ that extremizes the integral:
	\begin{equation*}
		J = \int_{x_1}^{x_2} f(y, y', x) \, dx, \quad \text{where } y' = \frac{dy}{dx}
	\end{equation*}
	We define a varied path $y(x, \alpha) = y(x, 0) + \alpha \eta(x)$, where $y(x,0)$ is the true path and $\eta(x)$ is an arbitrary function such that $\eta(x_1) = \eta(x_2) = 0$.
	To obtain a stationary point, we require $\left( \frac{dJ}{d\alpha} \right)_{\alpha=0} = 0$.
	\begin{align*}
		\frac{dJ}{d\alpha} &= \int_{x_1}^{x_2} \left( \frac{\partial f}{\partial y} \frac{\partial y}{\partial \alpha} + \frac{\partial f}{\partial y'} \frac{\partial y'}{\partial \alpha} \right) dx \\
		&= \int_{x_1}^{x_2} \frac{\partial f}{\partial y} \eta(x) \, dx + \left[ \frac{\partial f}{\partial y'} \eta(x) \right]_{x_1}^{x_2} - \int_{x_1}^{x_2} \frac{d}{dx}\left(\frac{\partial f}{\partial y'}\right) \eta(x) \, dx
	\end{align*}
	The boundary term vanishes because $\eta(x_1)=\eta(x_2)=0$.
	\begin{equation*}
		\implies \frac{dJ}{d\alpha} = \int_{x_1}^{x_2} \left( \frac{\partial f}{\partial y} - \frac{d}{dx}\left(\frac{\partial f}{\partial y'}\right) \right) \eta(x) \, dx = 0
	\end{equation*}
	For this integral to be zero for any arbitrary continuous function $\eta(x)$, the integrand in the parenthesis must be identically zero. This gives the \textbf{Euler-Lagrange equation}:
	\begin{equation}
		\boxed{
			\frac{\partial f}{\partial y} - \frac{d}{dx}\left(\frac{\partial f}{\partial y'}\right) = 0
		}
	\end{equation}
	
	
	\section*{ Examples of Variational Problems}
	\begin{enumerate}
		\item \textbf{Shortest distance between two points in a plane} \\
		An element of length is $ds = \sqrt{dx^2+dy^2}$. We want to minimize $I = \int ds = \int \sqrt{1+y'^2} \, dx$.
		The function is $f = \sqrt{1+y'^2}$. Since $\frac{\partial f}{\partial y} = 0$, the Euler-Lagrange equation simplifies to $\frac{d}{dx}(\frac{\partial f}{\partial y'}) = 0$.
		This implies $\frac{\partial f}{\partial y'} = \frac{y'}{\sqrt{1+y'^2}} = C$ (a constant).
		Solving for $y'$ gives $y' = a$ (another constant), and integrating gives the equation of a straight line:
		\begin{equation*}
			y = ax + b
		\end{equation*}
		
		\item \textbf{Minimum surface of revolution} \\
		The area of a strip of the surface is $2\pi x \, ds = 2\pi x \sqrt{1+y'^2} \, dx$.
		The function is $f = x\sqrt{1+y'^2}$. The Euler-Lagrange equation is $\frac{\partial f}{\partial y} - \frac{d}{dx}(\frac{\partial f}{\partial y'}) = 0$.
		Since $\frac{\partial f}{\partial y} = 0$, we have $\frac{\partial f}{\partial y'} = \frac{x y'}{\sqrt{1+y'^2}} = a$ (a constant).
		Solving gives $\frac{dy}{dx} = \frac{a}{\sqrt{x^2-a^2}}$. Integrating this yields:
		\begin{equation*}
			y = a \cosh^{-1}\left(\frac{x}{a}\right) + b
		\end{equation*}
		which is the equation of a catenary.
		

        \begin{figure}[h]
        	\centering
        	\includegraphics[width=0.7\linewidth]{figure4}
          	\caption{}
	        \label{fig:figure4}
        \end{figure}

		\item \textbf{The Brachistochrone Problem} \\
		Find the path between two points that a particle under gravity will follow in the shortest time.
		Time is $t = \int dt = \int \frac{ds}{v}$. From conservation of energy, $v=\sqrt{2gy}$. The path length is $ds=\sqrt{1+y'^2}dx$.
		We must minimize $t_{12} = \int \frac{\sqrt{1+y'^2}}{\sqrt{2gy}} \, dx$.
		The function $f = \frac{\sqrt{1+y'^2}}{\sqrt{y}}$ is independent of $x$. For such cases, a first integral of the Euler-Lagrange equation exists: $f - y'\frac{\partial f}{\partial y'} = C$.
		This leads to the relation $y(1+y'^2) = \frac{1}{2gC^2} = \text{const}$.
		Using the substitution $y' = \cot(\theta/2)$, we find the solution to be a cycloid:
		\begin{equation*}
			x = a(\theta - \sin\theta), \quad y = a(1 - \cos\theta)
		\end{equation*}
		

        \begin{figure}[h]
        	\centering
        	\includegraphics[width=0.7\linewidth]{figure5}
	        \caption{}
        	\label{fig:figure5}
        \end{figure}
	\end{enumerate}
	
	% 2.3 Derivation of Lagrange's Equations from Hamilton's Principle
	\section*{2.3 Derivation of Lagrange's Equations from Hamilton's Principle}
	
	The variation of the integral
	$$ \delta J = \delta \int_a^b f(y_1, y_2, \dots, y_n, y'_1, y'_2, \dots, y'_n, x) dx $$
	To consider $J$ as a function of parameter $\alpha$, we introduce
	$$ y_i(x, \alpha) = y_i(x, 0) + \alpha \eta_i(x) $$
	where $\eta_i(x)$ are continuous through the second derivative.
	$$ \frac{\partial J}{\partial \alpha} = \int_a^b \sum_i (\frac{\partial f}{\partial y_i} \frac{\partial y_i}{\partial \alpha} + \frac{\partial f}{\partial y'_i} \frac{\partial y'_i}{\partial \alpha}) dx $$
	We integrate by parts the integral (the second term)
	$$ \int_a^b \frac{\partial f}{\partial y'_i} \frac{\partial^2 y_i}{\partial \alpha \partial x} dx = [\frac{\partial f}{\partial y'_i} \frac{\partial y_i}{\partial \alpha}]_a^b - \int_a^b \frac{d}{dx}(\frac{\partial f}{\partial y'_i}) \frac{\partial y_i}{\partial \alpha} dx $$
	Since all curves pass through the fixed end points, the first term vanishes. $\delta J$ becomes
	$$ \delta J = \int_a^b \sum_i (\frac{\partial f}{\partial y_i} - \frac{d}{dx} \frac{\partial f}{\partial y'_i}) \delta y_i dx $$
	where $\delta y_i = (\frac{\partial y_i}{\partial \alpha})_0 d\alpha$ and $\delta y_i|_a = \delta y_i|_b = 0$.
	$$ \delta J = (\frac{\partial J}{\partial \alpha})_0 d\alpha $$
	Since $y_i$ are independent and so are $\delta y_i$.
	$$ \Rightarrow \frac{\partial f}{\partial y_i} - \frac{d}{dx}(\frac{\partial f}{\partial y'_i}) = 0, \quad i=1, 2, \dots, n $$
	Now, $x \to t$
	$$ y_i \to q_i $$
	$$ f(y_i, y'_i, x) \to L(q_i, \dot{q}_i, t) $$
	$$ \Rightarrow \frac{d}{dt}(\frac{\partial L}{\partial \dot{q}_i}) - \frac{\partial L}{\partial q_i} = 0 $$
	
	% 2.4 Extending Hamilton's Principle to systems with Constraints
	\section*{2.4 Extending Hamilton's Principle to systems with Constraints}
	
	Let's consider holonomic constraints, to eliminate the extra virtual displacements, we use the method of Lagrange undetermined multipliers.
	We modify the action
	$$ I = \int_{t_1}^{t_2} (L + \sum_{\alpha=1}^{m} \lambda_{\alpha} f_{\alpha}) dt $$
	where $q_k$ and $\lambda_{\alpha}$ vary independently to obtain $n+m$ equations.
	$$ \delta I = \int_{t_1}^{t_2} \sum_{k=1}^{n} (\frac{\partial L}{\partial q_k} - \frac{d}{dt}\frac{\partial L}{\partial \dot{q}_k}) \delta q_k dt + \int_{t_1}^{t_2} \sum_{\alpha=1}^{m} \sum_{k=1}^{n} \lambda_{\alpha} \frac{\partial f_{\alpha}}{\partial q_k} \delta q_k = 0 $$
	We choose $\lambda_{\alpha}$'s so that $m$ equations are satisfied for arbitrary $\delta q_i$ and then choose variations of the $\delta q_i$ in the remaining $n-m$ equations. Thus we obtain
	$$ \frac{d}{dt} \frac{\partial L}{\partial \dot{q}_k} - \frac{\partial L}{\partial q_k} + \sum_{\alpha=1}^{m} \lambda_{\alpha} \frac{\partial f_{\alpha}}{\partial q_k} = 0, \quad \text{for } k=1, \dots, n, n+1, \dots $$
	We rewrite it as:
	$$ \frac{d}{dt}(\frac{\partial L}{\partial \dot{q}_k}) - \frac{\partial L}{\partial q_k} = \sum_{\alpha=1}^{m} \lambda_{\alpha} \frac{\partial f_{\alpha}}{\partial q_k} = Q_k $$
	where the $Q_k$ are generalized forces.
	
	\subsection*{Example}
	A smooth solid hemisphere of radius $a$ placed with its flat side down and fastened to the Earth whose gravitational acceleration $g$. Place a small mass $m$ at the top of it with an infinitesimal displacement off center so the mass slides down without friction. The hemisphere is vertical to $z$ and $x-z$ plane contains the initial motion of mass.
	
	Let $\theta$ be the angle from the top of the sphere to the mass.
	The Lagrangian:
	$$ L = \frac{1}{2} m (\dot{x}^2 + \dot{y}^2 + \dot{z}^2) - mgz $$
	(ignore $y$)
	$$ L = \frac{1}{2} m a^2 \dot{\theta}^2 - mga \cos\theta \quad (x = a \sin\theta, z = a \cos\theta) $$
	The constraint equation is
	$$ a - \sqrt{x^2+z^2} = 0 $$
	Expressing it in terms of $x^2+z^2=r^2$ and $\tan\theta = x/z$.
	Lagrange's equations are
	$$ m a \ddot{\theta}^2 - Mg \cos\theta + \lambda = 0 $$
	$$ m a^2 \ddot{\theta} + Mg a \sin\theta = 0 $$
	$$ \Rightarrow \ddot{\theta}^2 = \frac{2g}{a} \cos\theta + \frac{2\lambda}{ma} \quad \text{and} \quad \lambda = Mg(3\cos\theta - 2) $$
	Tip:
	$$ \frac{d}{dt}(\frac{\partial L}{\partial q_k}) - \frac{\partial L}{\partial q_k} + \sum_{\alpha=1}^{m} \lambda_{\alpha} \frac{\partial f_{\alpha}}{\partial q_k} = 0 $$
	or
	$$ \frac{d}{dt}(\frac{\partial L}{\partial \dot{q}_i}) - \frac{\partial L}{\partial q_i} = \lambda \frac{\partial \phi}{\partial q_i} \quad (q_i = x, z, \theta) $$
	where $\phi = \sqrt{x^2+z^2} + a = 0$ (equations of constraints)
	Since $\lambda=0$ when $\theta = \cos^{-1}(\frac{2}{3})$, the mass leaves the sphere at that angle.
	
	For semi-holonomic constraints
	$$ f_{\alpha}(q_1, \dots, q_n, \dot{q}_1, \dots, \dot{q}_n, t) = 0; \quad \alpha = 1, 2, \dots, n $$
	$$ \Rightarrow f_{\alpha} = \sum_{k=1}^{n} a_{\alpha k} \dot{q}_k + a_0 = 0 $$
	$$ \delta \int_{t_1}^{t_2} (L + \sum_{\alpha=1}^{m} \mu_{\alpha} f_{\alpha}) dt = 0 $$
	where $\mu$ is to distinguish these multiplies from the holonomic.
	We assume $\mu_{\alpha} = \mu_{\alpha}(t)$
	$$ \Rightarrow \frac{d}{dt}(\frac{\partial L}{\partial \dot{q}_k}) - \frac{\partial L}{\partial q_k} = \sum_{\alpha=1}^{m} \mu_{\alpha} \frac{\partial f_{\alpha}}{\partial \dot{q}_k} $$
	
	\subsection*{Example (figure 8, page 150)}
	The two generalized coordinates are $r, \theta$.
	And the equation of constraints is
	$$ r d\theta = dx $$
	$$ T = \frac{M}{2}\dot{x}^2 + \frac{M}{2}r^2\dot{\theta}^2, \quad V=Mg((L-x)\sin\phi) $$
	$$ L = T-V $$
	Tip: the coefficients in the constraint
	for $f(q_i, \dot{q}_i, \dots, q_n, t) = 0$
	$$ \Rightarrow \sum_k \frac{\partial f}{\partial q_k} dq_k + \frac{\partial f}{\partial t} dt = 0 $$
	$$ a_k = \frac{\partial f}{\partial q_k}, \quad a_t = \frac{\partial f}{\partial t} $$
	$$ \Rightarrow a_\theta = r, \quad a_x = -1 $$
	Thus, the Lagrange's equations are
	$$ M\ddot{x} - Mg\sin\phi + \lambda = 0 $$
	$$ Mr^2\ddot{\theta} - \lambda r = 0 $$
	And $r\dot{\theta} = \dot{x}$ differentiating it with respect to time $r\ddot{\theta} = \ddot{x}$
	Hence, we have
	$$ M\ddot{x} = \lambda \quad \text{and} \quad \ddot{x} = \frac{g\sin\phi}{2} \quad \text{with} \quad \lambda = \frac{Mg\sin\phi}{2} $$
	and
	$$ \ddot{\theta} = \frac{g\sin\phi}{2r} $$
	The friction of constraint is
	$$ \lambda = \frac{Mg\sin\phi}{2} $$
	From $\ddot{x} = v \frac{dv}{dx}$ we obtain the velocity reaching the bottom:
	$$ v = \sqrt{gL\sin\phi} $$
	
	% 2.5 Advantages of a Variational Principle Formulation
	\section*{2.5 Advantages of a Variational Principle Formulation}
	
	As stated, Hamilton's principle is contained for all of the mechanics of holonomic systems with forces derivable from potentials.
	As the variation at the end points is zero, the addition of the arbitrary time derivative to the Lagrangian doesn't affect the variational behaviour of the integral.
	The Lagrangian formulation can also be extended to describe systems that are not normally considered in dynamics.
	
	\subsection*{Example}
	$$ T = \frac{1}{2} L \dot{q}^2, \quad \mathcal{F} = \frac{1}{2} R \dot{q}^2, \quad U = qV $$
	The equation of motion:
	$$ V = L\ddot{q} + R\dot{q} = L\dot{I} + RI $$
	where $I = \dot{q}$ is the electric current.
	A solution for a battery connected to the current at time $t=0$:
	$$ I = I_0(1-e^{-Rt/L}) $$
	where $I_0 = V/R$ is the final steady current flow.
	The mechanical analog is a sphere of radius $a$ and effective mass $m'$ falling in a viscous fluid of constant density and viscosity $\eta$ under the force of gravity. The direction of motion is along the $y$ axis.
	$$ \Rightarrow T = \frac{1}{2}m'\dot{y}^2, \quad \mathcal{F} = 3\pi\eta a \dot{y}^2, \quad U = m'gy, \quad F_f = 6\pi\eta a \dot{y} $$
	The Lagrange's equations give:
	$$ m'\ddot{y} = m'g - 6\pi\eta a \dot{y} $$
	Using $v = \dot{y}$ (the motion starts from rest at $t=0$)
	$$ v = v_0(1-e^{-t/\tau}) \quad \text{where} \quad \tau = m'/(6\pi\eta a) \quad \text{and} \quad v_0 = m'g/(6\pi\eta a) $$
	
	\subsection*{Example}
	An inductance, $L$, is in series with a capacitance, $C$.
	The capacitor acts as a source of potential energy given by $q^2/C$.
	The Lagrangian produces the equation
	$$ L\ddot{q} + \frac{q}{C} = 0 \quad \Rightarrow \quad \text{solution: } q=q_0 \cos\omega_0 t $$
	where $q_0$ is the charge in the capacitor at $t=0$.
	The quantity $\omega_0 = \frac{1}{\sqrt{LC}}$ is the resonant frequency.
	The analog in mechanics is the simple harmonic oscillator where
	$$ L = \frac{1}{2}m\dot{x}^2 - \frac{1}{2}kx^2 $$
	$$ \Rightarrow m\ddot{x} + kx = 0 \quad \Rightarrow \quad x = x_0 \cos\omega_0 t \quad \text{with} \quad \omega_0 = \sqrt{k/m} $$
	$L \to$ mass
	$R \to$ Stokes' Law type of frictional drag
	$\frac{1}{C} \to$ Hooke's law spring constant
	
	A system of coupled circuits:
	$$ L = \frac{1}{2}\sum_{j,k} L_{jk} \dot{q}_j \dot{q}_k + \frac{1}{2}\sum_{j,k} M_{jk} \dot{q}_j \dot{q}_k - \frac{1}{2}\sum_j \frac{q_j^2}{C_j} + \sum_j e_j(t)q_j $$
	and a dissipation function
	$$ \mathcal{F} = \frac{1}{2}\sum_j R_j \dot{q}_j^2 $$
	The Lagrange equations are
	$$ L_j \frac{d^2q_j}{dt^2} + \sum_{j,k} M_{jk} \frac{d^2q_k}{dt^2} + R_j \frac{dq_j}{dt} + \frac{q_j}{C_j} = E_j(t) $$
	where $E_j(t)$ are external emf's.
	
	
    \begin{figure}[h]
	    \centering
	    \includegraphics[width=0.7\linewidth]{figure6}
    	\caption{}
	    \label{fig:figure6}
    \end{figure}
    
    
    \section*{2.6 Conservation Theorems and Symmetry Properties}
    \begin{enumerate}
    	\item Let's consider a system of mass points under the influence of forces derived from potentials dependent on position only. Then
    	$$ \frac{\partial L}{\partial \dot{x}_i} = \frac{\partial T}{\partial \dot{x}_i} - \frac{\partial V}{\partial \dot{x}_i} = \frac{\partial T}{\partial \dot{x}_i} = \frac{\partial}{\partial \dot{x}_i} \sum_i \frac{1}{2} m_i (\dot{x}_i^2 + \dot{y}_i^2 + \dot{z}_i^2) = m_i \dot{x}_i = p_{ix} $$
    	Hence, we define the generalized momentum as
    	$$ P_j = \frac{\partial L}{\partial \dot{q}_j} $$
    	or can be called as canonical momentum.
    	
    	For example, in the electromagnetic field
    	$$ L = \sum_i \frac{1}{2} m_i \dot{\vec{r}}_i^2 - \sum_i q_i \phi(\vec{r}_i) + \sum_i q_i \vec{A}(\vec{r}_i) \cdot \vec{v}_i $$
    	$$ \Rightarrow P_{ix} = \frac{\partial L}{\partial \dot{x}_i} = m_i \dot{x}_i + q_i A_x $$
    	
    	If L of a system doesn't contain a given coordinate $q_j$, then the coordinate is said to be cyclic or ignorable.
    	For a cyclic coordinate, the Lagrange equation of motion
    	$$ \frac{d}{dt} \left( \frac{\partial L}{\partial \dot{q}_j} \right) = 0 \quad \left( \frac{\partial L}{\partial q_j} = 0 \right) \quad \text{or} \quad \frac{d P_j}{dt} = 0 $$
    	which means that $P_j = \text{constant}$.
    	Hence, the generalized momentum conjugate to a cyclic coordinate is conserved.
    	
    	\item If V is not a function of the velocities, then
    	$$ \frac{\partial V}{\partial q_j} = -Q_j $$
    	$$ P_j = \frac{\partial T}{\partial \dot{q}_j} = - \frac{\partial V}{\partial q_j} = Q_j = \sum_i \vec{F}_i \cdot \frac{\partial \vec{r}_i}{\partial q_j} $$
    	Be definition, we have
    	$$ \frac{\partial \vec{r}_i}{\partial q_j} = \lim_{\delta q_j \to 0} \frac{\vec{r}_i(q_j + \delta q_j) - \vec{r}_i(q_j)}{\delta q_j} = \frac{d\vec{r}_i}{dq_j} = \vec{n} $$
    	where $\vec{n}$ is unit vector along the direction of the translation.
    	
    	\begin{figure}[h]
    		\centering
    		\includegraphics[width=0.7\linewidth]{figure7}
    		\caption{}
    		\label{fig:figure7}
    	\end{figure}
    	
    	Note that $T = \frac{1}{2} \sum_i m_i \dot{\vec{r}}_i^2$.
    	The conjugate momentum is $P_j = \frac{\partial T}{\partial \dot{q}_j} = \sum_i m_i \dot{\vec{r}}_i \cdot \frac{\partial \dot{\vec{r}}_i}{\partial \dot{q}_j}$.
    	$$ \left( \text{Tip: } \frac{\partial \dot{\vec{r}}_i}{\partial \dot{q}_j} = \frac{\partial \vec{r}_i}{\partial q_j} \right) $$
    	Combining above, we have $P_j = \vec{n} \cdot \sum_i m_i \dot{\vec{r}}_i$.
    	
    	For a cyclic coordinate, where $q_j$ doesn't appear in V, therefore
    	$$ -\frac{\partial V}{\partial q_j} = Q_j = 0 = P_j $$
    	which is the conservation theorem for linear momentum.
    	
    	\item In a similar fashion, if $dq_j$ correspond to a rotation of the system of particles around some axis. 
    	 
        \begin{figure}[h]
        	\centering
        	\includegraphics[width=0.7\linewidth]{figure8}
        	\caption{}
        	\label{fig:figure8}
        \end{figure}
        
    	Obviously, $|d\vec{r}_i| = r_i \sin\theta \, dq_j$ and $|\frac{\partial \vec{r}_i}{\partial q_j}| = r_i \sin\theta$.
    	Since $\frac{\partial \vec{r}_i}{\partial q_j}$ is perpendicular to both $\vec{r}_i$ and $\vec{n}$, it can be written as:
    	$$ \frac{\partial \vec{r}_i}{\partial q_j} = \vec{n} \times \vec{r}_i $$
    	The generalized force becomes: $Q_j = \sum_i \vec{F}_i \cdot (\vec{n} \times \vec{r}_i)$ reducing to $Q_j = \vec{n} \cdot \sum_i (\vec{r}_i \times \vec{F}_i) = \vec{n} \cdot \sum_i \vec{N}_i = \vec{n} \cdot \vec{N}$.
    	$$ (\text{Tip: } \vec{a} \cdot (\vec{b} \times \vec{c}) = \vec{b} \cdot (\vec{c} \times \vec{a}) = \vec{c} \cdot (\vec{a} \times \vec{b})) $$
    	$$ P_j = \frac{\partial T}{\partial \dot{q}_j} = \sum_i m_i \dot{\vec{r}}_i \cdot \frac{\partial \vec{r}_i}{\partial q_j} = \sum_i m_i \dot{\vec{r}}_i \cdot (\vec{n} \times \vec{r}_i) = \vec{n} \cdot \sum_i (\vec{r}_i \times m_i \vec{v}_i) = \vec{n} \cdot \sum_i \vec{L}_i = \vec{n} \cdot \vec{L} $$
    	We see that if $q_j$ is cyclic ($Q_j = 0$), the the angular momentum is conserved.
    	
    	\item In other words,
    	If a generalized coordinate corresponding to a displacement is cyclic, (the translation of the system has no effect) the corresponding linear momentum is conserved.
    	If a generalized rotation coordinate is cyclic (the system is invariant under rotation about the given axis) the conjugate angular momentum is conserved.
    	$\Rightarrow$ The conservation theorems are connected with the symmetry properties.
    \end{enumerate}
    
    \section*{2.7 Energy function and conservation of energy}
    \begin{enumerate}
    	\item Consider L as a function of $q_j$, $\dot{q}_j$ and $t$.
    	Then the total time derivative of L is:
    	$$ \frac{dL}{dt} = \sum_j \frac{\partial L}{\partial q_j} \dot{q}_j + \sum_j \frac{\partial L}{\partial \dot{q}_j} \ddot{q}_j + \frac{\partial L}{\partial t} $$
    	From Lagrange's equations:
    	$$ \frac{\partial L}{\partial q_j} = \frac{d}{dt}\left(\frac{\partial L}{\partial \dot{q}_j}\right) $$
    	We obtain
    	$$ \frac{dL}{dt} = \sum_j \frac{d}{dt}\left(\frac{\partial L}{\partial \dot{q}_j}\right) \dot{q}_j + \sum_j \frac{\partial L}{\partial \dot{q}_j} \ddot{q}_j + \frac{\partial L}{\partial t} $$
    	$$ = \sum_j \frac{d}{dt}\left(\dot{q}_j \frac{\partial L}{\partial \dot{q}_j}\right) + \frac{\partial L}{\partial t} $$
    	Therefore,
    	$$ \frac{d}{dt}\left(\sum_j \dot{q}_j \frac{\partial L}{\partial \dot{q}_j} - L \right) + \frac{\partial L}{\partial t} = 0 $$
    	We define the energy function as:
    	$$ h(q_j, \dot{q}_j, t) = \sum_j \dot{q}_j \frac{\partial L}{\partial \dot{q}_j} - L $$
    	and the total time derivative of h:
    	$$ \frac{dh}{dt} = - \frac{\partial L}{\partial t} $$
    	If L doesn't contain t explicitly, then h is conserved.
    	
    	\item Normally, $T = T_0 + T_1 + T_2$ as discussed.
    	$\Rightarrow L(q, \dot{q}, t) = L_0(q, t) + L_1(\dot{q}, q, t) + L_2(\ddot{q}, \dot{q}, q, t)$
    	According to Euler's theorem, if f is a homogeneous function of degree n in the variables $x_i$, then
    	$$ \sum_i x_i \frac{\partial f}{\partial x_i} = n f $$
    	Applied to h: $\Rightarrow h = 2L_2 + L_1 - L = L_2 - L_0$.
    	If V doesn't depend on generalized velocities, then $L_2 = T$ and $L_0 = -V \Rightarrow h = T + V = E$.
    	
    	\item Finally, we consider frictional forces derivable from a dissipation function $\mathcal{F}$.
    	When, we obtain: $\frac{dh}{dt} + \frac{\partial L}{\partial t} = \sum_j \dot{q}_j \frac{\partial \mathcal{F}}{\partial \dot{q}_j}$
    	since $\mathcal{F}$ is a homogeneous function of the $\dot{q}$'s of degree 2, we have, applying Euler's theorem
    	$$ \frac{dh}{dt} = -2\mathcal{F} $$
    	Also, if L isn't an explicit function of time and $h = E$, we have
    	$$ \frac{dE}{dt} = -2\mathcal{F} $$
    \end{enumerate}
\end{document}