\documentclass[12pt]{article}
\usepackage{amsmath}
\usepackage{amsfonts}
\usepackage{amssymb}
\usepackage{graphicx}
\usepackage{braket}
\usepackage{physics}
\usepackage{bm}
\usepackage[a4paper, margin=1in]{geometry}

\title{\textbf{The Hamilton Equations of Motion}}
\author{}
\date{}

\begin{document}
	\maketitle
	\vspace{-2cm}
	\section{8.1 Legendre Transformations and the Hamilton Equations of Motion}
	
	\subsection{From Lagrangian to Hamiltonian Formalism}
	For a system with $n$ degrees of freedom, we have $n$ equations of motion of the form
	$$
	\frac{d}{dt}\left(\frac{\partial L}{\partial \dot{q}_i}\right) - \frac{\partial L}{\partial q_i} = 0
	$$
	We require $2n$ independent first-order equations of motion in terms of $2n$ independent variables to describe the behavior of the system point in a phase space.
	
	The conjugate momenta $p_i$ is defined by
	$$
	p_i = \frac{\partial L(q_j, \dot{q}_j, t)}{\partial \dot{q}_j} \quad (\text{no sum on } j)
	$$
	The quantities $(q, p)$ are known as the canonical variables.
	
	\subsection{The Legendre Transformation}
	We consider a function of only two variables $f(x, y)$. The differential of $f$ has the form
	$$
	df = u\,dx + v\,dy
	$$
	where $u = \frac{\partial f}{\partial x}$, $v = \frac{\partial f}{\partial y}$.
	
	Let $g$ be a function of $u$ and $y$ defined by
	$$
	g = f - ux
	$$
	whose differential has the form
	$$
	dg = df - u\,dx - x\,du
	$$
	or
	$$
	dg = v\,dy - x\,du
	$$
	The quantities $x$ and $v$ are given by
	$$
	x = -\frac{\partial g}{\partial u}, \quad v = \frac{\partial g}{\partial y}
	$$
	
	\subsection{Example: Thermodynamics}
	The Legendre transformation is used in thermodynamics. For example: The differential change in energy $dU$ whose corresponding change in heat content $dQ$ and the work done $dW$:
	$$
	dU = dQ - dW
	$$
	For a gas undergoing a reversible process,
	$$
	dU = T\,dS - P\,dV
	$$
	where $U(S, V)$ is a function of entropy $S$, and the volume $V$. And $T, P$ are given by
	$$
	T = \frac{\partial U}{\partial S}, \quad P = -\frac{\partial U}{\partial V}
	$$
	The enthalpy $H(S, P)$ is generated by the Legendre transformation
	$$
	H = U + PV
	$$
	which gives: $dH = T\,dS + V\,dP$ where $T = \frac{\partial H}{\partial S}$, $V = \frac{\partial H}{\partial P}$.
	
	Additional Legendre Transformations generate the Helmholtz free energy $F(T, V)$ and the Gibbs free energy $G(T, P)$:
	\begin{align*}
		F &= U - TS \\
		G &= H - TS
	\end{align*}
	
	\subsection{ The Transformation to Hamiltonian Mechanics}
	The transformation from $(q, \dot{q}, t)$ to $(q, p, t)$.
	The total differential of the Lagrangian $L(q_i, \dot{q}_i, t)$ is
	$$
	dL = \sum_i \frac{\partial L}{\partial q_i} dq_i + \sum_i \frac{\partial L}{\partial \dot{q}_i} d\dot{q}_i + \frac{\partial L}{\partial t} dt
	$$
	The canonical momentum $p_i = \partial L / \partial \dot{q}_i$ gives:
	$$
	p_i = \frac{\partial L}{\partial \dot{q}_i}
	$$
	So, $dL$ can be written as
	$$
	dL = \sum_i \dot{p}_i dq_i + \sum_i p_i d\dot{q}_i + \frac{\partial L}{\partial t} dt
	$$
	The Hamiltonian $H(q, p, t)$ is generated by the Legendre transformation
	$$
	H(q, p, t) = \sum_i p_i \dot{q}_i - L(q, \dot{q}, t)
	$$
	whose differential is
	\begin{align*}
		dH &= \sum_i \dot{q}_i dp_i + \sum_i p_i d\dot{q}_i - dL \\
		&= \sum_i \dot{q}_i dp_i + \sum_i p_i d\dot{q}_i - \left( \sum_i \dot{p}_i dq_i + \sum_i p_i d\dot{q}_i + \frac{\partial L}{\partial t} dt \right) \\
		&= \sum_i \dot{q}_i dp_i - \sum_i \dot{p}_i dq_i - \frac{\partial L}{\partial t} dt
	\end{align*}
	Since $H = H(q, p, t)$, its differential is also given by
	$$
	dH = \sum_i \frac{\partial H}{\partial q_i} dq_i + \sum_i \frac{\partial H}{\partial p_i} dp_i + \frac{\partial H}{\partial t} dt
	$$
	We thus obtain Hamilton's canonical equations:
	$$
	\begin{cases}
		\dot{q}_i = \frac{\partial H}{\partial p_i} & \text{1} \\
		-\dot{p}_i = \frac{\partial H}{\partial q_i} & \text{2} \\
		-\frac{\partial L}{\partial t} = \frac{\partial H}{\partial t} & \text{3}
	\end{cases}
	$$
	
	\subsection{ Hamiltonian in Terms of Energy}
	$H$, in terms of the generalized velocities of degree 1, 2, can be written as:
	$$
	H = \sum_i p_i \dot{q}_i - L = \sum_i \dot{q}_i \frac{\partial L}{\partial \dot{q}_i} - L
	$$
	If $L = L_2 + L_1 + L_0$, where $L_k$ are homogeneous functions of $\dot{q}_i$ of degree $k$, then by Euler's theorem on homogeneous functions:
	$$
	\sum_i \dot{q}_i \frac{\partial L}{\partial \dot{q}_i} = \sum_i \dot{q}_i \frac{\partial (L_2+L_1+L_0)}{\partial \dot{q}_i} = 2L_2 + L_1
	$$
	So the Hamiltonian becomes
	$$
	H = (2L_2 + L_1) - (L_2 + L_1 + L_0) = L_2 - L_0
	$$
	Where $L_2 = T$ is the kinetic energy, and if the forces are derivable from a conservative potential $V$, then $L_0 = -V$.
	Then
	$$
	H = T + V = E
	$$
	
	\subsection{ Hamiltonian for Quadratic Lagrangians}
	If $L$ is a quadratic function of the generalized velocities and $L_1$ is a linear function of the same variables with the following dependencies:
	$$
	L(q, \dot{q}, t) = L_2(q, \dot{q}, t) + L_1(q, \dot{q}, t) + L_0(q, t)
	$$
	where
	$$
	L_2 = \frac{1}{2} \tilde{\dot{q}} T \dot{q}, \quad L_1 = \tilde{a} \dot{q}
	$$
	where $T_{ij}$'s and $a_i$'s are functions of $q$'s and $t$.
	We can form the $\dot{q}_i$'s into a single column matrix $\dot{q}$. Then:
	$$
	L(q, \dot{q}, t) = \frac{1}{2} \tilde{\dot{q}} T \dot{q} + \tilde{a} \dot{q} + L_0(q,t)
	$$
	where the single row matrix $\tilde{\dot{q}}$ is the transpose of $\dot{q}$ and $a$ is a column matrix, and $T$ is a square $n \times n$ matrix.
	
	We consider the special case where $q_i = \{x, y, z\}$ and $T$ is diagonal.
	Then
	$$
	\frac{1}{2} \tilde{\dot{q}} T \dot{q} = \frac{1}{2} \begin{pmatrix} \dot{x} & \dot{y} & \dot{z} \end{pmatrix} \begin{pmatrix} m & 0 & 0 \\ 0 & m & 0 \\ 0 & 0 & m \end{pmatrix} \begin{pmatrix} \dot{x} \\ \dot{y} \\ \dot{z} \end{pmatrix} = \frac{1}{2} m (\dot{x}^2 + \dot{y}^2 + \dot{z}^2)
	$$
	and
	$$
	\tilde{a} \dot{q} = \begin{pmatrix} a_x & a_y & a_z \end{pmatrix} \begin{pmatrix} \dot{x} \\ \dot{y} \\ \dot{z} \end{pmatrix} = a_x \dot{x} + a_y \dot{y} + a_z \dot{z} = \mathbf{a} \cdot \mathbf{\dot{q}}
	$$
	In this notation, the conjugate momenta is (since $T$ is symmetric):
	$$
	p = T\dot{q} + a
	$$
	or
	$$
	\dot{q} = T^{-1}(p-a) \quad (\text{if } T^{-1} \text{ exists})
	$$
	The corresponding equation for $\tilde{\dot{q}}$ is
	$$
	\tilde{\dot{q}} = (\tilde{p} - \tilde{a}) (T^{-1})^T
	$$
	Then the final form for the Hamiltonian, using $H = L_2 - L_0 = \frac{1}{2}\tilde{\dot{q}}T\dot{q} - L_0$, is
	\begin{align*}
		H(q, p, t) &= \frac{1}{2} \left[ (\tilde{p} - \tilde{a}) (T^{-1})^T \right] T \left[ T^{-1} (p - a) \right] - L_0(q, t) \\
		&= \frac{1}{2} (\tilde{p} - \tilde{a}) (T^{-1})^T T T^{-1} (p - a) - L_0(q, t)
	\end{align*}
	Since T is symmetric, $T^{-1}$ is also symmetric, so $(T^{-1})^T = T^{-1}$.
	$$
	H(q, p, t) = \frac{1}{2} (\tilde{p} - \tilde{a}) T^{-1} (p - a) - L_0(q, t)
	$$
	\section*{Advanced Mechanics Notes}
	
	\subsection*{Matrix Inverse and Tensors}
	where $T^{-1} = \frac{\tilde{T}^c}{|T|}$, with $\tilde{T}^c$ the cofactor matrix whose elements $(\tilde{T}^c)_{jk} = (-1)^{j+k} (\text{j-th row of } T) (\text{k-th column of } T)$.
	
	Let a tensor $T$ be defined as:
	\[ T = \begin{bmatrix} m & m \\ m & m \end{bmatrix}, \quad T^{-1} = \begin{bmatrix} m & m \\ m & m \end{bmatrix}, \quad T_c = \begin{bmatrix} m & m \\ m & m \end{bmatrix} \]
	The determinant is $|T| = m^2$.
	
	\subsection*{6. Central Force Field}
	We consider the spatial motion of a particle in a central force field, using $(r, \theta, \phi)$.
	The kinetic energy is:
	\[ T = \frac{1}{2} m v^2 = \frac{1}{2} m (\dot{r}^2 + r^2 \sin^2\phi \, \dot{\theta}^2 + r^2 \dot{\phi}^2) \]
	The Hamiltonian is the total energy $T+V$:
	\[ H(r, \theta, \phi, p_r, p_\theta, p_\phi) = \frac{1}{2m} (p_r^2 + \frac{p_\theta^2}{r^2} + \frac{p_\phi^2}{r^2 \sin^2\theta}) + V(r) \]
	We choose the Cartesian coordinates:
	\[ T = \frac{1}{2} m v^2 = \frac{m}{2} \dot{x}_i \dot{x}_i \]
	So,
	\[ H(x_i, p_i) = \frac{p_i p_i}{2m} + V(r) \]
	\[ = \frac{p \cdot p}{2m} + V(\sqrt{x_i x_i}) \]
	Whenever a vector is used from here on to represent canonical momenta, it will refer to the momenta conjugate to Cartesian position coordinates.
	
	\subsection*{7. Electromagnetic Field}
	We consider a single particle of mass $m$ and charge $q$ moving in an electromagnetic field.
	The Lagrangian is:
	\[ L = T - V = \frac{1}{2} m v^2 - q\phi + q\vec{A} \cdot \vec{v} \]
	\[ = \frac{m}{2} \dot{x}_i \dot{x}_i + q A_i \dot{x}_i - q\phi \]
	where $\phi, \vec{A}$ are functions of $x_i$ and time.
	
	The canonical momenta are:
	\[ p_i = m\dot{x}_i + qA_i \]
	The Hamiltonian is:
	\[ H = \frac{(p_i - qA_i)(p_i - qA_i)}{2m} + q\phi \]
	By vector $\vec{p}$, we write that:
	\[ H = \frac{1}{2m}(\vec{p} - q\vec{A})^2 + q\phi \]
	where $\vec{p}$ refers only to momenta conjugate to $x_i$.
	
	\subsection*{8. System with n Degrees of Freedom}
	For a system of $n$ degrees of freedom, we construct a column matrix $\eta$ with $2n$ elements such that $y_i = q_i$ and $y_{i+n} = p_i$ for $i \leq n$.
	The column matrix $\frac{\partial H}{\partial \eta}$ has the elements:
	\[ \left( \frac{\partial H}{\partial \eta} \right)_i = \frac{\partial H}{\partial q_i}, \quad \left( \frac{\partial H}{\partial \eta} \right)_{i+n} = \frac{\partial H}{\partial p_i}; \quad i \leq n \]
	Let $J$ be the $2n \times 2n$ square matrix:
	\[ J = \begin{bmatrix} 0 & I \\ -I & 0 \end{bmatrix} \]
	with the inverse
	\[ \tilde{J} = \begin{bmatrix} 0 & -I \\ I & 0 \end{bmatrix} \]
	which means $\tilde{J} = \tilde{J}^{-1} = J$. So $J\tilde{J}=I$ and $J^2 = -I$.
	Thus $|J| = \pm 1$.
	Hence, Hamilton's equations of motion are:
	\[ \dot{\eta} = J \frac{\partial H}{\partial \eta} \]
	For two coordinate variables:
	\[ \begin{bmatrix} \dot{q}_1 \\ \dot{q}_2 \\ \dot{p}_1 \\ \dot{p}_2 \end{bmatrix} = \begin{bmatrix} 0 & 0 & 1 & 0 \\ 0 & 0 & 0 & 1 \\ -1 & 0 & 0 & 0 \\ 0 & -1 & 0 & 0 \end{bmatrix} \begin{bmatrix} \partial H / \partial q_1 \\ \partial H / \partial q_2 \\ \partial H / \partial p_1 \\ \partial H / \partial p_2 \end{bmatrix} = \begin{bmatrix} \partial H / \partial p_1 \\ \partial H / \partial p_2 \\ -\partial H / \partial q_1 \\ -\partial H / \partial q_2 \end{bmatrix} \]
	
	\subsection*{8.2 Cyclic Coordinates and Conservation Theorems}
	We see that $\dot{p}_j = -\frac{\partial H}{\partial q_j}$.
	Hence a cyclic coordinate will be absent from the Hamiltonian and conversely, if a generalized coordinate does not occur in $H$, the conjugate momentum is conserved.
	
	If $L$ and $H$ are not an explicit function of $t$, then $H$ is a constant of motion.
	\[ \frac{dH}{dt} = \sum_i \left( \frac{\partial H}{\partial q_i} \dot{q}_i + \frac{\partial H}{\partial p_i} \dot{p}_i \right) + \frac{\partial H}{\partial t} \]
	Since $\frac{\partial H}{\partial q_i} = -\dot{p}_i$ and $\frac{\partial H}{\partial p_i} = \dot{q}_i$,
	we can write that $\frac{dH}{dt} = \frac{\partial H}{\partial t}$.
	
	We define the generalized coordinates $r_m = r_m(q_1, \dots, q_n; t)$ if they don't depend explicitly upon the time and if the potential is velocity independent.
	\[ H = T + V \]
	
	We consider a one-dimensional system . Suppose a point mass $m$ is attached to a spring of force constant $k$, the other end of which is fixed on a massless cart that is being moved uniformly by an external device with velocity $v_0$.
	
	\begin{figure}[h]
		\centering
		\includegraphics[width=0.7\linewidth]{figure1}
		\caption{}
		\label{fig:figure1}
	\end{figure}
	
	The Lagrangian is:
	\[ L(x, \dot{x}, t) = T - V = \frac{m\dot{x}^2}{2} - \frac{k}{2}(x - v_0 t)^2 \]
	The corresponding equation of motion:
	\[ m\ddot{x} = -k(x - v_0 t) \]
	
	We change the unknown to $x'(t)$ defined as:
	\[ x' = x - v_0 t \]
	Noting that $\ddot{x}' = \ddot{x}$, thus:
	\[ m\ddot{x}' = -kx' \]
	We see that $x'$ is the displacement of the particle relative to the cart. The equation exhibits simple harmonic motion.
	Since the potential energy doesn't involve generalized velocity, the Hamiltonian relative to $x$ is the sum of kinetic and potential energies.
	\[ H(x, p, t) = T + V = \frac{p^2}{2m} + \frac{k}{2}(x - v_0 t)^2 \]
	which is explicitly a function of $t$, it's not conserved.
	
	In terms of $x'$:
	\[ L(x', \dot{x}') = \frac{m\dot{x}'^2}{2} + m\dot{x}'v_0 + \frac{mv_0^2}{2} - \frac{kx'^2}{2} \]
	\[ H'(x', p') = \frac{(p' - mv_0)^2}{2m} + \frac{kx'^2}{2} - \frac{mv_0^2}{2} \]
	which is the new conserved Hamiltonian and leads to the same motion for the particle.
	
	\subsection*{8.3 Routh's Procedure}
	\begin{enumerate}
		\item We consider some coordinate, say $q_n$, is cyclic. The Lagrangian is a function of $q, \dot{q}$:
		\[ L = L(q_1, \dots, q_{n-1}; \dot{q}_1, \dots, \dot{q}_n; t) \]
		In this condition, $p_n$ is some constant. $H$ has the form:
		\[ H = H(q_1, \dots, q_{n-1}; p_1, \dots, p_{n-1}; \alpha; t) \]
		We obtain that $\dot{q}_n = \frac{\partial H}{\partial \alpha}$.
		
		\item We consider the cyclic coordinates are $q_{s+1}, \dots, q_n$, and a function $R$ (Routhian), defined as:
		\[ R(q_1, \dots, q_s; \dot{q}_1, \dots, \dot{q}_s; p_{s+1}, \dots, p_n; t) = \sum_{k=s+1}^n p_k \dot{q}_k - L \]
		or
		\[ R = H_{cycl} (p_k) - L_{noncycl} (q_i, \dot{q}_i, \ddot{q}_i) \]
		The Lagrange equations for the $s$ coordinates:
		\[ \frac{d}{dt}\left(\frac{\partial R}{\partial \dot{q}_i}\right) - \frac{\partial R}{\partial q_i} = 0, \quad i=1, \dots, s \]
		while for the $n-s$ coordinates:
		\[ \frac{\partial R}{\partial q_i} = -\dot{p}_i = 0, \quad \frac{\partial R}{\partial p_i} = \dot{q}_i, \quad i=s+1, \dots, n \]
		
		\item We consider the Kepler problem, where the inverse-square central force $f(r)$ derived from the potential $V(r) = -k/r$.
		Then:
		\[ L = \frac{1}{2}m(\dot{r}^2 + r^2\dot{\theta}^2) + \frac{k}{r} \]
		The ignorable coordinate is $\theta$. The Routhian is:
		\[ R(r, \dot{r}, p_\theta) = \frac{p_\theta^2}{2mr^2} - \frac{1}{2}m\dot{r}^2 - \frac{k}{r} \]
		The equation of motion corresponding is:
		\[ \ddot{r} - \frac{p_\theta^2}{mr^3} + \frac{k}{mr^2} = 0 \]
		and
		\[ p_\theta = 0, \quad \frac{p_\theta}{mr^2} = \dot{\theta} \Rightarrow p_\theta = mr^2\dot{\theta} = \text{constant} \]
	\end{enumerate}
	\section*{8.4 The Hamiltonian Formulation of Relativistic Mechanics}
	
	\subsection*{1. Single-particle Lagrangian and Hamiltonian}
	For a single-particle Lagrangian:
	\begin{equation}
		L = -mc^2\sqrt{1-\beta^2} - V
	\end{equation}
	and the Hamiltonian:
	\begin{equation}
		H = T + V
	\end{equation}
	The energy $T$ can be expressed in terms of $p_i$:
	\begin{equation}
		T^2 = p^2c^2 + m^2c^4
	\end{equation}
	So,
	\begin{equation}
		H = \sqrt{p^2c^2 + m^2c^4} + V
	\end{equation}
	When the system consists of an electromagnetic field:
	\begin{equation}
		L = -mc^2\sqrt{1-\beta^2} + q\vec{A}\cdot\vec{v} - q\phi
	\end{equation}
	Thus:
	\begin{equation}
		H = T + q\phi
	\end{equation}
	The canonical momenta conjugate to the Cartesian $P^i = mu^i + qA^i$. Thus the final form:
	\begin{equation}
		H = \sqrt{(\vec{P}-q\vec{A})^2c^2 + m^2c^4} + q\phi
	\end{equation}
	We note that $(H-q\phi)/c$ is the zeroth component of the 4-vector $mu^\nu + qA^\nu$.
	
	\subsection*{2. Covariant Formulation}
	The Lagrangian is in the $(q_1, \dots, q_n, t)$ configuration space.
	\begin{equation}
		\Lambda(q, \dot{q}', t, t') = t' L(q, \frac{\dot{q}'}{t'}, t)
	\end{equation}
	The momentum conjugate to $t$ is
	\begin{equation}
		P_t = \frac{\partial\Lambda}{\partial t'} = L + t'\frac{\partial L}{\partial t'}
	\end{equation}
	Since $\dot{q} = \dot{q}'/t'$, the relation becomes
	\begin{equation}
		P_t = L - \frac{\dot{q}'_i}{t'^2}\frac{\partial L}{\partial \dot{q}_i} = L - \dot{q}_i\frac{\partial L}{\partial \dot{q}_i} = -H
	\end{equation}
	
	\subsubsection*{1) For a single free particle}
	The covariant Lagrangian
	\begin{equation}
		\Lambda(x^\mu, u^\mu) = \frac{1}{2} m u_\mu u^\mu
	\end{equation}
	also leads to the equations of motion.
	
	The Hamiltonian is
	\begin{equation}
		H_c = \frac{p_\mu p^\mu}{2m}
	\end{equation}
	
	\subsubsection*{2) In an electromagnetic field}
	A covariant Lagrangian is:
	\begin{equation}
		\Lambda(x^\mu, u^\mu) = \frac{1}{2} m u_\mu u^\mu + q u_\mu A^\mu(x_\nu)
	\end{equation}
	with the canonical momenta
	\begin{equation}
		p_\mu = m u_\mu + q A_\mu
	\end{equation}
	and the corresponding Hamiltonian
	\begin{equation}
		H_c' = \frac{(p_\mu - qA_\mu)(p^\mu - qA^\mu)}{2m}
	\end{equation}
	Both $H_c'$ and $H_c$ are constant and equal to $-\frac{mc^2}{2}$.
	
	\subsection*{The equations of motion for one particle}
	which is due to the covariant Hamiltonian:
	\begin{equation}
		\frac{dx^\mu}{d\tau} = \frac{\partial H_c'}{\partial p_\mu} g^{\mu\nu}
	\end{equation}
	and
	\begin{equation}
		\frac{dp^\mu}{d\tau} = -\frac{\partial H_c'}{\partial x_\mu} g^{\mu\nu}
	\end{equation}
	One of the $\nu=0$ equations is the constitutive equation for $p^0$:
	\begin{equation}
		u^0 = \frac{\partial H_c'}{\partial p_0} = \frac{1}{m}(P^0 - qA^0)
	\end{equation}
	or
	\begin{equation}
		p^0 = \frac{1}{c}(T+q\phi) = \frac{H_c'}{c}
	\end{equation}
	The other can be written as
	\begin{equation}
		\frac{1}{\sqrt{1-\beta^2}}\frac{dp^0}{dt} = -\frac{1}{c}\frac{\partial H_c}{\partial t}
	\end{equation}
	or
	\begin{equation}
		\frac{dH}{dt} = \sqrt{1-\beta^2} \frac{\partial H_c}{\partial t}
	\end{equation}
	\section*{8.5 Derivation of Hamilton's Equations from a Variational Principle}
	
	We see that, from Hamilton's principle,
	\[ \delta I = \delta \int_{t_1}^{t_2} L \, dt = 0 \]
	\[ = \delta \int_{t_1}^{t_2} (p_i \dot{q}_i - H(q, p, t)) \, dt = 0 \]
	which is the modified Hamilton's principle.
	We rewrite that, for a space of $2n$ dimensions
	\[ \delta I = \delta \int_{t_1}^{t_2} f(q, \dot{q}, p, \dot{p}, t) \, dt = 0 \]
	The Euler-Lagrange equations are
	\[ \frac{d}{dt}\left(\frac{\partial f}{\partial \dot{q}_j}\right) - \frac{\partial f}{\partial q_j} = 0, \quad j=1, \dots, n \]
	\[ \frac{d}{dt}\left(\frac{\partial f}{\partial \dot{p}_j}\right) - \frac{\partial f}{\partial p_j} = 0, \quad j=1, \dots, n \]
	Since $f$ contains $\dot{q}_j$ only through $p_i \dot{q}_i$ and $q_j$ only in $H$
	\[ \dot{p}_j + \frac{\partial H}{\partial q_j} = 0 \]
	Similarly,
	\[ \dot{q}_j - \frac{\partial H}{\partial p_j} = 0 \]
	We can subtract or add the total time derivative of an arbitrary function $F(q, p, t)$ to the integrand without affecting the validity.
	For example: $\frac{d}{dt}(F_1)$
	\[ \Rightarrow \delta \int_{t_1}^{t_2} (-p_i \dot{q}_i - H(q, p, t)) \, dt = 0 \]
	
	\section*{8.6 The Principle of Least Action}
	\begin{enumerate}
		\item We define the possible varied paths as
		\[ q_i(t, \alpha) = q_i(t, 0) + \alpha \eta_i(t) \]
		where $\alpha$ is an infinitesimal parameter that goes to zero for the correct path.
		
		\begin{figure}[h]
			\centering
			\includegraphics[width=0.7\linewidth]{figure2}
			\caption{}
			\label{fig:figure2}
		\end{figure}
		
		We evaluate the $\Delta$ variation of the action integral
		\[ \Delta \int_{t_1}^{t_2} L \, dt = \int_{t_1 + \delta t_1}^{t_2 + \delta t_2} L(\alpha) \, dt - \int_{t_1}^{t_2} L(0) \, dt \]
		\[ = [L(t_2) \Delta t_2 - L(t_1) \Delta t_1 + \int_{t_1}^{t_2} \delta L \, dt] \]
		\[ = \int_{t_1}^{t_2} \left[ \frac{\partial L}{\partial q_i} - \frac{d}{dt}\left(\frac{\partial L}{\partial \dot{q}_i}\right) \right] \delta q_i \, dt + \delta L \delta q_i \bigg|_{t_1}^{t_2} \]
		By Lagrange's equations:
		\[ \Delta \int_{t_1}^{t_2} L \, dt = [L \Delta t + p_i \delta q_i] \bigg|_{t_1}^{t_2} \]
		We see that $\Delta q_i = \delta q_i + \dot{q}_i \Delta t$.
		Hence
		\[ \Delta \int_{t_1}^{t_2} L \, dt = [L \Delta t - p_i \dot{q}_i \Delta t + p_i \Delta q_i] \bigg|_{t_1}^{t_2} \]
		\[ = (p_i \Delta q_i - H \Delta t) \bigg|_{t_1}^{t_2} \]
		We restrict that
		\begin{enumerate}
			\item Only consider systems where $L, H$ are not explicit functions of time $\rightarrow H$ is conserved.
			\item The variation is such that $H$ is conserved on both the varied and actual path.
			\item The varied paths are limited by requiring $\Delta q_i$ vanish at the end points.
		\end{enumerate}
		Then
		\[ \Delta \int_{t_1}^{t_2} L \, dt = -H(\Delta t_2 - \Delta t_1) \]
		and
		\[ \int_{t_1}^{t_2} L \, dt = \int_{t_1}^{t_2} p_i \dot{q}_i \, dt - H(t_2 - t_1) \]
		\[ \Rightarrow \Delta \int_{t_1}^{t_2} L \, dt = \Delta \int p_i \dot{q}_i \, dt - H(\Delta t_2 - \Delta t_1) \]
		We obtain the principle of least action
		\[ \Delta \int_{t_1}^{t_2} p_i \dot{q}_i \, dt = 0 \]
		If the trajectory is described by $\theta$, the modified Hamilton's principle is
		\[ \delta \int_0^\theta (p_i \dot{q}_i' - H)' \, d\theta = 0 \]
		Since $p_i$ don't change under the shift from $t$ to $\theta$ and $q_i' = \dot{q}_i'$ and the momentum conjugate to $t$ is $-H$. It can be written as
		\[ \delta \int_0^\theta \sum_{i=1}^{n+1} p_i q_i' \, d\theta = 0 \]
		
		\item The kinetic energy is a quadratic function of $\dot{q}_i$'s if the generalized coordinates don't involve $t$.
		\[ T = \frac{M_{jk}(\underline{q})}{2} \dot{q}_j \dot{q}_k \quad \text{and} \quad p_i \dot{q}_i = 2T \]
		The principle of least action is
		\[ \Delta \int_{t_1}^{t_2} T \, dt = 0 \]
		Further, if there are no external forces, we can also write that since $T$ and $H$ are conserved
		\[ \Delta (t_2 - t_1) = 0 \]
		
		\item We consider a curvilinear space whose metric is $g_{\mu\nu}$ and the interval traversed for displacements given by $dx^\mu \dots ds^2 = g_{\mu\nu} dx^\mu dx^\nu$ is the interval.
		For a configuration space whose $M_{jk}$ coefficients form the metric tensor. The element of path length in the space is
		\[ |dP|^2 = M_{jk} \, dq_j \, dq_k \]
		Thus,
		\[ T = \frac{1}{2} \left(\frac{dP}{dt}\right)^2 \quad \text{or} \quad dt = \frac{dP}{\sqrt{2T}} \]
		The principle of least action
		\[ \Delta \int_{t_1}^{t_2} T \, dt = 0 = \Delta \int_{P_1}^{P_2} \sqrt{T/2} \, dP \]
		\[ \Rightarrow \Delta \int_{P_1}^{P_2} \sqrt{H-V(q)} \, dP = 0 \]
		which is the Jacobi's form.
	\end{enumerate}
\end{document}