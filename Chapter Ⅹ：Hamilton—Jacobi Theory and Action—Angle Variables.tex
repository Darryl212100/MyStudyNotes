\documentclass{article}
\usepackage[a4paper, margin=1in]{geometry}
\usepackage{amsmath}
\usepackage{amsfonts}
\usepackage{amssymb}
\usepackage{graphicx}
\usepackage{braket}
\usepackage{physics}
\usepackage{bm}
\begin{document}
	
	\section*{Hamilton-Jacobi Theory and Action-Angle Variables}
	
	\subsection{10.1 The Hamilton-Jacobi Equation for Hamilton's Principal Function}
	We ensure the new variables to be constant in time by requiring the transformed Hamiltonian $K$ to be zero. The equations of motion are
	\begin{align*}
		\frac{\partial K}{\partial P_i} &= \dot{Q_i} = 0 \\
		-\frac{\partial K}{\partial Q_i} &= \dot{P_i} = 0
	\end{align*}
	And $K, P_i$ are generated by
	\[ K = H + \frac{\partial F}{\partial t} \]
	Only to zero if: $H(q, p, t) + \frac{\partial F}{\partial t} = 0$.
	
	For convenience, we take $F$ as a function of $q_i$ (the new momenta) and time: $F_2(q, P, t)$. By the transformation $p_i = \frac{\partial F_2}{\partial q_i}$, we obtain:
	\[ H\left(q_1, \dots, q_n; \frac{\partial F_2}{\partial q_1}, \dots, \frac{\partial F_2}{\partial q_n}; t\right) + \frac{\partial F_2}{\partial t} = 0 \]
	This is the Hamilton-Jacobi equation.
	
	We denote $F_2$ by $S$ which is the Hamilton's principal function
	\[ F_2 = S = S(q_1, \dots, q_n; \alpha_1, \dots, \alpha_{n+1}; t) \]
	where $\alpha_i$ are $n+1$ independent constants.
	
	We note that if $S$ is a solution to the H-J equation, so is $S+\alpha$ (where $\alpha$ is a constant). Hence the complete solution to the equation can be written as
	\[ S = S(q_1, \dots, q_n; \alpha_1, \dots, \alpha_n, t) \]
	We can take the $n$ constants to be the new momenta $P_i = \alpha_i$.
	And the $n$ transformation equations
	\begin{align*}
		p_i &= \frac{\partial S(q, P, t)}{\partial q_i} \\
		Q_i = \beta_i &= \frac{\partial S(q, \alpha, t)}{\partial \alpha_i}
	\end{align*}
	We may also represent $q_j$ in terms of $\alpha, \beta, t$
	\[ q_j = q_j(\alpha, \beta, t) \]
	so as for $P_i$: $P_i = P_i(\alpha, \beta, t)$.
	This gives solutions for Hamilton's equations of motion.
	
	The total time derivative of $S$ is
	\[ \frac{dS}{dt} = \frac{\partial S}{\partial q_i}\dot{q_i} + \frac{\partial S}{\partial t} \quad \text{since } P_i's \text{ are constants} \]
	\[ \frac{dS}{dt} = p_i \dot{q_i} - H = L \]
	So that
	\[ S = \int L dt + \text{constant} \]
	which is another expression of the Hamilton's principle.
	
	\subsection*{When Hamiltonian doesn't depend explicitly upon time}
	\[ S(q, \alpha, t) = W(q, \alpha) - \alpha t \]
	where $W$ is Hamilton's characteristic function such that:
	\[ \frac{dW}{dt} = \frac{\partial W}{\partial q_i} \dot{q_i} \]
	We have : $p_i = \frac{\partial W}{\partial q_i}$.
	Hence,
	\[ \frac{dW}{dt} = p_i \dot{q_i} \]
	This implies
	\[ W = \int p_i \dot{q_i} dt = \int p_i dq_i \]
	
	\subsection{10.2 The Harmonic Oscillator Problem as an Example}
	\begin{enumerate}
		\item For a one-dimensional harmonic oscillator, the Hamiltonian is
		\[ H = \frac{1}{2m}(p^2 + m^2 \omega^2 q^2) = E \]
		where $k = \sqrt{\frac{k}{m}}$ is the force constant.
		
		With $p = \frac{\partial S}{\partial q}$, we have the requirement the new Hamiltonian must vanish
		\[ \frac{1}{2m}\left[ \left(\frac{\partial S}{\partial q}\right)^2 + m^2 \omega^2 q^2 \right] + \frac{\partial S}{\partial t} = 0 \]
		with the solution of the form
		\[ S(q, \alpha, t) = W(q, \alpha) - \alpha t \]
		where $\alpha$ is constant. We obtain:
		\[ \frac{1}{2m}\left[ \left(\frac{\partial W}{\partial q}\right)^2 + m^2 \omega^2 q^2 \right] = \alpha \]
		By the relation $\frac{\partial S}{\partial t} + H = 0$ which reduces to $H = \alpha$.
		We thus have
		\[ W = \sqrt{2m\alpha} \int dq \sqrt{1 - \frac{m\omega^2 q^2}{2\alpha}} \]
		This implies
		\[ S = \sqrt{2m\alpha} \int dq \sqrt{1 - \frac{m\omega^2 q^2}{2\alpha}} - \alpha t \]
		
		With the transformation equation
		\[ \beta' = \frac{\partial S}{\partial \alpha} = \sqrt{\frac{m}{2\alpha}} \int \frac{dq}{\sqrt{1 - \frac{m\omega^2 q^2}{2\alpha}}} - t \]
		we can integrate to give
		\[ t + \beta' = \frac{1}{\omega} \arcsin(q\sqrt{\frac{m\omega^2}{2\alpha}}) \]
		which yields:
		\[ q = \sqrt{\frac{2\alpha}{m\omega^2}} \sin(\omega t + \beta) \]
		and
		\begin{align*}
			p &= \frac{\partial S}{\partial q} = \frac{\partial W}{\partial q} = \sqrt{2m\alpha - m^2\omega^2 q^2} \\
			&= \sqrt{2m\alpha(1 - \sin^2(\omega t + \beta))} \\
			&= \sqrt{2m\alpha} \cos(\omega t + \beta)
		\end{align*}
		With the initial condition $q_0, p_0$ at $t=0$
		\[ 2m\alpha = p_0^2 + m^2 \omega^2 q_0^2 \]
		\[ \tan \beta = \frac{m\omega q_0}{p_0} \]
		
		\item By substitution of $q$ into Hamilton's principal function
		\[ S = 2\alpha \int \cos^2(\omega t + \beta) dt - \alpha t = 2\alpha \int \left(\frac{\cos(2(\omega t + \beta)) + 1}{2}\right) dt \]
		Now the Lagrangian is
		\begin{align*}
			L &= \frac{1}{2m}(p^2 - m^2 \omega^2 q^2) \\
			&= \alpha(\cos^2(\omega t + \beta) - \sin^2(\omega t + \beta)) \\
			&= \alpha\left(\cos^2(\omega t + \beta) - (1 - \cos^2(\omega t + \beta))\right) \\
			&= \alpha(2\cos^2(\omega t + \beta) - 1)
		\end{align*}
		
		\item We then consider the two-dimensional anisotropic harmonic oscillator
		\[ E = \frac{1}{2m}(p_x^2 + p_y^2 + m^2 \omega_x^2 x^2 + m^2 \omega_y^2 y^2) \]
		where $\omega_i = \sqrt{\frac{k_i}{m}}$, $i=x,y$.
		
		We separate the principal function into two characteristic functions
		\[ S(x, y, \alpha, t) = W_x(x, \alpha) + W_y(y, \alpha) - \alpha t \]
		The H-J equation becomes
		\[ \frac{1}{2m}\left[ \left(\frac{\partial W_x}{\partial x}\right)^2 + m^2 \omega_x^2 x^2 + \left(\frac{\partial W_y}{\partial y}\right)^2 + m^2 \omega_y^2 y^2 \right] = \alpha \]
		The $y$-part:
		\[ \frac{1}{2m}\left(\frac{\partial W_y}{\partial y}\right)^2 + \frac{1}{2}m\omega_y^2 y^2 = \alpha_y \]
		Which yields
		\[ \frac{1}{2m}\left(\frac{\partial W_x}{\partial x}\right)^2 + \frac{1}{2}m\omega_x^2 x^2 = \alpha_x = \alpha - \alpha_y \]
		This implies
		\begin{align*}
			x &= \sqrt{\frac{2\alpha_x}{m\omega_x^2}} \sin(\omega_x t + \beta_x) \\
			y &= \sqrt{\frac{2\alpha_y}{m\omega_y^2}} \sin(\omega_y t + \beta_y) \\
			p_x &= \sqrt{2m\alpha_x} \cos(\omega_x t + \beta_x) \\
			p_y &= \sqrt{2m\alpha_y} \cos(\omega_y t + \beta_y)
		\end{align*}
		The total energy $E = \alpha_x + \alpha_y = \alpha$.
	\end{enumerate}
	\subsection{The Linear Case: $\beta=0$}
	\[
	\Rightarrow y = \frac{\sqrt{4\alpha}}{m\omega^2}\sin{\omega t}, \quad p_y = \sqrt{4m\alpha}\cos{\omega t}
	\]
	\[
	\theta = \frac{\pi}{4}, \quad p_\theta = 0
	\]
	The other case: $\beta = \frac{\pi}{2}$
	\[
	\Rightarrow r = r_0 = \sqrt{\frac{2\alpha}{m\omega^2}}, \quad p_r = 0
	\]
	\[
	\theta = \omega t, \quad p_\theta = m r_0^2 \omega
	\]
	
	\begin{figure}[h]
		\centering
		\includegraphics[width=0.7\linewidth]{figure1}
		\caption{}
		\label{fig:figure1}
	\end{figure}
	
	\subsection{10.3 The Hamilton-Jacobi Equation for Hamilton's Characteristic Function}
	\begin{enumerate}
		\item If $H$ doesn't involve time explicitly then the restricted H-J equation
		\[
		H(q_i, \frac{\partial W}{\partial q_i}) = \alpha_1
		\]
		If the generating function for a canonical transformation is $W(q, P)$, then the equations of transformation are
		\[
		p_i = \frac{\partial W}{\partial q_i}, \quad Q_i = \frac{\partial W}{\partial P_i} = \frac{\partial W}{\partial \alpha_i}
		\]
		Thus
		\[
		H(q_i, p_i) = \alpha_1
		\]
		which is the new canonical momentum.
		It also becomes:
		\[
		H(q_i, \frac{\partial W}{\partial q_i}) = \alpha_1
		\]
		Since $W$ doesn't involve time, it follows that $K = \alpha_1$.
		
		\item We again consider the 2-dimensional harmonic oscillator which is isotropic so that $k_x = k_y = k$ and $\omega_x = \omega_y = \omega$.
		And use polar coordinate
		\[
		x = r\cos\theta, \quad y = r\sin\theta
		\]
		\[
		p_x = m\dot{x}, \quad p_y = m\dot{y}, \quad p_\theta = mr^2\dot{\theta}
		\]
		\[
		H = \frac{1}{2m}(p_r^2 + \frac{p_\theta^2}{r^2} + m^2\omega^2 r^2)
		\]
		which is cyclic in $\theta$.
		
		The principal function
		\[
		S(r, \theta, \alpha, \alpha_\theta) = W_r(\alpha, r) + W_\theta(\theta, \alpha_\theta) - \alpha t
		\]
		where
		\[
		W_r(r, \alpha) + \theta \alpha_\theta - \alpha t
		\]
		a cyclic coordinate $q_i$ always has the characteristic function component $W_{q_i} = q_i \alpha_i$.
		
		The canonical momentum $p_\theta$ with $\theta$ is given by
		\[
		p_\theta = \frac{\partial S}{\partial \theta} = \alpha_\theta
		\]
		By substitution we find
		\[
		\frac{1}{2m} \left( (\frac{\partial W_r}{\partial r})^2 + \frac{\alpha_\theta^2}{r^2} \right) + \frac{1}{2}m\omega^2 r^2 = \alpha
		\]
		We write the Cartesian coordinate solution
		\[
		x = \sqrt{\frac{2\alpha}{m\omega^2}}\sin(\omega t + \epsilon), \quad p_x = \sqrt{2m\alpha}\cos(\omega t + \epsilon)
		\]
		\[
		y = \sqrt{\frac{2\alpha}{m\omega^2}}\sin\omega t, \quad p_y = \sqrt{2m\alpha}\cos\omega t
		\]
		And polar counterparts
		\[
		r = \sqrt{\frac{2\alpha}{m\omega^2}}\sqrt{\sin^2\omega t + \sin^2(\omega t+\epsilon)}, \quad p_r = m\dot{r}
		\]
		\[
		\theta = \tan^{-1}\left[\frac{\sin\omega t}{\sin(\omega t+\epsilon)}\right], \quad p_\theta = mr^2\dot{\theta}
		\]
		
		\item The momenta conjugate to the cyclic coordinates are all constant:
		\[
		p_i = \frac{\partial k}{\partial \dot{q}_i} = 0, \quad P_i = \alpha_i
		\]
		\[
		\Rightarrow \dot{Q}_i = \frac{\partial k}{\partial \alpha_i} = 1, \quad i \neq 1
		\]
		\[
		= 0, \quad i \neq 1
		\]
		with the solutions
		\[
		Q_1 = t + \beta_1 = \frac{\partial W}{\partial \alpha_1}
		\]
		\[
		Q_i = \beta_i = \frac{\partial W}{\partial \alpha_i}, \quad i \neq 1
		\]
		We denote the transformed momenta as $\gamma_i$
		\[
		\dot{Q}_i = \frac{\partial k}{\partial \gamma_i} = v_i
		\]
		where $v_i$ are function of $\gamma_i$
		\[
		\Rightarrow Q_i = v_i t + \beta_i
		\]
		
		\item In the conditions following, we can solve by either Hamilton's principal or characteristic function.
		When the Hamiltonian is any function of $t$,
		\[
		H(q, p, t)
		\]
		is conserved
		\[
		H(q, p) = \text{constant}
		\]
		We seek canonical transformation such that
		\begin{enumerate}
			\item all $Q_i, P_i$ are constants of motion
			\item all $P_i$ are constants
		\end{enumerate}
		We demand that the new Hamiltonian
		\begin{enumerate}
			\item shall vanish
			\[
			K=0
			\]
			\item shall be cyclic in all the coordinates
			\[
			K=H(P_i) = \alpha_1
			\]
		\end{enumerate}
		
		\item Then the new equations of motion are
		\begin{enumerate}
			\item 
			\[
			\dot{Q}_i = \frac{\partial k}{\partial P_i} = 0, \quad \dot{P}_i = -\frac{\partial k}{\partial Q_i} = 0
			\]
			\item
			\[
			\dot{Q}_i = \frac{\partial k}{\partial P_i} = v_i, \quad \dot{P}_i = -\frac{\partial k}{\partial Q_i} = 0
			\]
		\end{enumerate}
		with the solutions
		\begin{enumerate}
			\item 
			\[
			Q_i = \beta_i, \quad P_i = \gamma_i^2
			\]
			\item
			\[
			Q_i = v_i t + \beta_i, \quad P_i = \gamma_i
			\]
		\end{enumerate}
		The generating functions are
		\begin{enumerate}
			\item Principal Function $S(q, P, t)$
			\item Characteristic Function $W(q, P)$
		\end{enumerate}
		Satisfying the H-J equations:
		\begin{enumerate}
			\item 
			\[
			H(q, \frac{\partial S}{\partial q}, t) + \frac{\partial S}{\partial t} = 0
			\]
			\item
			\[
			H(q, \frac{\partial W}{\partial q}) - \alpha_1 = 0
			\]
		\end{enumerate}
		A complete solution contains
		\begin{enumerate}
			\item $n$ nontrivial constants of integration $\alpha, \dots, \alpha_n$
			\item $n-1$ nontrivial constants of integration with $\alpha_1, \dots, \alpha_n$
		\end{enumerate}
		The new constant momenta $P_i = \gamma_i$ can be chosen as any independent functions of $n$ constants of integration.
		\begin{enumerate}
			\item 
			\[
			P_i = \gamma_i(\alpha_1, \dots, \alpha_n)
			\]
			\item
			\[
			P_i = \gamma_i(\alpha_1, \dots, \alpha_n)
			\]
		\end{enumerate}
		So the complete solutions to H-J equation can be considered as functions of new momenta.
		\begin{enumerate}
			\item 
			\[
			S = S(q_i, \gamma_i, t)
			\]
			\item
			\[
			W = (q_i, \gamma_i)
			\]
		\end{enumerate}
		
		\item In particular, we choose one-half of the transformations equations as $\gamma_i$, one-half of the
		\begin{enumerate}
			\item 
			\[
			P_i = \frac{\partial S}{\partial q_i}
			\]
			\item
			\[
			P_i = \frac{\partial W}{\partial q_i}
			\]
		\end{enumerate}
		The other half
		\begin{enumerate}
			\item 
			\[
			Q_i = \frac{\partial S}{\partial \gamma_i} = \beta_i
			\]
			\item
			\[
			Q_i = \frac{\partial W}{\partial \gamma_i} = v_i(\gamma)t + \beta_i
			\]
		\end{enumerate}
		The generating functions are related by
		\[
		S(q, P, t) = W(q, P) - \alpha_1 t
		\]
		
	\end{enumerate}
	\section*{10.4 Separation of Variables in the H-J Equation}
	
	\subsection*{}
	\textbf{1.} We take $q_i$ as the separable coordinates.
	Thus we can use Hamilton's principal function of the form
	\begin{equation*}
		S = \sum_i S_i(q_i; \alpha_1, \dots, \alpha_n; t)
	\end{equation*}
	to split Hamilton-Jacobi equation into $n$ equations
	\begin{equation*}
		H_i\left(q_i, \frac{\partial S_i}{\partial q_i}; \alpha_1, \dots, \alpha_n; t\right) + \frac{\partial S_i}{\partial t} = 0
	\end{equation*}
	If the Hamiltonian doesn't depend upon time for each $S_i$ we have
	\begin{equation*}
		S_i(q_i; \alpha_1, \dots, \alpha_n; t) = W_i(q_i; \alpha_1, \dots, \alpha_n) - \alpha_i t
	\end{equation*}
	which provide $n$ restricted H-J equations
	\begin{equation*}
		H_i\left(q_i, \frac{\partial W_i}{\partial q_i}; \alpha_1, \dots, \alpha_n\right) = \alpha_i
	\end{equation*}
	where $\alpha_i$ are separation constants.
	
	\section*{10.5 Ignorable Coordinates and the Kepler Problem}
	
	\subsection*{}
	\textbf{1.} We suppose the cyclic coordinate is $q_1$ and the conjugate momentum $p_1$ is a constant, say $\gamma$.
	The H-J equation for $W$ is
	\begin{equation*}
		H\left(q_2, \dots, q_n; \frac{\partial W}{\partial q_2}, \dots, \frac{\partial W}{\partial q_n}\right) = \alpha_1
	\end{equation*}
	We try a separated solution of the form
	\begin{equation*}
		W = W_1(q_1, \alpha) + W'(q_2, \dots, q_n, \alpha)
	\end{equation*}
	where $W_1$ is the solution to
	\begin{equation*}
		p_1 = \gamma = \frac{\partial W_1}{\partial q_1}
	\end{equation*}
	and $W'$ is the only function in the H-J equation.
	The solution is $W_1 = \gamma q_1$.
	Thus, $W = W' + \gamma q_1$.
	
	\subsection*{}
	\textbf{2.} We let $s$ of the coordinates except be non-cyclic. The Hamiltonian is of the form:
	\begin{equation*}
		H(q_1, \dots, q_s, \frac{\partial S}{\partial q_{s+1}}, \dots, \frac{\partial S}{\partial q_n})
	\end{equation*}
	The characteristic function
	\begin{equation*}
		W(q_1, \dots, q_s; \alpha_1, \dots, \alpha_n) = \sum_{i=1}^{s} W_i(q_i; \alpha_1, \dots, \alpha_n) + \sum_{j=s+1}^{n} q_j \alpha_j
	\end{equation*}
	There are $s$ H-J equations
	\begin{equation*}
		H(q_i; \frac{\partial W_i}{\partial q_i}; \alpha_1, \dots, \alpha_n) = \alpha_i
	\end{equation*}
	
	\subsection*{}
	\textbf{3.} In general, $q_j$ is separable if the conjugate momentum $p_j$ can be segregated in the Hamiltonian in to some function $f(q_j, p_j)$.
	We seek a trial solution
	\begin{equation*}
		W = W_j(q_j, \alpha) + W'(q_i, \alpha)
	\end{equation*}
	where $q_i$ represents all $q$'s except $q_j$.
	then the H-J equation is
	\begin{equation*}
		H\left(q_i, \frac{\partial W'}{\partial q_i}, f\left(q_j, \frac{\partial W_j}{\partial q_j}\right)\right) = \alpha_1
	\end{equation*}
	We can solve for $f$:
	\begin{equation*}
		f\left(q_j, \frac{\partial W_j}{\partial q_j}\right) = g\left(q_i, \frac{\partial W'}{\partial q_i}, \alpha_1\right)
	\end{equation*}
	This can hold only if
	\begin{equation*}
		f\left(q_j, \frac{\partial W_j}{\partial q_j}\right) = \alpha_j = g\left(q_i, \frac{\partial W'}{\partial q_i}\right)
	\end{equation*}
	where $\alpha_j$ is independent of all $q$'s.
	
	\subsection*{}
	\textbf{4.} The Staeckel conditions for the separation of the H-J equations.
	\begin{enumerate}
		\item The Hamiltonian is conserved.
		\item The Lagrangian is no more than a quadratic function of the generalized velocity.
		\begin{equation*}
			H = \frac{1}{2} (\vec{p} - \vec{a}) T^{-1} (\vec{p} - \vec{a}) + V(q)
		\end{equation*}
		\item $\vec{a}$ is elements $a_i$ are functions of the corresponding coordinate only. $a_i = a_i(q_i)$.
		\item $V(q) = \sum_i \frac{V_i(q_i)}{\Phi_{i1}}$
		\item $\Phi_{ij}^{-1} = \frac{1}{T_{ii}}$ (no summation at $i$) where
		\begin{equation*}
			\left(\frac{\partial W_i}{\partial q_i} - a_i\right)^2 = 2 \delta_{ik} \Phi_{ij} \gamma_j
		\end{equation*}
		with $\gamma$ a constant unspecified vector.
	\end{enumerate}
	
	For a particle in an external force field.
	the matrix $T$ is diagonal, the elements of $T^{-1}$
	\begin{equation*}
		\Phi_{ij}^{-1} = \frac{1}{T_{ii}} = \frac{1}{m} \quad (\text{no summation})
	\end{equation*}
	If the Staeckel conditions are satisfied, then Hamilton's characteristic function is completely separable.
	\begin{equation*}
		W(q) = \sum_i W_i(q_i)
	\end{equation*}
	with $W_i$ satisfying
	\begin{equation*}
		\left(\frac{\partial W_i}{\partial q_i} - a_i\right)^2 = -2V_i(q_i) + 2\Phi_{ij} \gamma_j
	\end{equation*}
	where $\gamma_j$ are constants of integration.
	
	\subsection*{}
	\textbf{5.} We first consider the central force problem in terms of the polar coordinates $(r, \psi)$ in the plane of the orbit.
	\begin{equation*}
		H = \frac{1}{2m} (p_r^2 + \frac{p_\psi^2}{r^2}) + V(r)
	\end{equation*}
	which is cyclic in $\psi$.
	Hamilton's characteristic function
	\begin{equation*}
		W = W(r) + \alpha_\psi \psi
	\end{equation*}
	where $\alpha_\psi$ is the constant angular momentum $p_\psi$ conjugate to $\psi$.
	The H-J equation becomes
	\begin{equation*}
		\left(\frac{\partial W_r}{\partial r}\right)^2 + \frac{\alpha_\psi^2}{r^2} + 2m V(r) = 2m\alpha_1
	\end{equation*}
	where $\alpha_1$ is the total energy.
	Then we obtain the solution
	\begin{equation*}
		\frac{dW_r}{dr} = \sqrt{2m(\alpha_1 - V) - \frac{\alpha_\psi^2}{r^2}}
	\end{equation*}
	\begin{equation*}
		\Rightarrow W = \int dr \sqrt{2m(\alpha_1 - V) - \frac{\alpha_\psi^2}{r^2}} + \alpha_\psi \psi
	\end{equation*}
	\section*{10.6 Action-Angle Variables in Systems of one Degree of Freedom}
	\begin{enumerate}
		\item We now consider a conserved system with one degree of freedom so that
		\[ H(q,p) = \alpha_1 \]
		Solving for the momentum, we have that
		\[ p = p(q, \alpha_1) \]
		Two types of periodic motion may be distinguished
		\begin{enumerate}
			\item $p, q$ are both functions of time that are periodic with the same frequency. \\
			$\Rightarrow$ libration (a)
			\item $p$ is some periodic function of $q$ with period $q_0$. \\
			$\Rightarrow$ rotation (b) \\
			
			\begin{figure}[h]
				\centering
				\includegraphics[width=0.7\linewidth]{figure2}
				\caption{}
				\label{fig:figure2}
			\end{figure}
			
		\end{enumerate}
		We may consider the simple pendulum where $q$ is the angle of deflection $\theta$. The equation with the length $l$ and the potential energy taken as 0 at the point of suspension gives the total energy
		\[ E = \frac{p_\theta^2}{2ml^2} - mgl \cos\theta \quad \text{or} \quad p_\theta = \sqrt{2m^2l^2(E+mgl\cos\theta)} \]
		In matrix notation, we consider a diagonal T
		\[ T_{ii} = m, \quad T_{\theta\theta} = ml^2, \quad T_{\phi\phi} = mr^2\sin^2\theta \]
		\[ \Rightarrow V(q) = V_r(r) + \frac{V_\theta(\theta)}{r^2} + \frac{V_\phi(\phi)}{r^2\sin^2\theta} \]
		If $E < mgl$, then the physical motion of the system can only occur for $|\theta| < \arccos(-\frac{E}{mgl}) = \theta_0$. \\
		$\Rightarrow$ libration \\
		
		\begin{figure}[h]
			\centering
			\includegraphics[width=0.7\linewidth]{figure3}
			\caption{}
			\label{fig:figure3}
		\end{figure}
		
		If $E > mgl$, all $\theta$ correspond to physical motion and $\theta$ can increase without limit. \\
		$\Rightarrow$ rotation
		
		\item For both types of motion, we introduce a new variable $J$ to replace $\alpha_1$ as the transformed constant momentum.
		The action variable:
		\[ J = \oint p \, dq \]
		By $p=p(q, \alpha_1)$ we note that $J$ is always some function of $\alpha_1$.
		\[ \alpha_1 = H = H(J) \]
		Hamilton's characteristic function can be written
		\[ W = W(q, J) \]
		And the generalized coordinate conjugate to $J$ known as the angle variable is defined by
		\[ w = \frac{\partial W}{\partial J} \]
		The equation of motion for $w$ is
		\[ \dot{w} = \frac{\partial H(J)}{\partial J} = \nu(J) \]
		where $\nu$ is a constant function of $J$ only. The solution is
		\[ w = \nu t + \epsilon \]
		
		\item Consider the change in $w$ as $q$ goes through a complete cycle of libration or rotation.
		\[ \Delta w = \oint \frac{\partial w}{\partial q} \, dq \]
		\[ = \oint \frac{\partial^2 W}{\partial q \partial J} \, dq \]
		Since $J$ is a constant,
		\[ \Delta w = \frac{d}{dJ} \oint p \, dq = \frac{dJ}{dJ} = 1 \]
		Which means that $w$ changes by unity as $q$ goes through a complete period. If $\tau$ is the period of $q$, then
		\[ \Delta w = 1 = \nu\tau \]
		where $\nu = \frac{1}{\tau}$ is the reciprocal of the period which is the frequency associated with $q$.
		
		\item We consider the linear harmonic oscillator
		\[ J = \oint p \, dq = \oint \sqrt{2m\alpha - m^2\omega^2 q^2} \, dq \]
		where $\alpha$ is the constant total energy and $\omega = \sqrt{\frac{k}{m}}$. By substitution $q = \sqrt{\frac{2\alpha}{m\omega^2}} \sin\theta$,
		we obtain
		\[ J = \frac{2\alpha}{\omega} \int_0^{2\pi} \cos^2\theta \, d\theta = \frac{2\pi\alpha}{\omega} \]
		or
		\[ \alpha = H = \frac{J\omega}{2\pi} \]
		The frequency is
		\[ \frac{\partial H}{\partial J} = \nu = \frac{\omega}{2\pi} = \frac{1}{2\pi}\sqrt{\frac{k}{m}} \]
		By letting $2\pi w = \omega t + \epsilon$,
		\[ q = \sqrt{\frac{J}{\pi m\omega}} \sin(2\pi w) \]
		\[ p = \sqrt{\frac{m\omega J}{\pi}} \cos(2\pi w) \]
	\end{enumerate}
	
	\subsection*{Further Examination of the Central Force Problem}
	The transformation equations are
	\[ t + \beta_1 = \frac{\partial W}{\partial \alpha_1} = \int \frac{m \, dr}{\sqrt{2m(\alpha_1-V) - \frac{\alpha_\theta^2}{r^2}}} \]
	\[ \beta_2 = \frac{\partial W}{\partial \alpha_\theta} = -\int \frac{\alpha_\theta \, dr}{r^2 \sqrt{2m(\alpha_1-V) - \frac{\alpha_\theta^2}{r^2}}} + \psi \]
	By $u=\frac{1}{r}$, we can reduce to
	\[ \psi = \beta_2 - \int \frac{du}{\sqrt{\frac{2m(\alpha_1-V)}{l^2} - u^2}} \]
	
	\begin{enumerate}
		\setcounter{enumi}{5}
		\item We further examine the same central force problem in spherical polar coordinates.
		\[ H = \frac{1}{2m} \left( p_r^2 + \frac{p_\theta^2}{r^2} + \frac{p_\phi^2}{r^2\sin^2\theta} \right) + V(r) \]
		\[ W = W_r(r) + W_\theta(\theta) + W_\phi(\phi) \]
		where $\phi$ is cyclic $\Rightarrow W_\phi = \alpha_\phi \phi$.
		The H-J equation
		\[ \left(\frac{\partial W_r}{\partial r}\right)^2 + \frac{1}{r^2}\left[ \left(\frac{\partial W_\theta}{\partial \theta}\right)^2 + \frac{\alpha_\phi^2}{\sin^2\theta} \right] + 2mV(r) = 2mE \]
		We note that
		\[ \left(\frac{\partial W_\theta}{\partial \theta}\right)^2 + \frac{\alpha_\phi^2}{\sin^2\theta} = \alpha_\theta^2 \]
		\[ \Rightarrow \left(\frac{\partial W_r}{\partial r}\right)^2 + \frac{\alpha_\theta^2}{r^2} = 2m(E-V(r)) \]
		Note that $\alpha_\phi = p_\phi = \frac{\partial W_\phi}{\partial \phi}$. And $p_\theta^2 + \frac{p_\phi^2}{\sin^2\theta} = \alpha_\theta^2$.
		So that
		\[ H = \frac{1}{2m} \left( p_r^2 + \frac{\alpha_\theta^2}{r^2} \right) + V(r) \]
		We see that $\alpha_\theta = p_\theta = L$, $\alpha_1 = E$.
	\end{enumerate}
	\section*{10.7 Action-Angle Variables for Completely Seperable Systems}
	(No summation)
	
	\subsection*{Complete separability}
	Complete separability means:
	\[ p_i = \frac{\partial W_i(q_i; \alpha_1, \dots, \alpha_n)}{\partial q_i} \]
	\[ = p_i(q_i; \alpha_1, \dots, \alpha_n) \]
	The action variables $J_i$ are defined by
	\[ J_i = \oint p_i dq_i \]
	For all types of rotations, the period is $2\pi$, then $J_i = 2\pi p_i$ for all cyclic variables whose conjugate momentums are constants.
	
	It can also be written as
	\[ J = \oint \frac{\partial W_i(q_i; \alpha_1, \dots, \alpha_n)}{\partial q_i} dq_i \]
	Expressing $\alpha_i$'s in terms of $J_i$'s
	\[ W = W(q_1, \dots, q_n; J_1, \dots, J_n) = \sum_i W_j(q_j; J_1, \dots, J_n) \]
	while the Hamiltonian
	\[ H = \alpha_1 = H(J_1, \dots, J_n) \]
	In the system of one degree of freedom, we define conjugate angle variables $w_i$ by
	\[ w_i = \frac{\partial W}{\partial J_i} = \sum_j \frac{\partial W_j(q_j; J_1, \dots, J_n)}{\partial J_i} \]
	or
	\[ w_i = w_i(q_1, \dots, q_n; J_1, \dots, J_n) \]
	and the equations of motion:
	\[ \dot{w_i} = \frac{\partial H(J_1, \dots, J_n)}{\partial J_i} = \nu_i(J_1, \dots, J_n) \]
	where $\nu_i$'s are constants.
	\[ \Rightarrow w_i = \nu_i t + \delta_i \]
	
	\subsection*{Infinitesimal change}
	We consider an infinitesimal change denoted by $\delta$.
	\[ \delta W_i = \sum_j \frac{\partial W_i}{\partial J_j} dJ_j = \sum_j \int \frac{\partial p_i(q_i, J)}{\partial J_j} dq_i dJ_j \]
	\[ = \frac{\partial}{\partial J_j} \oint p_i(q_i, J) dJ_j \]
	The total change in $w_i$ is therefore
	\[ \Delta w_i = \frac{\partial}{\partial J_i} \oint p_j(q_j, J) dJ_j \]
	Since the $J_i$'s are all independent constants,
	\[ \Delta w_i = m_i \]
	or $\Delta \vec{w} = \vec{m}$.
	
	We assume all separable motions are libration. We define $\vec{J}$ as a function of $\vec{w}$ so that $\Delta \vec{J}=0$ corresponds to $\Delta \vec{w} = \vec{m}$.
	Since the number of cycles in the chosen motion of $q_j$ are arbitrary, $m_j$ can be taken as zero except for $m=1$. And all components of $q$ remain unchanged or return to their original values. $\Rightarrow$ some periodic functions of $w_i$ whose periods are unity.
	
	\subsection*{Fourier Expansion}
	By Fourier expansion:
	\[ q_k = \sum_{j_1=-\infty}^{\infty} \dots \sum_{j_n=-\infty}^{\infty} a_{j_1, \dots, j_n}^{(k)} e^{2\pi i (j_1 w_1 + \dots + j_n w_n)} \]
	(Libration)
	where we can view $j$'s as a vector in the same n-dimensional space with $\vec{w}$.
	\[ \Rightarrow q_k = \sum_j a_j^{(k)} e^{2\pi i \vec{j} \cdot \vec{w}} \quad (\text{Libration}) \]
	If we write: $\vec{w} = \vec{\nu}t + \vec{\delta}$
	\[ q_k(t) = \sum_j a_j^{(k)} e^{2\pi i \vec{j} \cdot (\vec{\nu}t + \vec{\delta})} \quad (\text{Libration}) \]
	where
	\[ a_j^{(k)} = \int_0^1 \dots \int_0^1 q_k(\vec{w}) e^{-2\pi i \vec{j} \cdot \vec{w}} (d\vec{w}) \]
	We note that $w_k$ increased by unity during the cycle so $q_k - w_k q_k$ returns to its initial value.
	\[ q_k - w_k q_k = \sum_j a_j^{(k)} e^{2\pi i \vec{j} \cdot \vec{w}} \quad (\text{Rotation}) \]
	or
	\[ q_k = Q_k(v_k t + B_k) + \sum_j a_j^{(k)} e^{2\pi i \vec{j} \cdot (\vec{\nu}t + \vec{\delta})} \]
	Thus, for multiply periodic functions of the $w$'s
	\[ f(q,p) = \sum_j b_j e^{2\pi i \vec{j} \cdot \vec{w}} = \sum_j b_j e^{2\pi i \vec{j} \cdot (\vec{\nu}t + \vec{\delta})} \]
	
	\subsection*{2D Anisotropic Harmonic Oscillator}
	We consider a 2-dimensional anisotropic harmonic oscillator. In Cartesian coordinates, the Hamiltonian is
	\[ H = \frac{1}{2m} [ (p_x^2 + p_y^2 + m^2(\nu_x^2 x^2 + \nu_y^2 y^2)) + (p_x'^2 + p_y'^2 + m^2(\nu_x'^2 x'^2 + \nu_y'^2 y'^2)) ] \]
	Suppose the coordinates are rotated by $\alpha$ about the z-axis.
	\begin{align*}
		x' &= \frac{1}{\sqrt{2}} [x_0 \cos \alpha \pi (\nu_x t + \beta_x) + y_0 \cos \alpha \pi (\nu_y t + \beta_y)] \\
		y' &= \frac{1}{\sqrt{2}} [y_0 \cos \alpha \pi (\nu_y t + \beta_y) - x_0 \cos \alpha \pi (\nu_x t + \beta_x)]
	\end{align*}
	If $\nu_x / \nu_y$ is a rational number, these two expressions will be commensurate, corresponding to closed Lissajous figures. 
	
	\begin{figure}[h]
		\centering
		\includegraphics[width=0.7\linewidth]{figure4}
		\caption{}
		\label{fig:figure4}
	\end{figure}
	
	As the time interval $T$ containing $m$ complete cycles of $q_k$ plus a fraction of a cycle increases indefinitely.
	\[ \lim_{T\to\infty} \frac{m}{T} = \nu_k \]
	As $q_k$ goes through a complete cycle, when $w_k$ changes by a unity, the characteristic function increases by $J_k$.
	\[ W' = W - \sum_k w_k J_k \]
	remains unchanged when $w_k$ is increased by unity where $w_k = \frac{\partial W}{\partial J_k}$.
	
	\subsection*{Degeneracy Conditions}
	We now consider $m$ degeneracy conditions
	\[ \sum_{k=1}^n j_{ki} \nu_i = 0, \quad k=1, \dots, m. \]
	Considering $(w, J) \to (w', J')$, the generating function is
	\[ F_2 = \sum_{i=1}^m \sum_{k=1}^n J'_i j_{ki} w_k + \sum_{k=m+1}^n J'_k w_k \]
	\[ w_k' = \sum_{i=1}^m J'_i j_{ki}, \quad k=1, \dots, m \]
	\[ w_k' = w_k, \quad k=m+1, \dots, n \]
	The new frequencies
	\[ \nu_i' = \dot{w}_k' = \sum_{j=1}^n j_{ki} \nu_j = 0, \quad k=1, \dots, m \]
	\[ \nu_k' = \nu_k, \quad k=m+1, \dots, n \]
	where $w_k'$ are the angle variables.
	And the corresponding constant action variables
	\[ J_i = \sum_{k=1}^m j_{ik} J'_k + \sum_{k=m+1}^n J'_k \delta_{ki} \]
	Since $\nu_i' = \frac{\partial H}{\partial J_i'}$, the Hamiltonian must be independent of $J_i'$.
	As for $(w, p) \to (w', J')$, we have
	\[ w_i' = \frac{\partial W}{\partial J_i'} \]
	
\end{document}