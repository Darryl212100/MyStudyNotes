\documentclass{article}
\usepackage{amsmath}
\usepackage{amsfonts}
\usepackage{amssymb}
\usepackage{graphicx}
\usepackage{geometry}
\usepackage{braket}
\usepackage{physics}
\usepackage{bm}

\begin{document}
	
	\section*{VII The Classical Mechanics of the Special Theory of Relativity}
	
	\subsection*{7.1 Basic Postulates of the Special Theory}
	\begin{enumerate}
		\item The laws of physics are the same to all inertial observers.
		\item The speed of light is the same to all inertial observers.
	\end{enumerate}
	
	In the spacetime which satisfies the two postulates, the square of the distance $\Delta s^2$ between A and B
	\[ (\Delta s)^2 = c^2(\text{time interval})^2 - (\text{space interval})^2 \]
	We refer to a point in spacetime as an event.
	
	If
	\begin{itemize}
		\item[$(ds)^2 > 0$] the interval is timelike
		\item[$(ds)^2 < 0$] the interval is spacelike
		\item[$(ds)^2 = 0$] the interval is null/lightlike
	\end{itemize}
	
	If, to all inertial observers, objects that travel on timelike paths move less than the speed of light, they're called tardyons.
	If objects travel on spacelike paths and move faster than the speed of light, they're called tachyons.
	
	In the limit of small displacement, in a Cartesian coordinate system,
	\[ (ds)^2 = (cdt)^2 - (dx^2 + dy^2 + dz^2) \]
	
	The intervals between two events of spacetime in S and S' satisfy
	\[ ds'^2 = ds^2 \]
	which is called the square of the invariant spacetime interval.
	
	We call the time interval measured by a clock at rest with respect to a body the \textbf{proper time} of that body's frame, while the other inertial observer uses a time that is often called \textbf{laboratory time}.
	
	S' with $(t', x', y', z')$ is moving at a velocity $\vec{v}$ with respect to S with $(t, x, y, z)$.
	So:
	\[ c^2(d\tau)^2 = c^2(dt)^2 - v^2(dt)^2 = c^2(dt)^2(1 - \frac{v^2}{c^2}) \]
	or
	\[ dt = \frac{d\tau}{\sqrt{1 - v^2/c^2}} \]
	Since $dt > d\tau$, the effect on $dt$ is time dilation.
	
	\begin{figure}[h]
		\centering
		\includegraphics[width=1.0\linewidth]{figure1}
		\caption{}
		\label{fig:figure1}
	\end{figure}
	
	For event A at time $t_A$ and event B at time $t_B$. A is located at $x=y=z=0$.
	
	If $(ds_{AB})^2 > 0$, then all inertial observers will agree on the time order of the events.
	If $t_B$ is less than $t_A$ in one inertial frame, then $t_B$ is less than $t_A$ in all inertial frames. The region is the \textbf{past}.
	If for event C, $t_C$ is greater than $t_A$ for all inertial observers, the region is the \textbf{future}.
	
	If $(ds_{AD})^2 < 0$, there exist a set of inertial frames in which the relative order of $t_A$ and $t_D$ can be reversed or even made equal. The region can be referred as the \textbf{elsewhere} or the \textbf{past-future}.
	Separating the past-future and the elsewhere is the \textbf{null} or \textbf{light cone} where $ds^2=0$.
	
	\subsection*{7.2 Lorentz Transformations}
	We consider parallel Cartesian coordinate systems, S, S', whose origins coincide at $t=t'=0$, and whose relative velocity is $v$ along the $x$-axis as measured by S.
	We define: $\beta = \frac{v}{c}$, $\gamma = \frac{1}{\sqrt{1-\beta^2}}$.
	We have the two sets of coordinates:
	\[ ct' = \frac{ct - \beta x}{\sqrt{1-\beta^2}} = \gamma(ct - \beta x) \]
	
	Tip: $\Delta t' = \frac{\Delta t}{\sqrt{1-v^2/c^2}} \implies t' = \gamma \Delta t$. We assume that
	\[
	\begin{cases}
		t' = At + Bx \\
		x' = Ct + Dx
	\end{cases}
	\text{for S' to S}
	\]
	
	\[ x' = ct' \implies Ct + DC = CA + BC^2 \implies
	\begin{cases}
		A = \gamma \\
		B = -\frac{\gamma v}{c^2}
	\end{cases}
	\]
	\[ x' = \frac{x-\beta c t}{\sqrt{1-\beta^2}} = \gamma(x-\beta c t) \]
	\[ y' = y, \quad z' = z \]
	
	We only consider transformations for which $t' \to t, x' \to x$ as $\beta \to 0$.
	
	As matrices:
	\[
	\begin{pmatrix} ct' \\ x' \\ y' \\ z' \end{pmatrix}
	=
	\begin{pmatrix}
		\gamma & -\gamma\beta & 0 & 0 \\
		-\gamma\beta & \gamma & 0 & 0 \\
		0 & 0 & 1 & 0 \\
		0 & 0 & 0 & 1
	\end{pmatrix}
	\begin{pmatrix} ct \\ x \\ y \\ z \end{pmatrix}
	\]
	
	In the limit $\beta \ll 1$, we have the Galilean transformations.
	
	We use the notation $(\vec{r}, ct) \equiv (x,y,z,ct)$. For the case where $\vec{v}$ is not parallel to an axis:
	\begin{align*}
		ct' &= \gamma(ct - \vec{\beta} \cdot \vec{r}) \\
		\vec{r}' &= \vec{r} + \frac{(\vec{\beta} \cdot \vec{r})}{\beta^2}\vec{\beta}(\gamma-1) - \gamma\vec{\beta}ct
	\end{align*}
	
	Thus, as matrix, we have $\vec{x}' = L\vec{x}$ where
	\[
	L =
	\begin{pmatrix}
		\gamma & -\gamma\beta_x & -\gamma\beta_y & -\gamma\beta_z \\
		-\gamma\beta_x & 1+\frac{(\gamma-1)\beta_x^2}{\beta^2} & \frac{(\gamma-1)\beta_x\beta_y}{\beta^2} & \frac{(\gamma-1)\beta_x\beta_z}{\beta^2} \\
		-\gamma\beta_y & \frac{(\gamma-1)\beta_y\beta_x}{\beta^2} & 1+\frac{(\gamma-1)\beta_y^2}{\beta^2} & \frac{(\gamma-1)\beta_y\beta_z}{\beta^2} \\
		-\gamma\beta_z & \frac{(\gamma-1)\beta_z\beta_x}{\beta^2} & \frac{(\gamma-1)\beta_z\beta_y}{\beta^2} & 1+\frac{(\gamma-1)\beta_z^2}{\beta^2}
	\end{pmatrix}
	\]
	
	For a more general form:
	\[ \vec{x}' = L\vec{x} + \vec{a} \]
	where $L$ is a rotation of spacetime and $\vec{a}$ is a spacetime translation, which is the Poincaré transformation.
	
	\subsection*{7.3 Velocity Addition and Thomas Precession}
	Any homogeneous Lorentz transformation, L, can be written as:
	\[ L = R L_0 = L_0' R' \]
	where R is a rotation matrix and $L_0$ is called a restricted or proper Lorentz transformation, corresponds to a pure boost.
	Since R is not symmetric and $L_0$ is symmetric, L will have no symmetry, thus $R L_0 \neq L_0 R$.
	\[ \implies R L_0 = L_0' R' \]
	\subsection*{Continuation of 7.3}
	For any L, we have an inverse transformation $L^{-1}$ such that $L L^{-1} = L^{-1} L = I$ where I is the diagonal unit 4x4 matrix with elements $\delta_{\alpha\beta}$.
	
	\begin{enumerate}
		\item We consider $S_1, S_2, S_3$, with x axes aligned. Let $S_2$ be moving at a velocity $v$ along the common x-direction with respect to $S_1$, and let $S_3$ be moving at a velocity $v'$ along the common x-direction with respect to $S_2$.
		
		From $S_1$ to $S_3$,
		\[ L_{1 \to 3} = 
		\begin{pmatrix}
			\gamma' & -\gamma'\beta' & 0 & 0 \\
			-\gamma'\beta' & \gamma' & 0 & 0 \\
			0 & 0 & 1 & 0 \\
			0 & 0 & 0 & 1
		\end{pmatrix}
		\begin{pmatrix}
			\gamma & -\gamma\beta & 0 & 0 \\
			-\gamma\beta & \gamma & 0 & 0 \\
			0 & 0 & 1 & 0 \\
			0 & 0 & 0 & 1
		\end{pmatrix}
		\]
		where we define $\beta$ and $\gamma$ for $v$, and $\beta'$ and $\gamma'$ for $v'$.
		
		Let $\beta''$ be the speed of $S_3$ relative to $S_1$, and $\gamma''$ the associated factor.
		\[ L_{1 \to 3} = 
		\begin{pmatrix}
			\gamma'' & -\gamma''\beta'' & 0 & 0 \\
			-\gamma''\beta'' & \gamma'' & 0 & 0 \\
			0 & 0 & 1 & 0 \\
			0 & 0 & 0 & 1
		\end{pmatrix}
		\]
		where
		\[ \gamma'' = \gamma\gamma'(1+\beta\beta') \]
		\[ \beta'' = \frac{\beta+\beta'}{1+\beta\beta'} \]
		Thus, we have the \textbf{Thomas precession rotation}.
		
		\item We consider $S_1, S_2, S_3$ with $S_2$ moving at a velocity $\vec{v}$ with respect to $S_1$, and $S_3$ moving at a velocity $\vec{v}'$ with respect to $S_2$. We arrange that $\vec{\beta}$ is along the x-axis of $S_1$ and $\vec{\beta}'$ lies in the x'y' plane of $S_2$.
		Let L represent the transformation from $S_1$ to $S_2$ and L' the transformation from $S_2$ to $S_3$ with $\gamma$ and $\gamma'$ associated with $\vec{\beta}$ and $\vec{\beta}'$.
		\[ L = 
		\begin{pmatrix}
			\gamma & -\gamma\beta & 0 & 0 \\
			-\gamma\beta & \gamma & 0 & 0 \\
			0 & 0 & 1 & 0 \\
			0 & 0 & 0 & 1
		\end{pmatrix}
		\text{ and }
		L' = 
		\begin{pmatrix}
			\gamma' & -\gamma'\beta'_x & -\gamma'\beta'_y & 0 \\
			-\gamma'\beta'_x & 1+\frac{(\gamma'-1)\beta_x'^2}{\beta'^2} & \frac{(\gamma'-1)\beta'_x\beta'_y}{\beta'^2} & 0 \\
			-\gamma'\beta'_y & \frac{(\gamma'-1)\beta'_y\beta'_x}{\beta'^2} & 1+\frac{(\gamma'-1)\beta_y'^2}{\beta'^2} & 0 \\
			0 & 0 & 0 & 1
		\end{pmatrix}
		\]
		We assume that the components of $\vec{\beta}'$ are small.
		\[ L'' = L'L = 
		\begin{pmatrix}
			\gamma\gamma' & -\gamma\gamma'\beta & -\gamma'\beta'_y & 0 \\
			-\gamma\beta & \gamma\beta & \gamma\beta'_y & 0 \\
			-\gamma'\beta'_y & \gamma\beta'\beta_y' & \gamma' & 0 \\
			0 & 0 & 0 & 1
		\end{pmatrix}
		\]
		which is not symmetric. We shall write the velocity of $S_3$ as observed from $S_1$ as $\vec{\beta}''$. Since the off-diagonal elements corresponding to the z-axis are zero, this rotation is about an axis perpendicular to the xy plane. The boost from $S_1$ to $S_3$ is denoted by $\vec{\beta}''$, and we assume $\vec{\beta}'$ is small compared to $\vec{\beta}$ and $\gamma' \approx 1$.
		
		Then, to first order, the non-vanishing components of $\vec{\beta}''$ are:
		\[ \beta_x'' = \beta, \quad \beta_y'' = \frac{\beta_y'}{\gamma}, \quad \beta''^2 = \beta^2, \quad \gamma'' = \gamma \]
		and
		\[ L'' \approx 
		\begin{pmatrix}
			\gamma & -\gamma\beta & -\beta_y' & 0 \\
			-\gamma\beta & \gamma & \gamma^{-1}\beta_y' & 0 \\
			-\gamma\beta_y' & \gamma\beta_y'\beta & 1 & 0 \\
			0 & 0 & 0 & 1
		\end{pmatrix}
		\]
		A pure Lorentz transformation from $S_3$ to $S_1$, $L_{3\to1}$, would correspond to a large boost of $-\beta_x''$ and a small boost in the y'' axis of $-\beta_y''$.
		\[ L_{3\to1} = 
		\begin{pmatrix}
			\gamma'' & \gamma''\beta_x'' & \gamma''\beta_y'' & 0 \\
			\gamma''\beta_x'' & 1+\frac{(\gamma''-1)\beta_x''^2}{\beta''^2} & \frac{(\gamma''-1)\beta_x''\beta_y''}{\beta''^2} & 0 \\
			\gamma''\beta_y'' & \frac{(\gamma''-1)\beta_y''\beta_x''}{\beta''^2} & 1 & 0 \\
			0 & 0 & 0 & 1
		\end{pmatrix}
		\]
		In terms of $\vec{\beta}'$, the rotation matrix is
		\[ R = L'' L_{3\to1} = 
		\begin{pmatrix}
			1 & 0 & 0 & 0 \\
			0 & 1 & -(\gamma-1)\frac{\beta_y'}{\beta} & 0 \\
			0 & (\gamma-1)\frac{\beta_y'}{\beta} & 1 & 0 \\
			0 & 0 & 0 & 1
		\end{pmatrix}
		\]
		which implies $S_3$ is rotated with respect to $S_1$ about the z-axis through an infinitesimal angle
		\[ \Delta\Omega = (\gamma-1)\frac{\beta_y'}{\beta} = \frac{\gamma^2\beta}{c}(\frac{\gamma-1}{\gamma\beta^2}) \]
		
		\item Suppose $S_1$ is a laboratory system, while $S_2, S_3$ are two of the instantaneous rest systems a time $\Delta t$ apart in the particle's motion.
		By $\Delta\Omega$, we can see a change in the velocity $\Delta\vec{v}$, which has only a y-component since the initial x-axis has been chosen along the direction of $\vec{v} = \vec{\beta}c$.
		The infinitesimal rotation in this time can be written as
		\[ \Delta\vec{\Omega} = -(\gamma-1)\frac{\vec{v} \times \Delta\vec{v}}{v^2} \]
		Observed from the Laboratory system, the direction attached to the particle, precesses with:
		\[ \vec{\omega} = \frac{d\vec{\Omega}}{dt} = -(\gamma-1)\frac{\vec{v} \times \vec{a}}{v^2} \]
		where $\vec{a}'$ is the particle's acceleration seen from $S_1$.
		When $v$ is small enough that $\gamma \approx 1 + \frac{1}{2}\beta^2$
		\[ \vec{\omega} = -\frac{1}{2c^2}(\vec{a} \times \vec{v}) \]
		which is the \textbf{Thomas precession frequency}.
	\end{enumerate}
    \section*{7.4 Vectors and the Metric Tensor}
    In terms of the parameter $\lambda$ which is a measure of the length along the curve from A to B
    $$ V_{AB} = \left(\frac{dP}{d\lambda}\right)_{\lambda=0} = P_B - P_A $$
    where $P$ is an arbitrary one-dimensional curve in 4-dimensional spacetime.
    
    The components of the 4-vector which is a tangent vector to the curve are $v^0, v^1, v^2, v^3$. For curves that are timelike, the proper time $\tau$ in the laboratory coordinates are: $x^0 = ct(\tau)$, $x^1 = X(\tau)$, $x^2 = Y(\tau)$, $x^3 = Z(\tau)$, and the tangent to the curve is the four-velocity $u$.
    Hence, $u^0 = \frac{dx^0}{d\tau} = \gamma c$, $u^i = \frac{dx^i}{d\tau} = \gamma v_i$ where $v_i = dx^i/dt$ is the normal three-velocity with $v^2 = (v^1)^2 + (v^2)^2 + (v^3)^2$.
    
    We consider a set of vectors of 4-velocities for the particle termed a vector field: $e_0, e_1, e_2, e_3$ which constitute a coordinates basis.
    The position of a point on the curve $P(\tau)$ can be written as $P(\tau) = x^\mu(\tau) e_\mu$.
    
    \begin{figure}[h]
    	\centering
    	\includegraphics[width=1.0\linewidth]{figure2}
    	\caption{}
    	\label{fig:figure2}
    \end{figure}
    
    The 4-velocity $u = \frac{dP}{d\tau} = \frac{dx^\mu}{d\tau} e_\mu = u^\mu e_\mu$.
    The magnitude of which is a scalar whose values can vary to $\lambda$. This set of magnitudes is an example of a scalar field.
    The conversion of a vector field to a scalar field is an example of mapping.
    The metric tensor $g$ can be defined by:
    $$ g(u, v) = g(v, u) = u \cdot v $$
    which is the scalar product
    $$ \Rightarrow g_{\mu\nu} = g(e_\mu, e_\nu) = e_\mu \cdot e_\nu $$
    Example: we consider a two-dimensional Minkowski space with coordinate $ct$ and $x$ and a vector $v=(a,b)$, then $g(v,v) = a^2 - b^2$ and $g_{00}=1, g_{11}=-1$.
    
    We consider small displacements vectors between two points: $d\vec{s} = dx^\alpha e_\alpha$.
    For Minkowski coordinates,
    $$ (\Delta s)^2 = d\vec{s} \cdot d\vec{s} = \Delta x^\alpha e_\alpha \cdot \Delta x^\beta e_\beta = g_{\alpha\beta} \Delta x^\alpha \Delta x^\beta $$
    $$ = (c\Delta t)^2 - (\Delta x)^2 - (\Delta y)^2 - (\Delta z)^2 $$
    $$ ds^2 = g_{\alpha\beta} dx^\alpha dx^\beta $$
    
    Using the metric sign $ g = \begin{pmatrix} 1 & 0 & 0 & 0 \\ 0 & -1 & 0 & 0 \\ 0 & 0 & -1 & 0 \\ 0 & 0 & 0 & -1 \end{pmatrix} $.
    $$ u \cdot v = u^\alpha v^\beta g_{\alpha\beta} = u^0 v^0 - u^1 v^1 - u^2 v^2 - u^3 v^3 $$
    $$ u \cdot u = c^2 \quad \text{is the square of the magnitude of the four-velocity} $$
    The 4-momentum can be defined by: $P = m u$.
    So the length squared of it is: $P \cdot P = m^2 c^2$.
    Or: $P \cdot P = m^2 c^2 = m^2 c^2 \gamma^2 - m^2 v^2 \gamma^2 = \frac{E^2}{c^2} - \vec{p}^2$ where $\vec{p}$ is the length of the 3-momentum.
    We obtain $E^2 = m^2 c^4 + \vec{p}^2 c^2$.
    
    The relativistic kinetic energy $T$ is defined as:
    $$ T = E - mc^2 = mc^2(\gamma - 1) = \sqrt{(m c^2)^2 + \vec{p}^2 c^2} - m c^2 $$
    For $|\vec{p}| \ll m c$, a power series expansion gives
    $$ T \approx \frac{1}{2} m v^2 + O(v^4) $$
    
    \section*{7.5 1-Forms and Tensors}
    If the vector is thought of as a directed line, the 1-form is a set of numbered surfaces though which the vector passes.
    
    \begin{figure}[h]
    	\centering
    	\includegraphics[width=1.0\linewidth]{figure3}
    	\caption{}
    	\label{fig:figure3}
    \end{figure}
    
    If $\eta$ is a 1-form and $v$ is some vector the quantity denoted by $\langle \eta, v \rangle$ is a number that tells how many surfaces of $\eta$ are pierced by $v$.
    Such that $\langle v g, v \rangle = v \cdot v$ is the scalar contraction or the square of the magnitude of $v$.
    
    The gradient is a 1-form, if we consider $P$ with $\lambda$, where $x=0$ at $P_0$.
    $$ \partial_\lambda f = \frac{d}{d\lambda} f(P(\lambda)) = \frac{df}{dx^\alpha} \frac{dx^\alpha}{d\lambda} = v^\alpha \partial_\alpha f $$
    So: $\partial_\alpha = \partial e_\alpha = \frac{\partial}{\partial x^\alpha}$.
    
    \begin{figure}[h]
    	\centering
    	\includegraphics[width=1.0\linewidth]{figure4}
    	\caption{}
    	\label{fig:figure4}
    \end{figure}
    
    The gradient of the coordinates is defined as $\omega^\alpha = dx^\alpha$.
    $\omega^\alpha$ provides a set of basis 1-forms since $\langle \omega^\alpha, e_\beta \rangle = \delta^\alpha_\beta$.
    And any 1-form $\eta$ can be written as $\eta = \eta_\alpha \omega^\alpha$ where $\langle \eta, e_\alpha \rangle = \eta_\alpha$.
    For any vector $v$: $\langle \eta, v \rangle = \eta_\alpha v^\alpha$.
    
    We define the inverse metric by:
    $$ g^{\alpha\beta} g_{\beta\gamma} = \delta^\alpha_\gamma \quad \text{or} \quad g^{-1}g=gg^{-1}=I $$
    Thus: $u_d = g_{d\beta} u^\beta$, $u^\alpha = g^{\alpha\beta} u_\beta$.
    Therefore, $u \cdot v = g(u, v) = g_{\alpha\beta} u^\alpha v^\beta = u_\beta v^\beta = u_\alpha v^\alpha = u^\alpha v_\alpha$.
    
    We consider the vector $u$ with components $(a,b)$ the vector $v$ with $(c,d)$.
    \begin{align*}
    	g_{\alpha\beta} u^\alpha v^\beta &= (1)(a)(c) + (-1)(b)(d) = ac-bd \\
    	u^\alpha v_\alpha &= ac-bd \\
    	u_\alpha v^\alpha &= ac-bd
    \end{align*}
    
    For a vector $V = V^1 e_1 + V^2 e_2$, where $\{e_1, e_2\}$ with $e_1 \cdot e_1 = 1, e_2 \cdot e_2 = -1, e_1 \cdot e_2 = e_2 \cdot e_1 = 0$.
    Thus:
    \begin{align*}
    	V \cdot V &= \sum_{i,j=1}^2 V^i V^j e_i \cdot e_j = (V^1)^2 e_1 \cdot e_1 + V^1 V^2 e_1 \cdot e_2 + V^2 V^1 e_2 \cdot e_1 + (V^2)^2 e_2 \cdot e_2 \\
    	&= \sum_{i=1}^2 (V^i)^2 e_i \cdot e_i = (V^1)^2 - (V^2)^2
    \end{align*}
    We define the dual space with $\omega^1, \omega^2$ which have the properties:
    \begin{gather*}
    	\langle \omega^1, e_1 \rangle = 1, \quad \langle \omega^2, e_2 \rangle = 1 \\
    	\langle \omega^1, e_2 \rangle = \langle \omega^2, e_1 \rangle = e_1 \cdot \omega^1 = \omega^1 e_1 = 0 \\
    \end{gather*}
    So that $e_i$ is orthonormal to $\omega_i$.
    The 1-form $v$ corresponding to $V$ can be written as
    $$ v = V_1 \omega^1 + V_2 \omega^2 $$
    Whose (magnitude)$^2 = v \cdot v = V \cdot V = V_1 V^1 + V_2 V^2$.
    The scalar product of two vectors $V$ and $U$ in terms of 1-forms $v$ and $u$:
    Scalar product $V \cdot U = u \cdot V = U \cdot v = V \cdot u = V_\mu U^\mu = V^\mu U_\mu = V_1 U^1 + V_2 U^2$.
    
    Example: in a four-dimensional Minkowski space, the 1-form $v_\alpha$ with the vector $V_\beta$ is
    $$ v_0 = V^0, \quad v_1 = -V^1, \quad v_2 = -V^2, \quad v_3 = -V^3 $$
    So the squared length of $V$ is:
    $$ V^0 v_0 + V^1 v_1 + V^2 v_2 + V^3 v_3 = V^0 V^0 - V^1 V^1 - V^2 V^2 - V^3 V^3 $$
    
    We let $x^0, x^1, x^2, x^3$ be the coordinates in a frame $S$ and $x^{\alpha'} = x^{\alpha'}(x^0, x^1, x^2, x^3)$ be the transformed coordinates in the frame $s'$. The Lorentz transformation can be written as
    $$ x^{\alpha'} = L^{\alpha'}_{\ \beta} x^\beta \quad \text{and} \quad x^\alpha = L^\alpha_{\ \beta'} x^{\beta'} $$
    where $L^\alpha_{\ \beta'}$ is the inverse transformation of $L^{\alpha'}_{\ \beta}$.
    The basis vectors transform as
    $$ e_{\alpha'} = L^\beta_{\ \alpha'} e_\beta \quad \text{and} \quad e_\alpha = L^{\beta'}_{\ \alpha} e_{\beta'} $$
    Any vector transforms as $v = v^\alpha e_\alpha = v^{\beta'} e_{\beta'}$, so
    $$ \langle v, \eta \rangle = v^\alpha \eta_\alpha = v^{\alpha'} \eta_{\alpha'} $$
    This means that 1-forms transform as
    $$ \eta_{\alpha'} = \eta_\beta (L^{-1})^\beta_{\ \alpha'} \quad \text{and we have} $$
    $$ \omega^{\alpha'} = L^{\alpha'}_{\ \beta} \omega^\beta \quad \text{and} \quad \omega^\alpha = L^\alpha_{\ \beta'} \omega^{\beta'} $$
    It follows that
    $$ v^{\alpha'} = L^{\alpha'}_{\ \beta} v^\beta \quad \text{and} \quad v^\alpha = L^\alpha_{\ \beta'} v^{\beta'} $$
    $$ \eta_{\alpha'} = L^\beta_{\ \alpha'} \eta_\beta \quad \text{and} \quad \eta_\alpha = L^{\beta'}_{\ \alpha} \eta_{\beta'} $$
    
    For a tensor of rank $\binom{2}{3}, S$
    $$ S(\sigma, \omega, \lambda; v, u, e_\gamma) = \sigma_\alpha \lambda_\beta \omega_\epsilon v^\delta u^\gamma e_\gamma S^{\alpha\beta\epsilon}_{\ \ \ \ \ \delta\gamma} $$
    $$ = S^{\alpha\beta}_{\ \ \ \ \gamma\delta\epsilon} v^\gamma \sigma_\alpha \lambda_\beta $$
    where $S^{\alpha\beta\gamma}_{\ \ \ \ \delta\epsilon}$ are the components of $S$.
    We obtain the transformation law for them
    $$ S^{\alpha'\beta'}_{\ \ \ \ \ \gamma'} = S^{\alpha\beta}_{\ \ \ \ \gamma} L^{\alpha'}_{\ \alpha} L^{\beta'}_{\ \beta} L^\gamma_{\ \gamma'} $$
    The metric tensor can convert indices from vector to 1-form or 1-form to vector:
    $$ S^{\alpha\beta}_{\ \ \ \ \ \gamma} = g_{\gamma\epsilon} S^{\alpha\beta\epsilon} $$
    \section*{7.6 Forces in the Special Theory; Electromagnetism}
    
    \subsection*{}
    The generalized force from four-velocity
    $$ F^\mu = \frac{d(m_0 u^\mu)}{d\tau} $$
    
    The vector and scalar electromagnetic potentials form a four-vector: $A^\mu = (\phi/c, \vec{A})$
    
    If the potentials satisfy the Lorentz condition.
    $$ \Box A = \nabla \cdot A = \frac{\partial A^\mu}{\partial x^\mu} = \nabla \cdot \vec{A} + \mu_0 \epsilon_0 \frac{\partial \phi}{\partial t} = 0 $$
    where $\mu_0 \epsilon_0 = \frac{1}{c^2}$ and
    $$ \Box^2 \vec{A} = \nabla^2 \vec{A} = \frac{1}{c^2} \frac{\partial^2 \vec{A}}{\partial t^2} - \nabla^2 \vec{A} = \mu_0 \vec{j} $$
    for the space and time component
    $$ \Box^2 \phi = \nabla^2 \phi = \frac{1}{c^2} \frac{\partial^2 \phi}{\partial t^2} - \nabla^2 \phi = \frac{\rho}{\epsilon_0} $$
    
    The Lorentz force, in terms of $\phi, \vec{A}$:
    $$ \vec{F} = q \left\{ -\nabla \phi + \frac{\partial \vec{A}}{\partial t} + l \left[ \vec{v} \times (\nabla \times \vec{A}) \right] \right\} $$
    
    We generalize the Lorentz force law to:
    $$ \frac{dp_\mu}{d\tau} = q \left( u^\nu \frac{\partial A_\nu}{\partial x^\mu} - \frac{dA_\mu}{d\tau} \right) $$
    
    For the three-momentum, $\vec{p}$, and three-velocity $\vec{v}$
    $$ \frac{d\vec{p}}{dt} = q(\vec{E} + \vec{v} \times \vec{B}) $$
    
    We define a tensor $F$, named Faraday tensor, whose components will be the electromagnetic field tensor and with $u$ the 4-velocity
    $$ \frac{dP}{dT} = q F(u) $$
    In component notation:
    $$ \frac{dp^\mu}{d\tau} = q F^\mu_{\ \nu} u^\nu $$
    
    We consider a two-dimensional with a vector $u$, whose components are $(a,b)$ and a 1-form $\sigma$ with components $(c,d)$, then we examine a tensor $W$ of rank $\binom{1}{1}$
    $$ W(\sigma, u) = W^\alpha_\beta \sigma_\alpha u^\beta = W^0_0 c a + W^1_0 c b + W^0_1 d a + W^1_1 d b $$
    
    In a Minkowski space with pseudo-Cartesian coordinates, the components of the tensor $W$ of rank $\binom{1}{1}$ can be converted to a corresponding tensor of rank $\binom{0}{2}$ using $\{g_{00}=1, g_{11}=-1, g_{12}=-1, g_{33}=-1\}$ to give the following relations:
    $$ W_{00} = g_{0\alpha} W^\alpha_0, \quad W_{01} = g_{0\alpha} W^\alpha_1 $$
    $$ W_{10} = g_{1\alpha} W^\alpha_0, \quad W_{11} = g_{1\alpha} W^\alpha_1 $$
    
    We can construct a second-rank tensor called tensor product, $T = u \otimes v$.
    $$ (u \otimes v)(\sigma, \lambda) = \langle \sigma, u \rangle \langle \lambda, v \rangle $$
    The components of the tensor product are:
    $$ T^{\alpha \beta} = u^\alpha v^\beta $$
    or
    $$ (T^{\alpha \beta}) = \begin{pmatrix} ac & ad \\ bc & bd \end{pmatrix} $$
    where $(a,b)$, $(c,d)$ are components of $u,v$.
    
    For a higher-rank tensor, the gradient $\nabla$:
    $$ \nabla = i \frac{\partial}{\partial x} + j \frac{\partial}{\partial y} + k \frac{\partial}{\partial z} $$
    or
    $$ \partial_\nu = e_\nu \frac{\partial}{\partial x^\nu} + e_z \frac{\partial}{\partial x^2} + e_3 \frac{\partial}{\partial x^3} $$
    
    As an example, let $S$ be a $\binom{0}{3}$ rank tensor, then: $\nabla S(u,v,w, \xi)$ with $u,v,w$ fixed.
    $$ \nabla S(u,v,w, \xi) = \partial_\delta (S_{\alpha \beta \gamma} u^\alpha v^\beta w^\gamma) = \frac{\partial S_{\alpha \beta \gamma}}{\partial x^\delta} \xi^\delta u^\alpha v^\beta w^\gamma $$
    We can rewrite that:
    $$ \partial_\delta (S_{\alpha \beta \gamma}) = S_{\alpha \beta \gamma, \delta} \xi^\delta $$
    where $\xi^\delta$ define the direction of the gradient.
    
    We consider a 4-index tensor whose components are $R_{\alpha \beta \mu \nu}$. We can form a two-index tensor by inserting a basis 1-form into the first slot of the tensor definition, and the related basis vector in the third slot.
    $$ R(e_\alpha, u, w, v) = \lambda(u,v) $$
    or
    $$ M_{\mu \nu} u^\mu v^\nu = R_{\alpha \beta \mu \nu} u^\mu v^\nu \implies M_{\mu \nu} = R_{\alpha \beta \mu \nu} $$
    In three-dimensional Cartesian space, the divergence of $V$ is the quantity: $\nabla \cdot V = \frac{\partial V_x}{\partial x} + \frac{\partial V_y}{\partial y} + \frac{\partial V_z}{\partial z}$ while the 4-divergence is $\partial_\mu V^\mu$.
    $$ \Box_\mu = \nabla_\mu = w^\mu \frac{1}{\partial x^\mu} $$
    with $w^\mu$ the 1-form basic components. For example:
    $$ \frac{\partial J^\mu}{\partial x^\mu} = \Box \cdot J = \nabla \cdot J = \frac{\partial (\rho c)}{\partial c t} + \nabla \cdot j = \frac{\partial \rho}{\partial t} + \nabla \cdot j = 0 $$
    
    The d'Alembertian $\nabla^2 / \Box^2$ is
    $$ \Box^2 = \nabla \cdot \nabla = g^{\mu \nu} \frac{\partial}{\partial x^\mu} \frac{\partial}{\partial x^\nu} = \frac{1}{c^2} \frac{\partial^2}{\partial t^2} - (\frac{\partial^2}{\partial x^2} + \frac{\partial^2}{\partial y^2} + \frac{\partial^2}{\partial z^2}) $$
    For spacetime tensors, $S$:
    $$ \Box S(u,v) = \nabla \cdot S(u,v) = \nabla \cdot S(u,v, e_\alpha, u, e_\alpha) = S^\alpha_{\ \beta \gamma, \alpha} u^\beta v^\gamma $$
    In component form:
    $$ \nabla_\alpha S^\alpha_{\ \beta \gamma} = S^\alpha_{\ \beta \gamma, \alpha} $$
    
    We now define the wedge product / bivector / biform
    $$ u \wedge v = u \otimes v - v \otimes u $$
    In component form:
    $$ (u \wedge v)^{\alpha \beta} = u^\alpha v^\beta - v^\alpha u^\beta $$
    We consider $u = u^1 e_1 + u^2 e_2$, $v = v^1 e_1 + v^2 e_2$, we thus obtain $W = u \wedge v$ given by
    $$ W = \begin{pmatrix} u^1 v^1 - v^1 u^1 & u^1 v^2 - v^1 u^2 \\ u^2 v^1 - v^2 u^1 & u^2 v^2 - v^2 u^2 \end{pmatrix} = \begin{pmatrix} 0 & u^1 v^2 - v^1 u^2 \\ u^2 v^1 - v^2 u^1 & 0 \end{pmatrix} $$
    
    This produces Maxwell's equations:
    $$ c F^{\alpha \beta} = \begin{pmatrix} 0 & E_x & E_y & E_z \\ -E_x & 0 & cB_z & -cB_y \\ -E_y & -cB_z & 0 & cB_x \\ -E_z & cB_y & -cB_x & 0 \end{pmatrix} $$
    In Minkowski space, by the metric tensor
    $$ c F_{\alpha \beta} = \begin{pmatrix} 0 & -E_x & -E_y & -E_z \\ E_x & 0 & cB_z & -cB_y \\ E_y & -cB_z & 0 & cB_x \\ E_z & cB_y & -cB_x & 0 \end{pmatrix} $$
    $$ c F^\alpha_{\ \beta} = \begin{pmatrix} 0 & E_x & E_y & E_z \\ E_x & 0 & -cB_z & cB_y \\ E_y & cB_z & 0 & -cB_x \\ E_z & -cB_y & cB_x & 0 \end{pmatrix} $$
    
    Using the tensor / wedge product
    $$ F = F_{\alpha \beta} dx^\alpha \otimes dx^\beta = \frac{1}{2} F_{\alpha \beta} dx^\alpha \wedge dx^\beta $$
    which shows the antisymmetry.
    
    We can write Maxwell's equation in their normal component form using geometric notation
    $$ \nabla F = 0 \quad \text{and} \quad \nabla \cdot F = J $$
    where $J$ is the 4-current density with $(P_c, \vec{j})$ and $\vec{j}$ the three-current density.
    These equations produce
    $$ \nabla \cdot \vec{B} = 0, \quad \frac{\partial \vec{B}}{\partial t} + \nabla \times \vec{E} = 0 $$
    $$ \nabla \cdot \vec{E} = \rho / \epsilon_0, \quad \frac{1}{c^2} \frac{\partial \vec{E}}{\partial t} - \nabla \times \vec{B} = -\mu_0 \vec{j} $$
    
    The proper generalization of Newton's second law,
    $$ \frac{dp^\mu}{d\tau} = K^\mu $$
    where $K^\mu$ is a 4-vector force known as the Minkowski force.
    
    Example: the electromagnetic force is given by
    $$ K_\mu = -q \left( \frac{\partial (u^\nu A_\nu)}{\partial x^\mu} - \frac{d A_\mu}{d\tau} \right) $$
    The ordinary force $F_i$ and the spatial component $K_i$ are related by $F_i = K_i \sqrt{1- \beta^2} = \frac{dp_i}{dt}$.
    
    \section*{7.7 Relativistic Kinematics of Collisions and Many-Particle Systems}
    
    We consider the center-of-momentum frame (C-O-M) where the components of the spatial momentum of the initial particles add to zero. We define E and $\vec{P}$:
    $$ E = \sum E_i, \quad \vec{P} = \sum \vec{p}_i $$
    From $p \cdot p = E^2/c^2 - P^2$
    $$ p \cdot p = \left( \sum m_i \gamma_i \right)^2 c^2 - \left( \sum m_i \gamma_i \vec{v}_i \right)^2 > 0 $$
    
    As an example, we consider $(m_1, p_1)$ in the x-direction which suffers a head-on collision with $m_2$ at rest in the experimenter's frame.
    The initial 4-momentum:
    $$ p^\mu = [(m_1 \gamma_1 + m_2)c, m_1 \gamma_1 v_1^1, 0, 0] $$
    The length squared of which is
    $$ p^\mu p_\mu = (m_1^2 + m_2^2 + 2m_1 \gamma_1 m_2)c^2 $$
    In the C-O-M system, the total momentum is
    $$ [ (m_1 \gamma_1' + m_2 \gamma_2')c, 0, 0, 0 ] $$
    The space part of it vanishes.
    $$ m_1 \gamma_1' \vec{v}_1' c + m_2 \gamma_2' \vec{v}_2' c = 0 $$
    where $\vec{v}_1', \vec{v}_2'$ are the velocities of $m_1, m_2$ in C-O-M.
    \section*{Relativistic Kinematics}
    
    \subsection*{Two-Particle Reactions}
    The boost $\vec{\beta}'$ needed to go from the laboratory to the C-O-M frame is $\vec{\beta_s}' = -\vec{\beta}'$.
    The total squared momentum in the C-O-M frame:
    $$ P^{\mu'} P_{\mu'} = m_1^2 c^2 \gamma'^2 - m_1^2 v'^2 \gamma'^2 = m_1^2 c^2 $$
    We thus can solve for the boost velocity $\beta'$. In the two real roots, only the case of $\beta' < 1$ is physically meaningful.
    
    We consider a reaction initiated by two particles that produces another set of particles with $m_r$, $r=3,4,5...$ In the C-O-M system, the transformed total momentum is:
    $$ P^{\mu'} = (E'/c, 0, 0, 0) \quad \text{with} \quad M = E'/c^2 $$
    The square of $P^{\mu}$ is invariant in all Lorentz systems and conserved in the reaction.
    $$ P_\mu P^\mu = P'_{\mu'} P'^{\mu'} = \frac{E'^2}{c^2} = M^2 c^2 $$
    For the initial particles $P_\mu P^\mu = (P_{1\mu} + P_{2\mu})(P_1^\mu + P_2^\mu)$.
    The energy in the C-O-M system is given by:
    $$ E'^2 = M^2 c^4 = (m_1^2 + m_2^2)c^4 + 2(E_1 E_2 - c^2 \vec{P_1} \cdot \vec{P_2}) $$
    We suppose the particle 2 was initially stationary in the laboratory system: $\vec{P_2}=0$ and $E_2 = m_2 c^2$.
    $$ E'^2 = M^2 c^4 = (m_1^2 + m_2^2)c^4 + 2 m_2 c^2 E_1 $$
    
    \subsection*{ Threshold Energy}
    We denote $P^{\mu'}$, which is the total four-momentum in the C-O-M system.
    $$ P'_{\mu'} P'^{\mu'} = c^2 \left(\sum m_r\right)^2 $$
    which, by conservation of momentum, must be the same.
    For a stationary target, the incident kinetic energy at threshold is:
    $$ T_1 = \frac{(\sum m_r)^2 - (m_1+m_2)^2}{2m_2} c^2 $$
    We define the Q-value of the reaction as $Q = \left[\sum m_r - (m_1+m_2)\right]c^2$.
    Thus:
    $$ T_1 = \frac{Q^2 + 2Q(m_1+m_2)c^2}{2m_2c^2} $$
    By the historic production of an antiproton $\bar{P}$ by the reaction involving a proton $P$.
    $$ p+n \rightarrow p+n+p+\bar{p} $$
    where $n$ is a nucleon. We select $Q=2m_p c^2$ and thus $T_1 = 6m_p c^2 \approx 5.6 \text{ GeV}$.
    
    \subsection*{ Photomeson Production}
    We consider photomeson production:
    $$ \gamma + p \rightarrow \Sigma^0 + K^+ $$
    where $\gamma$ is an incoming photon (a zero-mass particle).
    $$ Q = (m_{\Sigma^0} + m_{K^+} - m_p)c^2 = 749 \text{ MeV} $$
    and the threshold energy for the photon is
    $$ T_\gamma = E_\gamma = \frac{Q^2 + 2Q m_p c^2}{2 m_p c^2} = 1.05 \text{ GeV} $$
    
    \subsection*{Center-of-Momentum System}
    The C-O-M system is the rest system for the total momentum M with $P^{0'} = Mc$. In any other system, the zeroth component of the 4-vector is $P^0 = M c \gamma$.
    In the laboratory system,
    $$ P^0 = \frac{1}{c}(E_1+E_2) = \frac{1}{c}(E_1+m_2c^2) $$
    where the last term holds only for a stationary target particle. Hence, the C-O-M system moves relative to the laboratory system such that
    $$ \gamma = \frac{E_1+m_2c^2}{Mc^2} $$
    At threshold all the reaction products are at rest in the C-O-M system so that $M = \sum m_r$, therefore
    $$ \gamma = \frac{T_1 + (m_1+m_2)c^2}{(\sum m_r) c^2} \quad (\text{threshold}) $$
    The kinetic energy of the $s$-th reaction product in the laboratory system is
    $$ T_s = m_s c^2 (\gamma_s - 1) $$
    The Lorentz transformation from the Laboratory to the C-O-M system is defined by $\gamma$:
    $$ \gamma = \frac{E_1+m_2 c^2}{\sqrt{2m_2 c^2 E_1 + (m_1^2+m_2^2)c^4}} = \frac{T_1+(m_1+m_2)c^2}{\sqrt{2m_2c^2 T_1 + (m_1+m_2)^2 c^4}} $$
    In the C-O-M system, the spatial part of the total momentum four-vector is zero. In any other system, the spatial part is $M c \beta \gamma$.
    
    \begin{figure}[h]
    	\centering
    	\includegraphics[width=1.0\linewidth]{figure5}
    	\caption{}
    	\label{fig:figure5}
    \end{figure}
    
    \section*{Scattering}
    In the laboratory system, the spatial part is $\vec{P}_1$.
    $$ \vec{\beta} = \frac{\vec{P}_1 c}{E_1+m_2c^2} = \frac{\vec{P}_1 c}{T_1+(m_1+m_2)c^2} $$
    which, along the z-axis, gives $\beta_x=\beta_y=0$.
    The components of $P_1^{\mu'}$ in the C-O-M system are:
    \begin{align*}
    	p'_{1z} &= \gamma (p_{1z} - \beta E_1/c) \\
    	\frac{E_1'}{c} = p_1^{0'} &= \gamma\left(\frac{E_1}{c} - \beta p_{1z}\right)
    \end{align*}
    After the collision, the scattered particle's momentum $\vec{p_3}'$ is no longer along the z-axis, but its magnitude is the same as that of $\vec{p_1}'$. We define $\Theta$ as the angle between $\vec{p_3}'$ and the incident direction, then:
    $$ p'_{3x} = p'_1 \sin\Theta, \quad p'_{3z} = p'_1 \cos\Theta, \quad p_3^{0'} = p_1^{0'} = E_1'/c $$
    The transformation back to the laboratory system is with relative velocity $-\vec{\beta}$.
    \begin{align*}
    	p_{3x} &= p'_{3x} = p'_1 \sin\Theta \\
    	p_{3z} &= \gamma(p'_{3z} + \beta p_3^{0'}) = \gamma(p'_1 \cos\Theta + \beta E_1'/c) \\
    	E_3/c = p_3^0 &= \gamma(p_3^{0'} + \beta p'_{3z}) = \gamma(E_1'/c + \beta p'_1 \cos\Theta)
    \end{align*}
    The energy of the scattered particle, $E_3$, can be related to the scattering angle. Let's consider the case of Compton Scattering where $m_1=0$. Let $P = E_{initial}/(m_2 c^2)$. Then the ratio of final to initial kinetic energy is:
    $$ \frac{T_3}{T_1} = \frac{2P(1+E_1/T_1)}{(1+P)^2 + 2PE_1} (1-\cos\Theta) $$
    where the variable definitions might be specific to the context of the problem.
    
    If $P=1$, the relativistic corrections cancel completely and $E_3$ has the minimum after scattering:
    $$ (E_3)_{\min} = E_1 \frac{(1-P)^2}{(1+P)^2+2PE_1} $$
    In the non-relativistic limit, the minimum fractional energy after scattering is
    $$ \frac{(E_3)_{\min}}{E_1} = \left(\frac{1-P}{1+P}\right)^2, \quad E_1 \ll 1 $$
    In the ultra-relativistic region, when $PE_1 \gg 1$, the minimum kinetic energy after scattering is
    $$ (T_3)_{\min} = \frac{(m_2-m_1)^2 c^2}{2m_2}, \quad PE_1 \gg 1 $$
    We obtain the relation between the scattering angles in the C-O-M ($\Theta$) and laboratory ($\vartheta$) systems by noting that:
    $$ \tan\vartheta = \frac{p_{3x}}{p_{3z}} = \frac{p'_1 \sin\Theta}{\gamma(p'_1 \cos\Theta + \beta E_1'/c)} = \frac{\sin\Theta}{\gamma(\cos\Theta + \beta/\beta'_1)} $$
    where $\beta'_1 = p'_1 c / E'_1$ is the velocity of particle 1 in the C-O-M frame.
    The term $\beta/\beta'_1$ can be expressed as:
    $$ \frac{\beta}{\beta'_1} = \frac{\beta p_1 c}{E_1 - \beta p_1 c} $$
    By the relations:
    $$ E_1 - \beta p_1 c = \frac{m_1(m_1+m_2)c^2 + m_1c^2 T_1}{(m_1+m_2)c^2 + T_1} $$
    we have
    $$ \tan\vartheta = \frac{\sin\Theta}{\gamma\left[\cos\Theta + f(P,E_1)\right]} $$
    where $f(P,E_1)$ represents the ratio of velocities $\beta/\beta'_1$.
    
    \section*{7.8 Relativistic Angular Momentum}
    
    For a single particle, we define an antisymmetric tensor of rank $\binom{2}{0}$ in Minkowski space:
    $$ m = x \wedge p $$
    whose elements would be
    $$ m^{\mu\nu} = x^\mu p^\nu - x^\nu p^\mu $$
    The derivative with respect to the proper time
    $$ \frac{dm}{d\tau} = u \wedge p + x \wedge K, = x \wedge K $$
    where $u \wedge p$ vanishes by the antisymmetry of the wedge product and $K$ is the Minkowski force.
    In component notation:
    $$ \frac{dm^{\mu\nu}}{d\tau} = x^\mu K^\nu - x^\nu K^\mu $$
    We define the relativistic generalization of the torque
    $$ N = x \wedge K $$
    whose components are
    $$ N^{\mu\nu} = x^\mu K^\nu - x^\nu K^\mu $$
    Thus:
    $$ \frac{dm}{d\tau} = N $$
    whose components are
    $$ \frac{dm^{\mu\nu}}{d\tau} = N^{\mu\nu} $$
    
    For a system, a total angular momentum 4-tensor can be defined as:
    $$ M = \sum m_s \quad \text{or} \quad M^{\mu\nu} = \sum_s m_s^{\mu\nu} $$
    With respect to some other reference event $\vec{a}$, the total momentum is
    $$ M(a) = \sum_s (x_s - a) \wedge p_s = M(0) - a \wedge p $$
    By definition in some particular Lorentz frame, the components of $M^{\mu\nu}$
    $$ M^{ij} = \sum (x_s^i p_s^j - x_s^j p_s^i) = c \sum (t_s^i p_s^j - x_s^j \frac{E_s}{c^2}) $$
    and
    $$ M^{0j} = - M^{j0} $$
    In the C-O-M frame, the total linear momentum $P = \sum p_s$ vanishes, and:
    $$ M^{0j} = -c \sum x_s^j \frac{E_s}{c^2} $$
    If the system is such that total angular momentum is conserved along with other components
    $$ \sum x_s^j E_s = \text{constant} $$
    Conservation of total linear momentum means that $E = \sum E_s$ is also conserved.
    We thus define a spatial point $R_j$
    $$ R_j = \frac{\sum x_s^j E_s}{\sum E_s} $$
    
    \section*{7.9 The Lagrangian Formulation of Relativistic Mechanics}
    
    We consider a function $L$ to obtain the Euler-Lagrange equations
    $$ \delta I = \delta \int_{t_1}^{t_2} L dt = 0 $$
    The relativistic Lagrangian:
    $$ L = -m c^2 \sqrt{1 - \beta^2} - V $$
    where $\beta=v/c$ with $v$ the speed of the particle in the Lorentz frame.
    From
    $$ \frac{d}{dt} \left( \frac{\partial L}{\partial v^i} \right) - \frac{\partial L}{\partial x^i} = 0 $$
    where
    $$ \frac{\partial L}{\partial v^i} = \frac{m v^i}{\sqrt{1-\beta^2}} = p^i $$
    Then:
    $$ \frac{d}{dt} \frac{m v^i}{\sqrt{1-\beta^2}} = -\frac{\partial V}{\partial x^i} = F^i $$
    The canonical momenta $P$ is defined by
    $$ P_i = \frac{\partial L}{\partial \dot{q}^i} $$
    If $L$ does not contain time explicitly, we obtain
    $$ h = \dot{q}^i P_i - L $$
    $$ = \frac{m v_i v^i}{\sqrt{1-\beta^2}} + mc^2 \sqrt{1-\beta^2} + V $$
    is the total energy
    $$ h = T + V + mc^2 = E' $$
    
    We consider a single particle of charge $q$ in an electromagnetic field is
    $$ L = -mc^2 \sqrt{1-\beta^2} - q\phi + q\vec{A} \cdot \vec{v} $$
    Note that $P \neq mu$.
    We can write
    $$ p^i = mu^i + qA^i $$
    
    \subsection*{1) Motion under a constant force; hyperbolic motion}
    We take $x$ axis as the direction of the constant force.
    $$ L = -mc^2 \sqrt{1-\beta^2} - max $$
    where $\beta = \dot{x}/c$ and $a$ is the constant magnitude of the force per unit mass.
    The equation of motion:
    $$ \frac{d}{dt} \left( \frac{B}{\sqrt{1-\beta^2}} \right) = \frac{a}{c} $$
    which leads to
    $$ \frac{B}{\sqrt{c^2 + (at+\alpha)^2}} = \frac{at+\alpha}{\sqrt{c^2 + (at+\alpha)^2}} \quad \text{or} \quad \frac{B}{\sqrt{1-\beta^2}} = \frac{at+\alpha}{c} $$
    where $\alpha$ is a constant of integration.
    $$ x - x_0 = c \int_0^t \frac{(at'+\alpha)dt'}{\sqrt{c^2 + (at'+\alpha)^2}} $$
    $$ = \frac{c}{a} [ \sqrt{c^2+(at+\alpha)^2} - \sqrt{c^2+\alpha^2} ] $$
    If the particle starts at rest from $x_0=0$ and $u=0, \alpha=0$, then
    $$ (x+\frac{c^2}{a})^2 - c^2t^2 = \frac{c^4}{a^2} $$
    which is the hyperbola in the $x,t$ plane.
    
    \subsection*{2) The relativistic one-dimensional harmonic oscillator}
    With $V(x) = \frac{k}{2}x^2$.
    Since $L$ is not explicitly a function of $t$ and $V$ is independent of velocity, $E$ is constant.
    We obtain $\frac{1}{c^2}(\frac{dx}{dt})^2 = 1 - \frac{m^2c^4}{(E-V)^2}$.
    Then the period of the oscillator motion
    $$ T = \frac{4}{c} \int_0^b \frac{dx}{\sqrt{1 - \frac{m^2c^4}{(E-V)^2}}} $$
    We now rewrite the energy as
    $$ E = mc^2(1+E) $$
    so that
    $$ \frac{E-V}{mc^2} = 1+E-kx+\frac{k}{mc^2}x^2 $$
    where $k=\frac{K}{2mc^2}$.
    Thus:
    $$ T \approx \frac{4}{c} \int_0^b \frac{dx}{\sqrt{2k(b^2-x^2)}} [1 - \frac{3}{4}c^2(b^2-x^2)] $$
    with $x = b \sin\phi$.
    $$ T \approx \frac{2\pi}{c\sqrt{2k}} (1-\frac{3}{8}kb^2) = 2\pi\sqrt{\frac{m}{K}} (1-\frac{3}{16}\frac{kb^2}{mc^2}) $$
    Then:
    $$ \frac{\Delta\nu}{\nu_0} = -\frac{\Delta\tau}{T_0} \approx \frac{3}{16}\frac{kb^2}{mc^2} = \frac{3}{8} \frac{E}{mc^2} $$
    
    \subsection*{3) Motion of a charged particle in a constant magnetic field}
    With the scalar potential $\phi=0$ and $\vec{A}$ appropriate to a constant magnetic field, the Lorentz force
    $$ \vec{F} = q(\vec{v} \times \vec{B}) $$
    Hence, the equation of motion
    $$ \frac{d\vec{p}}{dt} = q(\vec{v} \times \vec{B}) = \frac{q}{m\gamma}(\vec{p} \times \vec{B}) $$
    Clearly, $\vec{F} \cdot \vec{v} = 0$, so E must be a constant.
    The vector $\vec{p}$ is precessing around the direction of the magnetic field with a frequency
    $$ \Omega = \frac{qB}{m\gamma} $$
    The magnitude of the linear momentum in the plane
    $$ p = m\gamma r \Omega $$
    The particle rotates with the radius
    $$ r = \frac{p}{qB} $$
    which depends on $p/q (=Br)$, which is called magnetic rigidity.
    \section*{7.10 Covariant Lagrangian Formulations}
    
    \subsection*{(1)}
    We consider a system of only one particle. We choose the proper time $\tau$ as the invariant parameter. But the various components of the generalized velocity $u^\nu$ obey the relation:
    $$ u \cdot u = u^\nu u_\nu = c^2 $$
    which shows they are not independent.
    
    Thus, we assume the choice of some Lorentz-invariant quantity $\theta$ with no further specification than that it be a monotonic function of the progress of the world point along the particle's world line. Then:
    $$ x'^\nu = \frac{dx^\nu}{d\theta} $$
    A suitably covariant Hamilton's principle appears as:
    $$ \delta I = \delta \int_{\theta_1}^{\theta_2} \Lambda(x^\mu, x'^\mu) \, d\theta $$
    where $\Lambda$ must be a world scalar and $(x^\mu, x'^\mu)$ means a function of all or any of these.
    
    The Euler-Lagrange equations corresponding are:
    $$ \frac{d}{d\theta} \left(\frac{\partial \Lambda}{\partial x'^\mu}\right) - \frac{\partial \Lambda}{\partial x^\mu} = 0 $$
    
    \subsection*{(2)}
    One way of seeking $\Lambda$ is to transform the action integral from the usual integral over $t$ to one over $\theta$ and to treat the time $t$ appearing explicitly in the Lagrangian as an additional generalized coordinate.
    
    Since $\theta$ must be a function of $t$ in Lorentz frame (monotonic), we have:
    $$ \frac{dx^0}{dt} = \frac{dx^0}{d\theta} \frac{d\theta}{dt} \quad \implies \quad c = x'^0 \frac{d\theta}{dt} $$
    Hence:
    $$ I = \int_{t_1}^{t_2} L(x^j, \dot{x}^j, t) \, dt = \int_{\theta_1}^{\theta_2} L\left(x^j, c\frac{x'^j}{x'^0}\right) \frac{x'^0}{c} \, d\theta $$
    A suitable $\Lambda$ is given by:
    $$ \Lambda(x^\mu, x'^\mu) = \frac{x'^0}{c} L\left(x^\mu, c\frac{x'^\mu}{x'^0}\right) $$
    which is a homogeneous function of the generalized velocities $x'^\mu$ in the first degree.
    $$ \Lambda(x^\mu, a x'^\mu) = a \Lambda(x^\mu, x'^\mu) $$
    By Euler's theorem on homogeneous functions, if $\Lambda$ is homogeneous to first degree in $x'^\mu$:
    $$ \Lambda = x'^\mu \frac{\partial \Lambda}{\partial x'^\mu} $$
    And $\Lambda$ identically satisfies the relation:
    $$ \left[ \frac{d}{d\theta}\left(\frac{\partial \Lambda}{\partial x'^\mu}\right) - \frac{\partial \Lambda}{\partial x^\mu} \right] x'^\mu = 0 $$
    
    The Lagrangian for the free particle is:
    $$ L = -mc^2 \sqrt{1 - \frac{\dot{\vec{x}} \cdot \dot{\vec{x}}}{c^2}} $$
    By the transformation, a possible covariant Lagrangian is then:
    $$ \Lambda = -mc\sqrt{x'_\mu x'^\mu} $$
    With the Euler-Lagrange equations:
    $$ \frac{d}{d\theta} \left( \frac{mc x'_\mu}{\sqrt{x'_\nu x'^\nu}} \right) = 0 $$
    where $\theta$ must be a monotonic function of the proper time $\tau$ so that derivatives with respect to $\theta$ are in terms of $\tau$.
    $$ x'_\mu = \frac{dx_\mu}{d\theta} = \frac{d\tau}{d\theta} u_\mu $$
    The Lagrangian equations are:
    $$ \frac{d}{d\tau} \left( \frac{m c u_\mu}{\sqrt{u_\nu u^\nu}} \right) = \frac{d(mu_\mu)}{d\tau} = 0 $$
    
    In terms of $\tau$, the covariant Lagrange equations are:
    $$ \frac{d}{d\tau} \left(\frac{\partial \Lambda}{\partial u^\nu}\right) - \frac{\partial \Lambda}{\partial x^\nu} = 0 $$
    All that is required is that $\Lambda$ be a world scalar that leads to the correct equations of motion. For example, for a free particle, $\Lambda$ would be:
    $$ \Lambda = \frac{1}{2} m u_\nu u^\nu $$
    
    \subsection*{(3)}
    A Lagrangian for a particle in an electromagnetic field is:
    $$ \Lambda(x^\mu, u^\mu) = \frac{1}{2} m u_\mu u^\mu + \frac{q}{c} u_\mu A^\mu(x^\lambda) $$
    The corresponding Lagrange equations yield the equation of motion:
    $$ \frac{d}{d\tau}(m u^\nu) = -\frac{q}{c}\frac{dA^\nu}{d\tau} + \frac{\partial}{\partial x_\nu} \left(\frac{q}{c} u^\mu A_\mu\right) $$
    
    Note that the mechanical momentum four-vector $P^\mu$ differs from the canonical momentum $p^\mu$.
    $$ p^\mu = g^{\mu\nu} \frac{\partial \Lambda}{\partial u^\nu} = m u^\mu + \frac{q}{c} A^\mu = P^\mu + \frac{q}{c} A^\mu $$
    The canonical momentum $p^0$ conjugate to $x^0$ is:
    $$ p^0 = \frac{E}{c} + \frac{q}{c}\phi = \frac{1}{c} \mathcal{E} $$
    where $E$ is the mechanical energy and $\mathcal{E}$ is the total energy of the particle $E+q\phi$.
    
    The canonical momenta conjugate to $\vec{x}$ form the components of a spatial Cartesian vector $\vec{p}$ related to $\vec{P}$ by:
    $$ \vec{p} = \vec{P} + \frac{q}{c}\vec{A} $$
    and the energy relation is:
    $$ \mathcal{E}^2 = c^2\left(\vec{p} - \frac{q}{c}\vec{A}\right)^2 + m^2c^4 $$
    
\end{document}