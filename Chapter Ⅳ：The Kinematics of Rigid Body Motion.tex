\documentclass{article}
\usepackage{amsmath}
\usepackage{amssymb}
\usepackage{amsfonts}
\usepackage{graphicx}
\usepackage{geometry}
\geometry{a4paper, margin=1in}
\begin{document}
	
	\section*{Chapter VI: The Kinematics of Rigid Body Motion}
	
	\subsection*{4.1 The Independent Coordinates of a Rigid Body}
	\begin{enumerate}
		\item For a rigid body with $N$ particles, the degrees of freedom are reduced by the constraints $r_{ij} = c_{ij}$.
		
		To fix a rigid body, we just need to specify any three non-collinear points, thus the degrees of freedom can't be more than 9, and with $r_{12} = c_{12}$, $r_{23} = c_{23}$, $r_{13} = c_{13}$, they're reduced to 6. 
		
        \begin{figure}[h]
        	\centering
         	\includegraphics[width=0.7\linewidth]{figure1}
	        \caption{}
        	\label{fig:figure1}
        \end{figure}
        
		\item For two Cartesian coordinates, the direction cosines are defined by
		\[
		\begin{cases}
			\cos\theta_{11} = \cos(\mathbf{i'}, \mathbf{i}) = \mathbf{i'} \cdot \mathbf{i} \\
			\cos\theta_{12} = \cos(\mathbf{i'}, \mathbf{j}) = \mathbf{i'} \cdot \mathbf{j} \\
			\cos\theta_{21} = \cos(\mathbf{j'}, \mathbf{i}) = \mathbf{j'} \cdot \mathbf{i}, \text{etc.}
		\end{cases}
		\]
		In terms of the unit vectors of the unprimed system, the unit vector in the primed system can be expressed
		\begin{align*}
			\mathbf{i'} &= \cos\theta_{11}\mathbf{i} + \cos\theta_{12}\mathbf{j} + \cos\theta_{13}\mathbf{k} \\
			\mathbf{j'} &= \cos\theta_{21}\mathbf{i} + \cos\theta_{22}\mathbf{j} + \cos\theta_{23}\mathbf{k} \\
			\mathbf{k'} &= \cos\theta_{31}\mathbf{i} + \cos\theta_{32}\mathbf{j} + \cos\theta_{33}\mathbf{k}
		\end{align*}
		Thus, we can write
		\[
		\mathbf{r} = x\mathbf{i} + y\mathbf{j} + z\mathbf{k} = x'\mathbf{i'} + y'\mathbf{j'} + z'\mathbf{k'}
		\]
		where $x' = (\mathbf{i'} \cdot \mathbf{r})$, $y' = (\mathbf{j'} \cdot \mathbf{r})$, $z' = (\mathbf{k'} \cdot \mathbf{r})$. Note that $\mathbf{i'} \cdot \mathbf{j'} = \mathbf{j'} \cdot \mathbf{k'} = \mathbf{k'} \cdot \mathbf{i'} = 0$ and $\mathbf{i'} \cdot \mathbf{i'} = \mathbf{j'} \cdot \mathbf{j'} = \mathbf{k'} \cdot \mathbf{k'} = 1$.
	
		\begin{figure}[h]
			\centering
			\includegraphics[width=0.7\linewidth]{figure2}
			\caption{}
			\label{fig:figure2}
		\end{figure}
		
		We can use the Kronecker symbol
		\[
		\sum_{l=1}^{3} \cos\theta_{li'} \cos\theta_{lj'} = \delta_{ij}
		\]
	\end{enumerate}
	
	\subsection*{4.2 Orthogonal Transformations}
	\begin{enumerate}
		\item As showed before, for $x \to x_1, y \to x_2, z \to x_3$, we write $a_{ij} = \cos\theta_{ij}$, thus
		\begin{align*}
			x'_1 &= a_{11}x_1 + a_{12}x_2 + a_{13}x_3 \\
			x'_2 &= a_{21}x_1 + a_{22}x_2 + a_{23}x_3 \\
			x'_3 &= a_{31}x_1 + a_{32}x_2 + a_{33}x_3
		\end{align*}
		We use the summation convention introduced by Einstein
		\[
		x'_i = \sum_{j} a_{ij}x_j, \quad i=1,2,3
		\]
		and $\sum_i x'_i x'_i$ can be written as $x'_i x'_i$.
		
		Thus,
		\[
		x'_i x'_i = x_j x_j = a_{ij} a_{ik} x_j x_k
		\]
		\[
		\Rightarrow a_{ij} a_{ik} = \delta_{jk}, \quad j,k=1,2,3
		\]
		which is the orthogonal condition.
		
		The matrix of transformation is
		\[
		A = 
		\begin{pmatrix}
			a_{11} & a_{12} & a_{13} \\
			a_{21} & a_{22} & a_{23} \\
			a_{31} & a_{32} & a_{33}
		\end{pmatrix}
		\]
		such that $\mathbf{x'} = A\mathbf{x}$.
	\end{enumerate}
	
	\subsection*{4.3 Formal Properties of the Transformation Matrix}
	\begin{enumerate}
		\item We consider two successive transformations
		\[
		\mathbf{x'} = B\mathbf{x} \quad \text{and} \quad \mathbf{x''} = A\mathbf{x'}
		\]
		or
		\[
		x'_k = b_{kj}x_j \quad \text{and} \quad x''_i = a_{ik}x'_k
		\]
		Thus, we have
		\[
		x''_i = a_{ik}b_{kj}x_j \Rightarrow x''_i = c_{ij}x_j
		\]
		where $C=AB$ where $c_{ij} = a_{ik}b_{kj}$.
		
		or $C=BA$ is defined by $d_{ij} = b_{ik}a_{kj}$.
		
		\item
		\[
		x = 
		\begin{bmatrix}
			x_1 \\
			x_2 \\
			x_3
		\end{bmatrix}, \quad
		x' = 
		\begin{bmatrix}
			x'_1 \\
			x'_2 \\
			x'_3
		\end{bmatrix}
		\]
		\[
		(Ax)_i = a_{ij}x_j = x'_i \Rightarrow x' = Ax
		\]
		
		\item $(AB)C = A(BC)$
		
		\item $C = A+B \Rightarrow c_{ij} = a_{ij} + b_{ij}$
		
		\item For $A^{-1}$ where $x_i = a_{ij}x'_j$
		\[
		\Rightarrow a_{ki}a_{ij} = \delta_{kj} \quad \text{and} \quad AA^{-1} = 
		\begin{bmatrix}
			1 & & \\
			& \ddots & \\
			& & 1
		\end{bmatrix}
		\]
		where $x=Ix$.
		
		\item Let's consider $a_{kl}a_{ki}a_{ij}$ or $c_{lj} a_{ij}$ with $c_{li} = a_{kl}a_{ki}$ or $a_{ki}d_{kj}$ with $d_{kj}=a_{ki}a_{ij}$.
		
		$a_{kl}\delta_{kj} = a_{lj}$
		
		\item $A \to A^{-1}: a'_{ij} = a_{ji}$. For orthogonal matrices, $A^{-1} = \tilde{A}$ where $\tilde{A}$ is the transposed matrix, and thus we have
		\[
		a_{ki}a_{ji} = \delta_{kj} \Rightarrow A\tilde{A}=I
		\]
		
		\item $A_{ij}x_j = x_j(A)_{ji} \Rightarrow Ax=x\tilde{A}$
		
		\item If $A_{ij} = A_{ji}$, the matrix is symmetric.
		
		If $A_{ij} = -A_{ji}$, the matrix is antisymmetric.
		
		\item $|AB|=|A||B|$
		
		$A\tilde{A}=I \Rightarrow |A||\tilde{A}| = |A|^2 = 1$
		
		$|A^{-1}||A|=1$
	\end{enumerate}
	
	\subsection*{4.4 The Euler Angles}
	We consider a transformation
	\[
	xyz \xrightarrow[\text{about } z]{\text{rotate by } \phi \text{ counterclockwise}} \xi\eta\zeta \xrightarrow[\text{about } \xi]{\text{rotate by } \theta \text{ counterclockwise}} \xi'\eta'\zeta'
	\]
	\[
	\xrightarrow[\text{about } \zeta']{\text{rotate by } \psi \text{ counterclockwise}} x'y'z'
	\]
	
	\begin{figure}[h]
		\centering
		\includegraphics[width=0.7\linewidth]{figure3}
		\caption{}
		\label{fig:figure3}
	\end{figure}
	
	
	We have
	\[
	\begin{cases}
		\text{\textcircled{1}} \quad \xi = Dx \\
		\text{\textcircled{2}} \quad \xi' = C\xi \\
		\text{\textcircled{3}} \quad x' = B\xi'
	\end{cases}
	\quad \text{where } \xi \text{ and } x \text{ are column matrices}
	\]
	Hence, $A=BCD$ where
	\[
	B = 
	\begin{bmatrix}
		\cos\psi & \sin\psi & 0 \\
		-\sin\psi & \cos\psi & 0 \\
		0 & 0 & 1
	\end{bmatrix}
	\]
	\[
	C = 
	\begin{bmatrix}
		1 & 0 & 0 \\
		0 & \cos\theta & \sin\theta \\
		0 & -\sin\theta & \cos\theta
	\end{bmatrix}
	\]
	\[
	D = 
	\begin{bmatrix}
		\cos\phi & \sin\phi & 0 \\
		-\sin\phi & \cos\phi & 0 \\
		0 & 0 & 1
	\end{bmatrix}
	\]
	
	\subsection*{4.5 The Cayley-Klein Parameters and Related Quantities}
	We consider four parameters $\alpha, \beta, \gamma, \delta$ with the constraints that $\beta = -\gamma^*$ and $\delta = \alpha^*$.
	
	The transformation matrix
	\[
	A = 
	\begin{bmatrix}
		\frac{1}{2}(\alpha^2 - \gamma^2 + \delta^2 - \beta^2) & \frac{i}{2}(\gamma^2 - \alpha^2 + \delta^2 - \beta^2) & \gamma\delta - \alpha\beta \\
		\frac{1}{2}(\alpha^2 + \gamma^2 - \beta^2 - \delta^2) & \frac{1}{2}(\alpha^2 + \gamma^2 + \beta^2 + \delta^2) & -i(\alpha\beta + \gamma\delta) \\
		\beta\delta - \alpha\gamma & i(\alpha\gamma + \beta\delta) & \alpha\delta + \beta\gamma
	\end{bmatrix}
	\]
	And we choose the Euler parameters $e_0, e_1, e_2, e_3$ where $\sum_{i=0}^{3} e_i^2 = 1$ and
	\begin{align*}
		\alpha &= e_0 + ie_3 \\
		\beta &= e_2 + ie_1
	\end{align*}
	
	\subsection*{4.6 Euler's Theorem on the Motion of a Rigid Body}
	\begin{enumerate}
		\item \textbf{Euler's Theorem:} The general displacement of a rigid body with one point fixed is a rotation about some axis.
		\[
		\Rightarrow R' = AR = R \lambda
		\]
		where $\lambda$ is a complex constant where we have the eigenvalue equations
		\[
		(A-\lambda I)R=0 \Rightarrow |A-\lambda I| = 
		\begin{vmatrix}
			a_{11}-\lambda & a_{12} & a_{13} \\
			a_{21} & a_{22}-\lambda & a_{23} \\
			a_{31} & a_{32} & a_{33}-\lambda
		\end{vmatrix} = 0
		\]
		which is the characteristic equation.
		
		The eigenvalue equation can also be written:
		\[
		\sum_j a_{ij}x_{jk} = \lambda_k x_{ik} = \sum_j x_{ij} \delta_{jk} \lambda_k
		\]
	\end{enumerate}
	\subsection*{4.6 Euler's Theorem (Continued)}
	We designate the matrix by
	\[
	\Lambda = 
	\begin{bmatrix}
		\lambda_1 & 0 & 0 \\
		0 & \lambda_2 & 0 \\
		0 & 0 & \lambda_3
	\end{bmatrix}
	\]
	Hence: $AX = X\Lambda \Rightarrow X^{-1}AX = \Lambda$.
	\begin{enumerate}
		\setcounter{enumi}{1}
		\item Consider the expression $(A-I)\tilde{A} = I - \tilde{A}$ (orthogonal matrices).
		\[
		\Rightarrow |A-I||\tilde{A}| = |I-\tilde{A}| = |I-A|
		\]
		Since $|-B| = (-1)^n |B|$, we have $|I-A| = 0$.
		
		\item \textbf{Chasles' Theorem:} The most general displacement of a rigid body is a translation plus a rotation.
	\end{enumerate}
	
	\subsection*{4.7 Finite Rotations}
	\begin{enumerate}
		\item For $\vec{r} = \vec{OP}$, $\vec{r'} = \vec{OQ}$, $\vec{ON} = \vec{n}(\vec{n} \cdot \vec{r})$, $\vec{NP} = \vec{r} - \vec{n}(\vec{n} \cdot \vec{r})$.
		\[
		\vec{NQ} = \vec{r} \times \vec{n}, \quad \vec{r'} = \vec{ON} + \vec{NV} + \vec{VQ}
		\]
		\[
		\Rightarrow \vec{r'} = \vec{r}\cos\Phi + \vec{n}(\vec{n} \cdot \vec{r})(1-\cos\Phi) + (\vec{r} \times \vec{n})\sin\Phi
		\]
		which is the rotation formula.
		
		\begin{figure}[h]
			\centering
			\includegraphics[width=0.7\linewidth]{figure4}
			\caption{}
			\label{fig:figure4}
		\end{figure}
		
	\end{enumerate}
	
	\subsection*{4.8 Infinitesimal Rotations}
	\begin{enumerate}
		\item From $\vec{x}$ to $\vec{x'}$, we consider
		\[
		x'_i = x_i + \epsilon_{i1}x_1 + \epsilon_{i2}x_2 + \epsilon_{i3}x_3 \Rightarrow x'_i = (\delta_{ij} + \epsilon_{ij})x_j
		\]
		where $\epsilon_{11}, \epsilon_{12}$ are infinitesimals.
		\[
		\Rightarrow x' = (I+\epsilon)x
		\]
		In fact, $(I+\epsilon_1)(I+\epsilon_2) = I+\epsilon_1+\epsilon_2$ where $\epsilon_1\epsilon_2$ is neglected.
		
		$A = I+\epsilon$ and $A^{-1} = I-\epsilon$ since $AA^{-1}=I$.
		
		Further, as for orthogonality, $\tilde{A} = I + \tilde{\epsilon} = A^{-1} = I-\epsilon \Rightarrow \tilde{\epsilon} = -\epsilon$.
		
		\item For Euler infinitesimal rotation:
		\[
		A = 
		\begin{bmatrix}
			1 & d\phi & 0 \\
			-d\phi & 1 & d\theta \\
			0 & -d\theta & 1
		\end{bmatrix}
		\text{ and }
		d\Omega = \hat{i}d\theta + \hat{k}(d\phi+d\psi)
		\]
		\[
		\epsilon =
		\begin{bmatrix}
			0 & d\Omega_3 & -d\Omega_2 \\
			-d\Omega_3 & 0 & d\Omega_1 \\
			d\Omega_2 & -d\Omega_1 & 0
		\end{bmatrix}
		\]
		we write the form: $\vec{r'} - \vec{r} = d\vec{r} = \epsilon\vec{r}$
		\[
		\Rightarrow
		\begin{cases}
			dx_1 = x_2 d\Omega_3 - x_3 d\Omega_2 \\
			dx_2 = x_3 d\Omega_1 - x_1 d\Omega_3 \\
			dx_3 = x_1 d\Omega_2 - x_2 d\Omega_1
		\end{cases}
		\implies d\vec{r} = \vec{r} \times d\vec{\Omega}
		\]
		
		\item In another fashion: $\vec{r'} - \vec{r} = d\vec{r} = \vec{r} \times \vec{n}d\Phi$ where $d\vec{\Omega} = \vec{n}d\Phi$.
		
		The magnitude is: $dr = r\sin\theta d\Phi$.
		
		\begin{figure}[h]
			\centering
			\includegraphics[width=0.7\linewidth]{figure5}
			\caption{}
			\label{fig:figure5}
		\end{figure}
		
		
	\end{enumerate}
	
	\subsection*{4.9 Rate of Change of a Vector}
	\begin{enumerate}
		\item For a general vector $\vec{G}$, we have
		\[
		(d\vec{G})_{\text{space}} = (d\vec{G})_{\text{body}} + (d\vec{G})_{\text{rot}}
		\]
		where $(d\vec{G})_{\text{rot}} = d\vec{\Omega} \times \vec{G}$.
		\[
		\left(\frac{d\vec{G}}{dt}\right)_{\text{space}} = \left(\frac{d\vec{G}}{dt}\right)_{\text{body}} + \vec{\omega} \times \vec{G}
		\]
		where $\vec{\omega}dt = d\vec{\Omega}$ and $\vec{\omega}$ is along the instantaneous axis of rotation.
		
		We have $dG_i = a_{ji}dG'_j + da_{ji}G'_j$.
		
		where $a_{ji}G'_j = dG_i$ and $d a_{ji} = (\tilde{\epsilon})_{ij} = -\epsilon_{ij}$.
		
		By antisymmetry of $\epsilon$, $-\epsilon_{ij} = -\epsilon_{ijk}d\Omega_k = \epsilon_{ikj}d\Omega_k$.
		\[
		\Rightarrow dG_i = dG'_i + \epsilon_{ikj} d\Omega_k G_j
		\]
		\[
		dG_i = dG'_i + (d\vec{\Omega} \times \vec{G})_i
		\]
		
		\item We consider the transformation $A=BCD$.
		
		where $\omega_\phi = \dot{\phi}$, $\omega_\theta=\dot{\theta}$, $\omega_\psi=\dot{\psi}$.
		\[
		\Rightarrow 
		\begin{cases}
			(\omega_\phi)x' = \dot{\phi}\sin\theta\sin\psi, \quad (\omega_\phi)y' = \dot{\phi}\sin\theta\cos\psi, \quad (\omega_\phi)z' = \dot{\phi}\cos\theta \\
			(\omega_\theta)x' = \dot{\theta}\cos\psi, \quad (\omega_\theta)y' = -\dot{\theta}\sin\psi, \quad (\omega_\theta)z' = 0 \\
			(\omega_\psi)z' = \dot{\psi}
		\end{cases}
		\]
		\[
		\Rightarrow
		\begin{cases}
			\omega_{x'} = (\omega_\phi)x' + (\omega_\theta)x' \\
			\omega_{y'} = (\omega_\phi)y' + (\omega_\theta)y' \\
			\omega_{z'} = (\omega_\phi)z' + (\omega_\psi)z'
		\end{cases}
		\]
	\end{enumerate}
	
	\subsection*{4.10 The Coriolis Effect}
	\begin{enumerate}
		\item We have $\vec{v}_s = \vec{v}_r + \vec{\omega} \times \vec{r}$ where $\vec{v}_s, \vec{v}_r$ are the velocities of the particle relative to the space and rotating set of axes.
		\[
		\Rightarrow \left(\frac{d\vec{v}_s}{dt}\right)_s = \vec{a}_s = \left(\frac{d}{dt}(\vec{v}_r + \vec{\omega}\times\vec{r})\right)_s + \vec{\omega} \times \vec{v}_s
		\]
		\[
		= \vec{a}_r + 2(\vec{\omega}\times\vec{v}_r) + \vec{\omega}\times(\vec{\omega}\times\vec{r})
		\]
		The equation of motion $\vec{F} = m\vec{a}_s$.
		\[
		\Rightarrow \vec{F} - 2m(\vec{\omega}\times\vec{v}_r) - m\vec{\omega}\times(\vec{\omega}\times\vec{r}) = m\vec{a}_r = \vec{F}_{\text{eff}}
		\]
		where $-2m(\vec{\omega}\times\vec{v}_r)$ is the Coriolis effect and $-m\vec{\omega}\times(\vec{\omega}\times\vec{r})$ $\{m\omega^2 r \sin\theta\}$ is the centrifugal force.
		
		\begin{figure}[h]
			\centering
			\includegraphics[width=0.7\linewidth]{figure6}
			\caption{}
			\label{fig:figure6}
		\end{figure}
		
		
		\item Example: the equation of motion:
		\[
		m\frac{d^2x}{dt^2} = -2m(\vec{\omega}\times\vec{v}_r)_x = -2m\omega v_z \sin\theta
		\]
		where $v_z = -gt$, $t = \sqrt{2z/g}$.
		
		The deflection caused by Coriolis effect on $v_z$ is
		\[
		x = \frac{\omega g}{3} t^3 \sin\theta = \frac{\omega}{3}\sqrt{\frac{8z^3}{g}} \sin\theta
		\]
		Normally, for the equator $(\theta = \pi/2)$ and $z=100$m,
		\[
		x \approx 2.2 \text{ cm}
		\]
	\end{enumerate}
\end{document}