\documentclass[12pt]{article}
\usepackage{amsmath}
\usepackage{amssymb}
\usepackage{graphicx}
\usepackage{physics}
\usepackage[a4paper, margin=1in]{geometry}

\begin{document}
	
	\section*{VI: Oscillations}
	
	\subsection*{6.1 Formulation of the Problem}
	
	\begin{enumerate}
		\item We consider conservative systems where potential energy, $V$, is a function of position only. The system is in equilibrium when the generalized forces are zero:
		\[
		Q_i = -\left(\frac{\partial V}{\partial q_i}\right)_0 = 0
		\]
		A stable equilibrium results if a small disturbance of the system from the equilibrium position results only in small, bounded motion. An unstable equilibrium position, if disturbed by an infinitesimal amount, eventually produces unbounded motion.
		
		\item The deviations of the generalized coordinates from equilibrium will be denoted by $\eta_i$:
		\[
		q_i = q_{i0} + \eta_i
		\]
		We expand $V$ in a Taylor series:
		\begin{align*}
			V(q_1, \dots, q_{3n}) = V(q_{01}, \dots, q_{0n}) &+ \sum_i \left(\frac{\partial V}{\partial q_i}\right)_0 \eta_i \\
			&+ \frac{1}{2} \sum_{i,j} \left(\frac{\partial^2 V}{\partial q_i \partial q_j}\right)_0 \eta_i \eta_j + \cdots
		\end{align*}
		
		\begin{figure}[h]
			\centering
			\includegraphics[width=0.7\linewidth]{figure1}
			\caption{}
			\label{fig:figure1}
		\end{figure}
		
		By the equilibrium condition, the term linear in $\eta_i$ vanishes, and we are left with the quadratic terms as the first approximation to $V$.
		\[
		V = \frac{1}{2} \sum_{i,j} V_{ij} \eta_i \eta_j
		\]
		By symmetry, the coefficients $V_{ij} = \left(\frac{\partial^2 V}{\partial q_i \partial q_j}\right)_0$ are symmetric:
		\[
		V_{ij} = V_{ji}
		\]
		
		\item For the kinetic energy:
		\[
		T = \frac{1}{2} \sum_{i,j} m_{ij} \dot{q}_i \dot{q}_j = \frac{1}{2} \sum_{i,j} m_{ij} \dot{\eta}_i \dot{\eta}_j
		\]
		where $m_{ij}$ are in general functions of the $q_k$.
		\[
		m_{ij}(q_1, \dots, q_{3n}) = m_{ij}(q_{01}, \dots, q_{0n}) + \sum_k \left(\frac{\partial m_{ij}}{\partial q_k}\right)_0 \eta_k + \cdots
		\]
		We denote the constant values of the $m_{ij}$ functions by $T_{ij}$. Therefore,
		\[
		T = \frac{1}{2} \sum_{i,j} T_{ij} \dot{\eta}_i \dot{\eta}_j \quad \text{where} \quad T_{ij} = T_{ji}
		\]
		Thus, the Lagrangian is:
		\[
		L = T - V = \frac{1}{2} \sum_{i,j} (T_{ij} \dot{\eta}_i \dot{\eta}_j - V_{ij} \eta_i \eta_j)
		\]
		The equations of motion are:
		\[
		\sum_j (T_{ij} \ddot{\eta}_j + V_{ij} \eta_j) = 0
		\]
		With no cross terms, the Lagrangian simplifies to:
		\[
		L = \frac{1}{2} \sum_j (T_j \dot{\eta}_j^2 - V_j \eta_j^2)
		\]
		This leads to uncoupled equations:
		\[
		T_j \ddot{\eta}_j + V_j \eta_j = 0
		\]
	\end{enumerate}
	
	\subsection*{6.2 The Eigenvalue Equation and the Principal Axis Transformation}
	\begin{enumerate}
		\item We try an oscillatory solution of the form:
		\[
		\eta_j = a_j e^{-i\omega t}
		\]
		which will have a nontrivial solution only if the determinant of the coefficients vanishes:
		\[
		| \mathbf{V} - \omega^2 \mathbf{T} | = 0
		\]
		For a two-dimensional system, this is:
		\[
		\begin{vmatrix}
			V_{11} - \omega^2 T_{11} & V_{12} - \omega^2 T_{12} \\
			V_{21} - \omega^2 T_{21} & V_{22} - \omega^2 T_{22}
		\end{vmatrix} = 0
		\]
		The equations may be written in matrix form. Let $\lambda = \omega^2$ and $\mathbf{a}$ be a column vector of the amplitudes $a_j$:
		\[
		\mathbf{V}\mathbf{a} = \lambda \mathbf{T}\mathbf{a}
		\]
		The transposed complex conjugate equation (where $\mathbf{a}^\dagger$ is the adjoint vector) is:
		\[
		\mathbf{a}^\dagger \mathbf{V} = \lambda^* \mathbf{a}^\dagger \mathbf{T}
		\]
		Left-multiplying the first equation by $\mathbf{a}^\dagger$, we get:
		\[
		\mathbf{a}^\dagger \mathbf{V} \mathbf{a} = \lambda \mathbf{a}^\dagger \mathbf{T} \mathbf{a}
		\]
		This leads to:
		\[
		0 = (\lambda - \lambda^*) \mathbf{a}^\dagger \mathbf{T} \mathbf{a}
		\]
		When $\mathbf{a}^\dagger \mathbf{T} \mathbf{a} > 0$, it becomes $\lambda = \lambda^*$, which means $\lambda$ is real. We can separate $\mathbf{a}$ into its real and imaginary components, $\mathbf{a} = \mathbf{a}_R + i\mathbf{a}_I$. Then:
		\[
		\mathbf{a}^\dagger \mathbf{T} \mathbf{a} = (\mathbf{a}_R^T - i\mathbf{a}_I^T) \mathbf{T} (\mathbf{a}_R + i\mathbf{a}_I) = \mathbf{a}_R^T \mathbf{T} \mathbf{a}_R + \mathbf{a}_I^T \mathbf{T} \mathbf{a}_I
		\]
		where the imaginary term vanishes.
		
		\item In terms of a column matrix $\boldsymbol{\eta}$, the kinetic energy is $T = \frac{1}{2} \dot{\boldsymbol{\eta}}^T \mathbf{T} \dot{\boldsymbol{\eta}}$. The eigenvalues $\lambda$ are real and given by the Rayleigh quotient:
		\[
		\lambda = \frac{\mathbf{a}^T \mathbf{V} \mathbf{a}}{\mathbf{a}^T \mathbf{T} \mathbf{a}}
		\]
		By the equations given, and in view of the reality of the eigenvalues and eigenvectors, we have for each eigenvector $\mathbf{a}_k$:
		\[
		\mathbf{V} \mathbf{a}_k = \lambda_k \mathbf{T} \mathbf{a}_k
		\]
		If all the roots of the secular equation are distinct, the eigenvectors are orthogonal with respect to $\mathbf{T}$:
		\[
		\mathbf{a}_l^T \mathbf{T} \mathbf{a}_k = 0, \quad l \neq k
		\]
		We can further require that they are normalized such that:
		\[
		\mathbf{a}_k^T \mathbf{T} \mathbf{a}_k = 1
		\]
		This uniquely fixes the one arbitrary component of each of the $n$ eigenvectors. If $\mathbf{A}$ is the matrix whose columns are the eigenvectors $\mathbf{a}_k$, these conditions can be written as:
		\[
		\mathbf{A}^T \mathbf{T} \mathbf{A} = \mathbf{I}
		\]
		If $\lambda_l = \lambda_k$, we have the case of degeneracy. We introduce the congruence transformation of a matrix $\mathbf{C}$:
		\[
		\mathbf{C}' = \mathbf{A}^T \mathbf{C} \mathbf{A}
		\]
		The set of eigenvalue equations can be written as $\mathbf{V}\mathbf{A} = \mathbf{T}\mathbf{A}\mathbf{\Lambda}$, where $\mathbf{\Lambda}$ is a diagonal matrix with elements $\lambda_k \delta_{kl}$.
		\[
		\Rightarrow \quad \mathbf{A}^T \mathbf{V} \mathbf{A} = \mathbf{A}^T \mathbf{T} \mathbf{A} \mathbf{\Lambda}
		\]
		Using the normalization condition, this yields:
		\[
		\mathbf{A}^T \mathbf{V} \mathbf{A} = \mathbf{\Lambda}
		\]
		
		\item As an example, we consider a particle of mass $m$ with two degrees of freedom $(x_1, x_2)$. We use Cartesian coordinates so that $T_{ij} = m\delta_{ij}$, which results in $\mathbf{A}^T\mathbf{A}=\mathbf{I}$ (for $m=1$) and $\mathbf{A}^T\mathbf{V}\mathbf{A} = \mathbf{\Lambda}$ (diagonal). The Lagrangian is:
		\[
		L = \frac{m}{2}(\dot{x}_1^2 + \dot{x}_2^2) - \frac{1}{2}(V_{11}x_1^2 + 2V_{12}x_1x_2 + V_{22}x_2^2)
		\]
		The secular equation is $|\mathbf{V} - \lambda\mathbf{I}| = 0$ (absorbing $m$ into $\lambda$ for simplicity):
		\[
		\begin{vmatrix}
			V_{11} - \lambda & V_{12} \\
			V_{21} & V_{22} - \lambda
		\end{vmatrix} = 0
		\]
		This gives the eigenvalues:
		\[
		\lambda_{1,2} = \frac{1}{2}(V_{11} + V_{22}) \pm \frac{1}{2}\sqrt{(V_{11} - V_{22})^2 + 4V_{12}^2}
		\]
		\textbf{Case 1:} $V_{11} > V_{22} > 0$ and $|V_{12}|$ is small. We write the small quantity $\delta = \frac{V_{12}^2}{V_{11} - V_{22}}$. The eigenvalues are, to first order in $\delta$:
		\begin{align*}
			\lambda_1 &= V_{11} + \delta \\
			\lambda_2 &= V_{22} - \delta
		\end{align*}
		whose eigenvectors are, to lowest order:
		\[
		\mathbf{a}_1 = \begin{pmatrix} 1 \\ \frac{V_{12}}{V_{11}-V_{22}} \end{pmatrix}, \quad \mathbf{a}_2 = \begin{pmatrix} -\frac{V_{12}}{V_{11}-V_{22}} \\ 1 \end{pmatrix}
		\]
		\textbf{Case 2:} $V_{12}^2 \gg (V_{11}-V_{22})^2$. Let $\epsilon = \frac{V_{11}-V_{22}}{V_{12}}$. To first order in $\epsilon$, the eigenvalues are:
		\begin{align*}
			\lambda_1 &\cong \frac{1}{2}(V_{11}+V_{22}) + V_{12} + \frac{1}{4}(V_{11}-V_{22})\epsilon \\
			\lambda_2 &\cong \frac{1}{2}(V_{11}+V_{22}) - V_{12} - \frac{1}{4}(V_{11}-V_{22})\epsilon
		\end{align*}
		whose eigenvectors are, to lowest order in $\epsilon$, given by the columns of the matrix $\mathbf{A}$:
		\[
		\mathbf{A} = \begin{pmatrix}
			\frac{1}{\sqrt{2}}(1 - \frac{1}{4}\epsilon) & \frac{1}{\sqrt{2}}(1 + \frac{1}{4}\epsilon) \\[1em]
			\frac{1}{\sqrt{2}}(1 + \frac{1}{4}\epsilon) & -\frac{1}{\sqrt{2}}(1 - \frac{1}{4}\epsilon)
		\end{pmatrix}
		\]
		\begin{figure}[h]
			\centering
			\includegraphics[width=0.7\linewidth]{figure2}
			\caption{}
			\label{fig:figure2}
		\end{figure}
	\end{enumerate}
	\section{Frequencies of Free Vibration and Normal Coordinates}
	
	The solutions of the secular equation, $|V - \omega^2 T| = 0$, are designated as $\omega_1^2, \omega_2^2, \dots, \omega_n^2$, which are the frequencies of free vibration.
	
	The general solution for the displacement coordinates $\eta_j$ is a superposition of all the normal modes. The motion in a given mode $k$ is described by $a_{jk} e^{-i\omega_k t}$. The general solution can be written in real form as:
	\[
	\eta_j(t) = \sum_k A_k \cos(\omega_k t + \delta_k)
	\]
	or using complex notation:
	\[
	\eta_j(t) = \text{Re} \left[ \sum_k C_k a_{jk} e^{-i\omega_k t} \right]
	\]
	where $a_{jk}$ are the components of the real, normalized eigenvectors.
	
	At $t=0$, the initial displacements and velocities are:
	\begin{align*}
		\eta_j(0) &= \text{Re} \left[ \sum_k C_k a_{jk} \right] = \sum_k a_{jk} \text{Re}(C_k) \\
		\dot{\eta}_j(0) &= \text{Re} \left[ \sum_k (-i\omega_k) C_k a_{jk} \right] = \sum_k a_{jk} \omega_k \text{Im}(C_k)
	\end{align*}
	The coefficients can be found by using the orthogonality relations of the eigenvectors. For column vectors $a_k$, the relations are $\tr{a_k} T a_l = \delta_{kl}$ and $\tr{a_k} V a_l = \omega_k^2 \delta_{kl}$.
	
	\subsection*{Normal Coordinates}
	We define a new set of coordinates $\xi_j$, called normal coordinates, through the linear transformation defined by the eigenvectors:
	\[
	\eta_j = \sum_k a_{jk} \xi_k \quad \text{or in matrix form} \quad \eta = A \xi
	\]
	where $A$ is the matrix whose columns are the eigenvectors $a_k$.
	
	The potential energy in these new coordinates is:
	\[
	V = \frac{1}{2} \tr{\eta} V \eta = \frac{1}{2} \tr{(A\xi)} V (A\xi) = \frac{1}{2} \tr{\xi} (\tr{A} V A) \xi
	\]
	The matrix $\tr{A}VA$ is diagonal, with the eigenvalues $\omega_k^2$ as its diagonal elements. Thus:
	\[
	V = \frac{1}{2} \sum_k \omega_k^2 \xi_k^2
	\]
	The kinetic energy is:
	\[
	T = \frac{1}{2} \tr{\dot{\eta}} T \dot{\eta} = \frac{1}{2} \tr{(\dot{A\xi})} T (A\dot{\xi}) = \frac{1}{2} \tr{\dot{\xi}} (\tr{A} T A) \dot{\xi}
	\]
	The orthogonality condition for the eigenvectors is $\tr{A}TA = I$ (the identity matrix). So the kinetic energy simplifies to:
	\[
	T = \frac{1}{2} \sum_k \dot{\xi}_k^2
	\]
	The Lagrangian $L = T-V$ in terms of normal coordinates is:
	\[
	L = \frac{1}{2} \sum_k (\dot{\xi}_k^2 - \omega_k^2 \xi_k^2)
	\]
	The Lagrangian is a sum of independent terms, one for each normal coordinate. The Euler-Lagrange equation for each $\xi_k$ gives:
	\[
	\ddot{\xi}_k + \omega_k^2 \xi_k = 0
	\]
	This is the equation for a simple harmonic oscillator, with the solution:
	\[
	\xi_k(t) = C_k e^{-i\omega_k t}
	\]
	Each normal coordinate $\xi_k$ is a simply periodic function involving only one of the resonant frequencies. The set $\{\xi_k\}$ are the normal coordinates of the system.
	
	\section{Example: Free Vibrations of a Linear Triatomic Molecule}
	
	\begin{figure}[h]
		\centering
		\includegraphics[width=0.7\linewidth]{figure3}
		\caption{}
		\label{fig:figure3}
	\end{figure}
	
	We consider a linear molecule with masses $m$, $M$, and $m$ connected by springs of constant $k$. The equilibrium distance between adjacent masses is $b$.
	
	The potential energy is given by:
	\[
	V = \frac{k}{2} [(x_2-x_1)-b]^2 + \frac{k}{2} [(x_3-x_2)-b]^2
	\]
	We introduce displacement coordinates $\eta_j = x_j - x_{0j}$ from the equilibrium positions $x_{0j}$. At equilibrium, $x_{02}-x_{01}=b$ and $x_{03}-x_{02}=b$.
	In terms of these coordinates, the potential energy is:
	\[
	V = \frac{k}{2} [(\eta_2 - \eta_1)^2 + (\eta_3 - \eta_2)^2] = \frac{k}{2} (\eta_1^2 + 2\eta_2^2 + \eta_3^2 - 2\eta_1\eta_2 - 2\eta_2\eta_3)
	\]
	The potential energy matrix $V$ is therefore:
	\[
	V = k
	\begin{pmatrix}
		1 & -1 & 0 \\
		-1 & 2 & -1 \\
		0 & -1 & 1
	\end{pmatrix}
	\]
	The kinetic energy is $T = \frac{1}{2} m\dot{\eta}_1^2 + \frac{1}{2} M\dot{\eta}_2^2 + \frac{1}{2} m\dot{\eta}_3^2$. The kinetic energy matrix $T$ is diagonal:
	\[
	T =
	\begin{pmatrix}
		m & 0 & 0 \\
		0 & M & 0 \\
		0 & 0 & m
	\end{pmatrix}
	\]
	The secular equation $|V - \omega^2 T|=0$ is:
	\[
	\begin{vmatrix}
		k-\omega^2 m & -k & 0 \\
		-k & 2k-\omega^2 M & -k \\
		0 & -k & k-\omega^2 m
	\end{vmatrix}
	= 0
	\]
	Solving this determinant yields a cubic equation for $\omega^2$:
	\[
	(k-\omega^2 m) [(2k-\omega^2 M)(k-\omega^2 m) - k^2] - k^2(\omega^2 m - k) = 0
	\]
	\[
	(k-\omega^2 m) \left[ (2k-\omega^2 M)(k-\omega^2 m) - 2k^2 \right] = 0
	\]
	Which leads to the solutions for the normal mode frequencies:
	\begin{align*}
		\omega_1^2 &= 0 \quad (\text{Translational mode}) \\
		\omega_2^2 &= \frac{k}{m} \quad (\text{Asymmetric stretch mode}) \\
		\omega_3^2 &= \frac{k}{m} \left( 1 + \frac{2m}{M} \right) \quad (\text{Symmetric stretch mode})
	\end{align*}
	
	\subsection*{Eigenvectors and Normal Coordinates}
	The components $a_{ij}$ of the eigenvectors are determined by $(V - \omega_j^2 T)a_j = 0$ along with the normalization condition $\tr{a_j} T a_j = 1$.
	
	\paragraph{Mode 1 ($\omega_1^2 = 0$):} This corresponds to pure translation of the molecule. The components are equal, $a_{11}=a_{21}=a_{31}$. Normalization $m(a_{11}^2 + a_{31}^2) + M a_{21}^2 = 1$ gives:
	\[
	a_{11} = \frac{1}{\sqrt{2m+M}}, \quad a_{21} = \frac{1}{\sqrt{2m+M}}, \quad a_{31} = \frac{1}{\sqrt{2m+M}}
	\]
	
	\paragraph{Mode 2 ($\omega_2^2 = k/m$):} In this mode, the center atom is at rest, while the two outer atoms vibrate exactly out of phase. The eigenvector components are:
	\[
	a_{12} = \frac{1}{\sqrt{2m}}, \quad a_{22} = 0, \quad a_{32} = \frac{-1}{\sqrt{2m}}
	\]
	
	\paragraph{Mode 3 ($\omega_3^2 = k/m(1+2m/M)$):} This is the "breathing" mode where the outer atoms move in phase with each other, but out of phase with the center atom. The components are:
	\[
	a_{13} = \frac{1}{\sqrt{2m(1+2m/M)}}, \quad a_{23} = \frac{-2m/M}{\sqrt{2m(1+2m/M)}}, \quad a_{33} = \frac{1}{\sqrt{2m(1+2m/M)}}
	\]
	
	\begin{figure}[h]
		\centering
		\includegraphics[width=0.7\linewidth]{figure4}
		\caption{}
		\label{fig:figure4}
	\end{figure}
	
	The normal coordinates may be found by inverting the transformation, $\xi = \tr{A} T \eta$:
	\begin{align*}
		\xi_1 &= \frac{1}{\sqrt{2m+M}} (m\eta_1 + M\eta_2 + m\eta_3) \quad (\text{proportional to center of mass position}) \\
		\xi_2 &= \sqrt{\frac{m}{2}} (\eta_1 - \eta_3) \\
		\xi_3 &= \sqrt{\frac{mM}{2(M+2m)}} \left( \eta_1 + \eta_3 - \frac{2}{m} \eta_2 \right) \quad (\text{Note: simplified form})
	\end{align*}
	In the actual molecule, there will also be normal modes of vibration perpendicular to the axis (bending modes).
	\section*{6.5 Forced Vibrations and the Effect of Dissipative Forces}
	
	\subsection*{1. Forced Undamped Vibrations}
	If $F_j$ is the generalized force corresponding to the coordinate $\eta_j$, then the generalized force $Q_i$ for the normal coordinate $\xi_i$ is:
	$$
	Q_i = a_{ji} F_j
	$$
	The equations of motion in normal coordinates become:
	$$
	\ddot{\xi}_i + \omega_i^2 \xi_i = Q_i
	$$
	For a sinusoidal driving force, $Q_i$ can be written as:
	$$
	Q_i = Q_{0i} \cos(\omega t + \delta_i)
	$$
	Thus, the equation of motion is:
	$$
	\ddot{\xi}_i + \omega_i^2 \xi_i = Q_{0i} \cos(\omega t + \delta_i)
	$$
	The particular solution has the form $\xi_i = B_i \cos(\omega t + \delta_i)$, where substituting into the equation gives the amplitude $B_i$:
	$$
	B_i = \frac{Q_{0i}}{\omega_i^2 - \omega^2}
	$$
	The complete motion for the original coordinates $\eta_j$ is then found by transforming back:
	$$
	\eta_j = a_{ji} \xi_i = a_{ji} \frac{Q_{0i}}{\omega_i^2 - \omega^2} \cos(\omega t + \delta_i)
	$$
	The closer the driving frequency $\omega$ approaches any natural frequency $\omega_i$, the stronger the phenomenon of \textbf{resonance}.
	
	\subsection*{2. Damped Vibrations and Rayleigh's Dissipation Function}
	We now consider dissipative forces. Let $\mathcal{F}$ be Rayleigh's dissipation function, which is a homogeneous quadratic function of the velocities:
	$$
	\mathcal{F} = \frac{1}{2} F_{ij} \dot{\eta}_i \dot{\eta}_j
	$$
	By symmetry, the coefficients of the dissipation tensor are symmetric, $F_{ij} = F_{ji}$. The complete set of Lagrange's equations of motion for free damped vibrations is:
	$$
	T_{ij}\ddot{\eta}_j + F_{ij}\dot{\eta}_j + V_{ij}\eta_j = 0
	$$
	When such simultaneous diagonalization of the kinetic energy ($T$), potential energy ($V$), and dissipation ($\mathcal{F}$) tensors is feasible, the equations of motion are decoupled in the normal coordinates, with the form:
	$$
	\ddot{\xi}_i + \mathcal{F}_i \dot{\xi}_i + \omega_i^2 \xi_i = 0 \quad (\text{no summation})
	$$
	where $\mathcal{F}_i$ are the nonnegative coefficients in the diagonalized form of $\mathcal{F}$. Seeking a solution of the form $\xi_i = C_i e^{-i\omega'_i t}$, we find the characteristic equation for the complex frequency $\omega'_i$:
	$$
	-(\omega'_i)^2 - i\omega'_i \mathcal{F}_i + \omega_i^2 = 0
	$$
	This quadratic equation has two solutions for $\omega'_i$:
	$$
	\omega'_i = \pm \sqrt{\omega_i^2 - \frac{\mathcal{F}_i^2}{4}} - \frac{i\mathcal{F}_i}{2}
	$$
	If the dissipation is small, the frequency of oscillation reduces from the friction-free value, and the complete motion is then simply an exponential damping of the free modes of vibration:
	$$
	\xi_i = C_i e^{-\mathcal{F}_i t/2} e^{-i \left( \sqrt{\omega_i^2 - \mathcal{F}_i^2/4} \right) t}
	$$
	For the general coupled system, suppose we seek a solution of the form $\eta_j = C a_j e^{-i\omega t}$. This leads to a set of simultaneous linear equations:
	$$
	V_{ij}a_j - i\omega F_{ij}a_j - \omega^2 T_{ij}a_j = 0
	$$
	To write this as a polynomial in $\gamma = -i\omega$, we have:
	$$
	(V_{ij} + \gamma F_{ij} + \gamma^2 T_{ij}) a_j = 0
	$$
	In matrix form, this is written as:
	$$
	\mathbf{V a} + \gamma \mathbf{F a} + \gamma^2 \mathbf{T a} = 0
	$$
	
	\subsection*{3. Forced Damped Oscillations}
	We finally consider forced sinusoidal oscillations in the presence of dissipative forces. The driving force is taken to be complex:
	$$
	F_j = F_{0j} e^{-i\omega t}
	$$
	The equations of motion are:
	$$
	V_{ij}\eta_j + F_{ij}\dot{\eta}_j + T_{ij}\ddot{\eta}_j = F_{0j} e^{-i\omega t}
	$$
	We seek a particular (steady-state) solution of the form $\eta_j = A_j e^{-i\omega t}$. Substituting this into the equation of motion, we obtain a set of inhomogeneous linear equations for the complex amplitudes $A_j$:
	$$
	(V_{ij} - i\omega F_{ij} - \omega^2 T_{ij}) A_j - F_{0j} = 0
	$$
	From Cramer's rule, the solution for each amplitude $A_j$ is given by the ratio of two determinants:
	$$
	A_j = \frac{D_j(\omega)}{D(\omega)}
	$$
	where $D(\omega)$ is the determinant of the coefficients of $A_j$:
	$$
	D(\omega) = \det(\mathbf{V} - i\omega \mathbf{F} - \omega^2 \mathbf{T})
	$$
	and $D_j(\omega)$ is the determinant of the same matrix, but with the $j$-th column replaced by the vector of driving forces $F_{0j}$.
	
	Since the matrices $\mathbf{V}$, $\mathbf{F}$, and $\mathbf{T}$ are real and symmetric, it can be shown that if $\omega_k$ is a complex root of $D(\omega) = 0$, then $-\omega_k^*$ is also a root. This allows $D(\omega)$ to be written in a factored form involving its complex roots. The magnitude squared of the determinant, which is related to the power absorption, can be found by rationalizing the expression:
	$$
	D^*(\omega) D(\omega)
	$$
	
	\section*{6.6 Beyond Small Oscillations: The Damped Driven Pendulum and the Josephson Junction}

\end{document}